\documentclass[14pt]{article}
\usepackage{researchPaper}

\title{Microwave Imaging Cheat Sheet}
\begin{document}
	\maketitle
		
		%Data Independent radar beamforming algorithms for breast cancer detection
		%	Byrne, O'Halloran, Glavin, Jones
		\section*{Delay and Sum (DAS) Beamformer}	
			The power associated with a given focal point (from the phase center of 
			the array) for a Delay and Sum Beamformer is given by:
			\begin{align}
				I(\bm{r}) &= \int_0^{T_{win}} [ \sum_{i=1}^M w_n S_i(t - \tau_i(\bm{r}))]^2 dt
			\end{align}
			where:
				\begin{itemize}
					\item	M:	Number of antennas
					\item	$S_i$:	Backscattered signal at antenna $i$
					\item	$\bm{r}$:	focal point $\bm{r} = (x,y,z)$	
					\item	$\tau_i(\bm{r}) = \frac{2d_i(\bm{r})}{v_p T_s}$:	$i$th discrete time delay
					\item	$d_n(\bm{r}) = |\bm{r} - \bm{r}_i|$
					\item	$\bm{r}_i$:	location of the $i$th antenna
					\item	$v_p$:	average velocity of the wave 
					\item	$T_{win}$:	length of the window
					\item	$T_s$:	sampling interval
					\item	$w_n$:	weight that can be applied to the signal, or just set to 1
				\end{itemize}
			
			An improved version of the delay and sum algorithm uses a quality factor,
			which measures the coherence of ultrawideband (UWB) scattering at a 
			particular focal point.  At the focal point $(\bm{r})$ energy is collected
			across the window for each multistatic signal and stored.  The energy is
			then cumuatively summed and plotted against the number of channels used.
			A second order polynomial is fitted to that curve, with the quadratic term
			taken to be the quality factor.  The improved algorithm is then given by
			\begin{align}
				I(\bm{r}) &= QF(\bm{r}) \int_0^{T_{win}} [\sum_{i=1}^{M(M+1)/2} w_i S_i(t - \tau_i(\bm{r}))]^2 dt
			\end{align}
	

		\section*{Robust Capon (RCB) Beamformer}
			%Zhisong Wang, Jian Li, Renbiao Wu
			%TimeDelay and TimeReversal Robust Capon Beamformers for Ultrasound Imaging
		\section*{Amplitude and Phase Estimation (APES)}
		\section*{Multistatic Adaptive Microwave Imaging (MAMI)}
	
\end{document}


