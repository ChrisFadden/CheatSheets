\documentclass[14pt]{extarticle}
\usepackage{researchPaper}
\usepackage{outlines}
\usepackage{mathrsfs}
\def\Definition{{\color{blue} \textbf{Definition:} }}
\def\Theorem{{\color{red} \textbf{Theorem:} }}
\newcommand*\pFq[2]{{}_{#1}F_{#2}}
%Hartshorne
\title{EE 550 Cheat Sheet}
\begin{document}
	\maketitle	
	\begin{outline}	
	\section*{Normed Spaces}	
		\1 \Definition \textbf{Norm}
			\2	$\norm{~\cdot~} : V \rightarrow [0,\infty)$
			\2	$\norm{x} = 0$ iff $x = 0_V$
			\2	$\norm{\alpha x} = \abs{\alpha} \norm{x}$
			\2	$\norm{x \oplus y} \le \norm{x} + \norm{y}$
		\1	\Theorem Norm is a uniformly continuous function
		\1	\Definition \textbf{Lipschitz Continuous}
			\2	$\norm{f(x) - f(y)}_W \le c\norm{x-y}_V$
				\3	$c \in (0,\infty)$ 
		\1	\Theorem If Lipschitz then uniformly continuous
			\2	\textbf{Converse not true}
		\1	\Definition \textbf{Contraction Mapping}
			\2	$\norm{f(x) - f(y)}_V \le \rho\norm{x - y}_V$
				\3	$\rho \in (0,1)$
			\2	$f$ is a contraction
		\1	\Definition \textbf{Norm Equivalence}
			\2	$0 < a_m\norm{x}_a \le \norm{x}_b \le a_M \norm{x}_a$
				\3	$\norm{~\cdot~}_a$, $\norm{~\cdot~}_b$ are norms on $V$
				\3	$a_m$, $a_M \in (0,\infty)$
			\2	Equivalent norms generate the same topology
		\1	\Theorem On a finite-dimensional vector space, all norms are equivalent
	\section*{Summable Sequences and Integrable Functions}
		\1	\Definition $\ell_p$
			\2	$\sum_{k=1}^{\infty} \abs{x_k}^p < \infty$ $\forall~x \in \ell_p$
			\2	$\norm{x}_{\ell_p} = (\sum_{k=1}^{\infty} \abs{x_k}^p)^{1/p}$
		\1	\Definition $\ell_{\infty}$
			\2	$\sup_{k \in \mathbb{N}} \abs{x_k} < \infty$
			\2	$\norm{x}_{\ell_{\infty}} = \sup_{k \in \mathbb{N}} \abs{x_k}$
		\1	\Definition $L_p$ 
			\2	$L_p(T) = \{x : \int_T \abs{x(t)}^p dt < \infty\}$
			\2	$\norm{x}_{L_p} = (\int_T \abs{x(t)}^p)^{1/p}$
			\2	$p \in [1,\infty)$
		\1	\Definition \textbf{Essential Supremum}
			\2	$\text{ess } \sup x = \sup_{t \in \mathbb{R}} \abs{x(t)}$ a.e.
		\1	\Definition $L_{\infty}$
			\2	$L_{\infty}(\mu) = \{x : \text{ess } \sup x < \infty\}$
			\2	$\norm{x}_{L_{\infty}} = \text{ess } \sup x$
		\1	\Theorem $\abs{\int f(t) d\mu(t)} \le \int \abs{f(t)} d\mu(t)$ if $f \in L_1(\mu)$
		\1	\Theorem \textbf{Holder Inequalities}
			\2	$\frac{1}{p} + \frac{1}{q} = 1$
			\2	$\abs{\sum_{j=1}^n x_jy_j} \le \sum_{j=1}^n\abs{x_jy_j} \le (\sum_{j=1}^n\abs{x_j}^p)^{1/p}
					(\sum_{j=1}^n\abs{y_j}^q)^{1/q}$
				\3	Valid as $n \rightarrow \infty$ if $x \in \ell_p(\mu)$ and $y \in \ell_q(\mu)$
			\2	$\abs{\int_{t \in \Omega} x(t)y(t) d\mu(t)} \le 
				\int_{t \in \Omega}\abs{x(t)y(t)}d\mu(t) \le$
			\2 $(\int_{t \in \Omega} \abs{x(t)}^pd\mu(t))^{1/p}(\int_{t \in \Omega}\abs{y(t)}^qd\mu(t))^{1/q}$
				\3	$x \in L_p(\mu)$, $y \in L_q(\mu)$
		\1	\Theorem \textbf{Minkowski Inequality}
			\2	$(\sum_{j=1}^n \abs{x_j \pm y_j}^p)^{1/p} \le (\sum_{j=1}^n \abs{x_j}^p)^{1/p}
				+ (\sum_{j=1}^n \abs{y_j}^p)^{1/p}$
				\3	Extends to infinite sums when $x,y \in \ell_p(\mu)$
			\2	$(\int_{t\in\Omega}\abs{x(t) \pm y(t)}^p)^{1/p} \le $
			\2	$(\int_{t \in \Omega}\abs{x(t)}^p)^{1/p} + (\int_{t \in \Omega} \abs{y(t)}^p)^{1/p}$
				\3	$x,y \in L_p(\mu)$
		\1	\Theorem \textbf{Sequence Inclusion Theorem}
			\2	$\ell_1 \subsetneq \ell_p \subsetneq \ell_q \subsetneq \ell_{\infty}$
				\3	$1 < p < q < \infty$
		\1	\Theorem \textbf{Integral Inclusion Theorem}
			\2	$L_{\infty}(\nu) \subsetneq L_q(\nu) \subsetneq L_p(\nu) \subsetneq L_1(\nu)$
				\3	$\nu$ is a \textbf{finite measure}
				\3	$1 < p < q < \infty$
		\1	\Theorem \textbf{Intersection Inclusion Theorem}
			\2	$L_1(\mu) \cap L_{\infty}(\mu) \subseteq L_p(\mu)$ 
				\3	$\mu$ is finite or $\sigma$-finite
	\section*{Completion of Vector Spaces}
		\1	\Definition \textbf{Schauder Basis}
			\2	$\norm{(x - \sum_{k=1}^n \alpha_k e^k)} \rightarrow 0$ as $n \rightarrow \infty$
			\2	$\{e^k\}$ is the Schauder basis
		\1	\Definition \textbf{Banach Space}
			\2	A complete normed vector space
		\1	\Theorem Every Cauchy sequence is bounded in a normed space
		\1	\Theorem A finite dimensional normed vector space over a complete field
				is a Banach space
		\1	\Theorem $\ell_p$ over a complete field is complete for $p \in [1,\infty]$
		\1	\Definition \textbf{Space of Convergent Sequences} $c_0$
			\2	Subspace of $\ell_{\infty}$ where all sequences converge
			\2	$c_0 \subseteq \ell_{\infty}$ where all sequences converge to zero
				\3	All sequences in $c$-space are bounded
		\1	\Theorem The space $c$ of all convergent sequences over a complete field
				is complete
		\1	\Definition \textbf{Space of Continuous Functions}
			\2	$C_p[a,b] \subsetneq L_p[a,b]$
			\2	Space of all continuous functions in $L_p$
		\1	\Theorem The space $C_{\infty}[a,b]$ is complete
			\2	$C_p[a,b]$ is not complete for $p \in [1,\infty)$
		\1	\Definition \textbf{Density in Vector Space}
			\2	$\forall~x \in V$, $\forall~\epsilon > 0$ $\exists~y \in S$ s.t. $\norm{x-y} < \epsilon$
			\2	$S \subseteq V$ is dense in $V$
		\1	\Theorem $\bar{S} = V$, i.e. closure of a dense subset is the set itself
		\1	\Definition \textbf{Separable Vector Space}
			\2	$V$ has a countable and dense subset (i.e. $\mathbb{Q} \subseteq \mathbb{R}$)
		\1	\Theorem $\ell_p$ is separable for $p \in [1,\infty)$
			\2	$\ell_{\infty}$ is not separable
	\section*{Inner Product Spaces}
		\1	\Definition \textbf{Inner Product}
			\2	$\braket{(x \oplus y)}{z} = \braket{x}{z} + \braket{y}{z}$
			\2	$\braket{\alpha x}{y} = \alpha \braket{x}{y}$
			\2	$\braket{x}{y} = \overline{\braket{x}{y}}$
			\2	$\braket{x}{x} > 0$ if $x \ne 0_V$
		\1	\Theorem \textbf{Cauchy Schwarz Inequality}
			\2	$\abs{\braket{x}{y}} \le \norm{x}\norm{y}$
				\3	Equality iff $x = \alpha y$ (colinear)
			\2	$\norm{x}^2 = \braket{x}{x}$
		\1	\Theorem Inner products are continuous functions
		\1	\Theorem Every inner product space is a normed space
			\2	Normed space is not necessarily an inner product space
			\2	Norm must satisfy parallelogram equality to be an inner product
		\1	\Theorem	\textbf{Parallelogram Equality}
			\2	$\norm{x \oplus y}^2 + \norm{x \oplus (-y)}^2 = 2\norm{x}^2 + 2\norm{y}^2$
		\1	\Theorem	\textbf{Polarization Identity}
			\2	$\braket{x}{y} = \frac{1}{4}(\norm{x \oplus y}^2 - \norm{x \oplus (-y)}^2 
			+ j\norm{x \oplus jy}^2 - j\norm{x \oplus j(-y)}^2)$
		\1	\Definition \textbf{Hilbert Space}
			\2	A complete inner product space
		\1	\Theorem Every inner product space is a metric space
	\section*{Orthogonality}
		\1	\Definition \textbf{Orthogonal}
			\2	$\braket{x}{y} = 0$ 
			\2	$x \perp y$
		\1	\Theorem \textbf{Pythagorean Theorem}
			\2	If $x \perp y$ then $\norm{x \oplus y}^2 = \norm{x}^2 + \norm{y}^2$
		\1	\Definition \textbf{Direct Sum}
			\2	$Y = U \bigoplus W$ if $\forall~y \in Y~\exists u \in U$ and $w \in W$
					such that $\exists!~y = u \oplus w$
			\2	$U$ is the algebraic complement of $W$
		\1	\Definition \textbf{Orthogonal Complement}
			\2	$M^{\perp} = \{x \in V~:~x \perp y~\forall y \in M\}$
		\1	\Theorem $M \cap M^{\perp} = 0_V$ if $0_V \in M$, and $\emptyset$ else
		\1	\Theorem $M \subseteq V$ then $M^{\perp}$ is a closed subspace of $(V,\braket{}{})$
		\1	\Theorem Properties of Orthogonal Complements
			\2	$S,T \subseteq V$
			\2	If $S \subseteq T$ then $T^{\perp} \subseteq S^{\perp}$ and $S^{\perp \perp} \subseteq T^{\perp \perp}$
			\2	$S \subseteq S^{\perp \perp}$
			\2	$\{0_V\}^{\perp} = V$, and $V^{\perp} = \{0_V\}$
			\2	If $S$ is dense in $V$, then $S^{\perp} = \{0_V\}$
		\1	\Theorem Properties of Hilbert Spaces and Orthogonal Complements
			\2	$G \subseteq H$
			\2	$\bar{G} = G^{\perp \perp}$
			\2	$G^{\perp} = \{0_V\}$ iff $G$ is dense in $H$
			\2	If $G$ is closed in $H$ and $G^{\perp} = \{0_V\}$ then $G = H$
		\1	\Theorem $F,G \le H$ closed.  If $F \perp G$ then $F \bigoplus G \le H$
		closed.
		\1	\Theorem \textbf{Projection Theorem}
			\2	$G \le H$ closed
			\2	$H = G \bigoplus G^{\perp}$
			\2	Each $x \in H$ can be uniquely expressed as $x = y \oplus z$ where
					$y \in G$ and $z \in G^{\perp}$ and $\norm{x}^2 = \norm{y}^2 + \norm{z}^2$
		\1	\Definition \textbf{Orthogonal Projection}
			\2	$\mathcal{R}(P) \perp \mathcal{N}(P)$
		\1	\Theorem Not all projections in an inner product space are orthogonal
		\1	\Theorem If $P$ is an orthogonal projection then so is $I - P$
		\1	\Theorem An orthogonal projection in an inner product space is continuous
		\1	\Theorem \textbf{Projection Theorem}
			\2	$G \le H$ closed
			\2	$\mathcal{R}(P)$ and $\mathcal{N}(P)$ are closed subspaces of $V$
			\2	$\mathcal{N}(P) = \mathcal{R}(P)^{\perp}$
		\1	\Definition \textbf{Minimizing Vector}
			\2	$W \le V$, and given arbitrary $y \in V$, if $\exists~z \in W$ s.t.
			\2	$\norm{(y - z)} \le \norm{(y - x)}$ $\forall~y \in V$ then $z \in W$ is unique
			\2	$z$ is minimizing vector iff $(y - z) \perp W$
		\1	\Theorem In a hilbert space, a minimizing vector in a closed subspace always exists.
	\section*{Fourier Theory}
		\1	\Definition \textbf{Orthonormal set}
			\2	$\braket{x^{\alpha}}{x^{\beta}} = \delta_{\alpha\beta}$
		\1	\Theorem Every orthonormal set of vectors in an inner product space is
				linearly independent
		\1	\Definition \textbf{Complete Orthonormal Set}
			\2	$x \perp x_{\alpha} = 0~\forall~\alpha \in I$ then $x = 0_V$
			\2	Complete orthonormal set in a Hilbert space is an orthonormal basis
			\2	Every orthornormal set of vectors in a Hilbert space can be extended to 
					forma an orthonormal basis
			\2	Two orthonormal bases of a Hilbert space have same cardinality
		\1	\Theorem \textbf{Bessel Inequality}
			\2	$\sum_k \abs{\braket{x}{x^k}}^2 \le \norm{x}^2$
		\1	\Theorem \textbf{Fourier Series Theorem}
			\2	$\{x^k\}$ is an orthonormal basis of $H$
			\2	Every $x \in H$ can be expanded as $\sum_k \braket{x}{x^k}x^k$
			\2	(Parseval) $\braket{x}{y} = \sum_k \braket{x}{x^k}\overline{\braket{y}{x^k}}$
			\2	$\norm{x}^2 = \sum_k \abs{\braket{x}{x^k}}^2 < \infty$
			\2	$\{x^k\} \subseteq V$, then $\bar{V} = H$
		\1	\Theorem Orthogonal Projection
			\2	$Px = \sum_k \braket{x}{x^k}x^k$ with $P : H \rightarrow V$, $\{x^k\} \in V$
		\1	\Theorem \textbf{Gram-Schmidt Orthonormalization}
			\2	$\{x^k\}$ countable set of linearly independent vectors in an inner product
					space.  Then $\exists$ an orthonormal sequence $\{e^k\} \subseteq V$
					s.t. $\text{Span}[e^1,...,e^n] = \text{Span}[x^1,...,x^n]$
		\1	\Definition \textbf{Fourier Transform}
			\2	$\hat{f}(\omega) = \frac{1}{\sqrt{2\pi}}\int_{\mathbb{R}} f(t)\exp(-j\omega t) dt$
				\3	$f \in L_1(\mu)$
			\2	$f(t) = \frac{1}{\sqrt{2\pi}}\int_{\mathbb{R}}\hat{f}(\omega)\exp(j\omega t) d\omega$
				\3	$\hat{f} \in L_1(\mu)$
		\1	\Theorem \textbf{Plancharel Theorem}
			\2	$f \in L_2(\mu)$ and $\hat{f} \in L_2(\mu)$
			\2	If $f \in L_1(\mu) \cap L_2(\mu)$ then $\hat{f}$ is the Fourier Transform of $f$
			\2	$\norm{\hat{f}}_{L_2} = \norm{f}_{L_2}$
			\2	$f \mapsto \hat{f}$ is a Hilbert space isomorphism from $L_2(\mu)$ to $L_2(\mu)$
		\1	\Definition \textbf{Hilbert Dimension}
			\2	Cardinality of an orthonormal basis of a Hilbert space
		\1	\Theorem $H$ is a separable Hilbert space iff it has a countable basis

	\end{outline}
\end{document}

