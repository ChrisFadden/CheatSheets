\documentclass[14pt]{extarticle}
\usepackage{researchPaper}
\usepackage{outlines}
\usepackage[dvipsnames]{xcolor}

\def\Definition{{\color{Blue} \textbf{Definition:} }}
\def\Theorem{{\color{Red} \textbf{Theorem:} }}

\title{Topology Cheat Sheet}
\begin{document}
	\maketitle


	\begin{outline}	
		\section{Sets and Functions}
		\1	\Definition \textbf{Set}
		\1	\Definition \textbf{Demorgan's Laws}
		\1	\Definition \textbf{Cartesian Product}
		\1	\Definition \textbf{Function (Mapping)}
			\2	Composition
			\2	Injective
			\2	Surjective
		\1	\Definition \textbf{Bijective (Invertible)}
		\1	\Definition \textbf{Ordered Sets}
		\1	\Definition \textbf{Bounded Set (Supremum and Infimum)}
		\1	\Definition \textbf{Sequence}
			\2	Strictly increasing
			\2	Strictly decreasing
		\1	\Theorem \textbf{Increasing Sequences converge to Supremum}
		\section{Metric Spaces}
		\1	\Definition \textbf{Metric Space}
		\1	\Definition \textbf{Discrete Metric}
		\1	\Theorem \textbf{Cauchy Schwarz Inequality}
		\1	\Theorem \textbf{Subsets inherit the metric}
		\1	\Definition \textbf{Open/Closed Balls}
		\1	\Definition \textbf{Open/Closed Sets}
			\2	\Theorem \textbf{Finite Intersection of Open Sets is Open}
		\1	\Theorem \textbf{Reverse Triangle Inequality}
		\1	\Definition \textbf{Interior, Closure, and Boundary}
		\1	\Theorem \textbf{$A \subset X$ is closed iff $A = cl(A)$}
		\1	\Theorem \textbf{$A \subset X$ is open iff $A = int(A)$}
		\1	\Theorem \textbf{$X \setminus int(A) = cl(X \setminus A)$}
		\1	\Theorem \textbf{$\partial A = cl(A) \setminus int(A)$}
		\1	\Definition \textbf{$E \subset X$ is dense if $cl(E) = X$} 
		\1	\Definition \textbf{$(X,d)$ is seperable if it admits a countable dense subset}
		\section{Sequences in Metric Spaces}
		\1	\Definition \textbf{Convergence in a metric space}
		\1	\Definition \textbf{Limit point}
		\1	\Definition \textbf{Subsequence}
		\1	\Theorem \textbf{Closure, Interior, and Boundary in Metric Spaces}
		\1	\Theorem	\textbf{Every convergent sequence is a Cauchy sequence}
		\1	\Definition \textbf{A metric space where every Cauchy sequence converges is complete}
		\1	\Definition \textbf{Diameter of $E \subset (X,d)$ is $diam(E) = \text{sup}\{d(x,y) ~:~x,y \in E\}$}
		\1	\Theorem	\textbf{Cantor's Theorem}
		\1	\Theorem	\textbf{$\mathbb{R}^n~n \in \mathbb{N}$ is a complete metric space}
		\1	\Definition \textbf{Bounded means $diam(A) < \infty$}
		\section{Functions Between Metric Spaces}
		\1	\Definition \textbf{Continuous function between metric spaces}	
		\1	\Theorem	\textbf{Properties of continuous functions between metric spaces}
		\1	\Theorem	\textbf{Urysohn's Lemma}
		\1	\Definition \textbf{Homemorphism}
		\1	\Theorem	\textbf{Properties of Homeomorphisms}
		\1	\Definition \textbf{Equivalent Metrics}
		\1	\Theorem \textbf{Equivalent metrics do not necessarily have the same Cauchy Sequences}
		\section{Compact Subsets in Metric Spaces}
		\1	\Definition \textbf{Cover}
		\1	\Theorem	\textbf{Extreme Value Theorem}
		\1	\Definition \textbf{Totally Bounded}
		\1	\Definition \textbf{Finite Intersection Property}
		\1	\Theorem	\textbf{Heine-Borel Theorem}
		\section{Connectedness in Metric Spaces}
		\1	\Definition \textbf{Connected Set}
		\1	\Theorem \textbf{Intermediate Value Theorem}
		\1	\Definition \textbf{Component of a metric space}
		\1	\Definition \textbf{$\varepsilon$-chain}
		\section{Baire Category Theorem}
		\1	\Theorem	\textbf{Baire Category Theorem}

	\end{outline}
\end{document}


