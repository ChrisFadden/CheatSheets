\documentclass[14pt]{extarticle}
\usepackage{researchPaper}
\usepackage{outlines}

\title{Complex Analysis Cheat Sheet}
\begin{document}
	\maketitle

%	https://wwwf.imperial.ac.uk/~kb514/M2PM3.pdf
	
	\begin{outline}		
		\1	Operations on Complex Numbers
			\2	Definition
				\3	$z = x + iy$
			\2	Addition
				\3 $(x + iy) + (w + iv) = (x + w) + i(v + y)$
			\2	Multiplication
				\3	$(x + iy)(u + iv) = (xu - yv) + i(xv + yu)$
			\2	Division
				\3 $(x + iy) / (u + iv) = \frac{xu + yv}{u^2 + v^2} + i\frac{yu - xv}{u^2 + v^2}$
			\2	Real and Imaginary Operators
				\3	$\Re{x + iy} = x$
				\3	$\Im{x + iy} = y$
			\2	Modulus
				\3	$|x + iy| = \sqrt{x^2 + y^2}$
			\2	Conjugate
				\3	$\overline{x + iy} = x - iy$
			\2	Polar representation
				\3	$x + iy = (x^2 + y^2)~exp(i~atan2(y,x)) = r~exp(i \theta)$
			\2	Argument Operator
				\3	$Arg\{r~exp(i \theta)\} = \theta$
		\1	Complex Exponential
			\2	$exp(i \theta) = cos(\theta) + i sin(\theta)$
		\1	Riemann Sphere
			\2	$\mathbb{C}_{\infty} = \{\mathbb{C} \cup \infty\}$
			\2	Extended Complex Plane
				\3	Extends the complex plane to include values at $\infty$
				\3	$\forall~z \in \mathbb{C}~:~z + \infty = \infty$
				\3	$\forall~z \in \mathbb{C}~:~z \cdot \infty = \infty$
		
		\1	Geometry of Complex Numbers
			\2	Open Disc: $D(z_0,R) = \{z~:~|z - z_0| < R\}$
			\2	Open Set: $\{\forall~z \in S~\exists~R : D(z,R) \in S\}$
			\2	Neighborhood:	An open disc around a point
	
		\1	Functions of a complex variable
			\2	limits and continuity
				\3	Limit must be independent of any path taken in the neighborhood around a point
				\3	Continuity implies the limit and the point both exist and coincide
			\2	complex derivative
				\3	$f'(z) = \lim_{h \rightarrow 0} \frac{f(z + h) - f(z)}{h}$
			\2	Holomorphic function
				\3	Differentiable at all $z \in D(z_0,\epsilon)$ 
				\3	Holomorphic = analytic
			\2	Entire function
				\3	A function that is differentiable (holomorphic) on all $\mathbb{C}$
		
		\1	Cauchy-Riemann Equations ($f(z) = u(x,y) + iv(x,y)$)
			\2	$\pd{u}{x} = \pd{v}{y}$
			\2	$\pd{u}{y} = -\pd{v}{x}$
			\2	Sufficient condition for differentiability
			\2	Harmonic Functions
				\3	$\nabla^2u = 0$
				\3	$\nabla^2v = 0$
				\3	$f(z)$ is analytic 	
		\1	Branches
			\2	Complex Logarithm
				\3	Complex logarithm is multiple valued
				\3	Principal Branch: $Log(z) = ln(|z|) + i~Arg(z)$
					\4	$Arg(z)$ uses $atan2(y,x)$
					\4	Different definition of $Arg(z)$ leads to a different branch
				\3	Branch Cuts
					\4	The points where a multi-valued function agree are branch points
					\4	Branch cuts are curves connecting branch points where there is
							a single analytic branch on the plane minus the curve
					\4	Transition a multi-valued function to a specific branch of that
							function
			\2	Complex powers = $z^c = exp(c Log(z))$	
		\1	Complex Integration
			\2	Contour Integral
				\3	Let $t$ parameterize the curve given by $\gamma$
				\3	$\int_{\gamma} f(z) dz = \int_a^b f(z(t))z'(t) dt$
			\2	Deformation Invariance Theorem
				\3	$f(z)$ is analytic 
				\3	$\gamma_0$ and $\gamma_1$ can be continuously deformed into each other
				\3	$\int_{\gamma_0} f(z)dz = \int_{\gamma_1} f(z) dz$
			\2	Cauchy's Integral Theorem: $\oint_{\gamma} f(z) dz = 0$
				\3	$f(z)$ must be holomorphic
			\2	Cauchy's Integral Formula
				\3	$f^{(n)}(a) = \frac{1}{2 \pi i} \oint_{\gamma} \frac{f(z)}{(z - a)^{n+1}}dz$
				\3	$\gamma$ is the curve of the boundary of domain $D$
				\3	$a$ is in the interior of $D$
		\1	Complex Derivatives and Integral Formulas
			\2	Cauchy Estimate: $|f^{(n)}(z_0)| \le \frac{n!}{r^n}max\{|f(z)| \in D(z_0,r)\}$	
			\2	Liouville's Theorem
				\3	Every bounded entire function must be constant
			\2	Mean Value Property of analytic functions: $f(a) = \frac{1}{2 \pi}\int_0^{2\pi} f(a + r~exp(i\theta))d\theta$
			\2	Maximum Modulus Principle
				\3	If $f$ is holomorphic, $|f|$ cannot exhibit a true local maximum
			\2	Poisson Integral Formula:	
				\3$f(r~exp(i\theta)) = \frac{1}{2\pi}\int_0^{2\pi} \frac{1 - r^2}{1 - 2rcos(\theta - t) + r^2} f(exp(i~t)) dt$
				\3 $0<r<1$
			\2	Schwarz Integral Formula
				\3	$f(z) = \frac{1}{2\pi i} \oint_{\gamma} \frac{\gamma + z}{\gamma - z}\Re{f(\gamma)} \frac{d\gamma}{\gamma} + i \Im{f(0)}$
				\3	$|\gamma| = 1$
				\3	$|z| < 1$
			\2	Laurent Series: $f(z) = \sum_{n=-\infty}^{\infty} a_n (z - c)^n$
				\3	$a_n = \frac{1}{2\pi i} \oint_{\gamma} \frac{f(z)}{(z - c)^{n+1}}dz$
				\3	Complex version of Taylor series
				\3	An analytic function can be defined in terms of a power series, such as Taylor or Laurent
		\1	Zeros, Poles and Singularities
			\2	Zeros
				\3	$a$ is a simple zero if: $f(z) = (z - a)^ng(z)$
				\3	$g(z)$,$f(z)$ holomorphic, $g(a) \neq 0$
				\3	$n$ is multiplicity or order of the zero
				\3	For holomorphic functions, zeros are isolated, there is a neighborhood with no other zeros
			\2	Poles
				\3	$a$ is a pole if: $f(z) = \frac{g(z)}{(z - p)^n}$
				\3	$n$ is multiplicity or order of the pole
			\2	Meromorphic Function
				\3	Function that is holomorphic everywhere except a set of isolated poles
			\2	Removable Singularity
				\3	Singularity is a pole of order 0
				\3	There is a holomorphic function that agrees everywhere with the function except at the singularity
				\3	The function with the singularity then takes the value of the holomorphic function at that point
			\2	Essential Singularity
				\3	Singularity that is not a pole nor a removable singularity	
		\1	Residues:	$Res(f,c) = \frac{1}{2 \pi i}\oint_{\gamma} f(z)dz$
			\2	$D = \{z : 0 < |z - c| < R\}$
			\2	$\gamma$ traces a circle counterclockwise around $c$	
			\2	Cauchy Residue Theorem:	$\oint_{\gamma}f(z)dz = 2\pi i \sum_k Res(f,a_k)$
		\1	Analytic Continuation
			\2	A function normally not defined outside of a domain can take values of
					an analytic function that agrees with it at all points in the domain
		\1	Conformal Maps
			\2	A function that preserves angles locally
			\2	Holomorphic
			\2	Derivative everywhere non-zero
			\2	Riemann Mapping Theorem	
				\3 Any simply connected subset of $\mathbb{C}$ is can be conformally mapped to the open unit disc
	\end{outline}
\end{document}


