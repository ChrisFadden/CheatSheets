\documentclass[14pt]{extarticle}
\usepackage{researchPaper}
\usepackage{outlines}

%Gatech ghomi

\title{Differential Geometry of Curves and Surfaces Cheat Sheet}

\begin{document}
	\maketitle
	\begin{outline}
		\1	Euclidean Geometry
			\2	Standard Inner Product: $\braket{p}{q} = p^1q^1 + p^2q^2 + ... $
			\2	Norm: $\norm{p} := \sqrt{(\braket{p}{p})}$
			\2	Cauchy Schwarz Inequality: $\braket{p}{q} \le \norm{p}\norm{q}$
		\1	Curve
			\2	$\alpha : I \rightarrow \mathbb{R}^n$
			\2	Reparameterization
				\3	$\beta : J \rightarrow \mathbb{R}^n$
				\3	Bijection $\theta : I \rightarrow J$
				\3	$\alpha(t) = \beta(\theta(t))$
			\2	Length of a curve
				\3	Partition $P = [a,b]$ where $a = t_0$ and $b = t_n$ 
				\3	$len(\alpha,P) = \sum_{i=1}^n \norm{\alpha(t_i) - \alpha(t_{i-1})}$
				\3	$len(\alpha) = \int_{I}\norm{\alpha'(t)}dt$
			\2	Regular Curve: $\norm{\alpha'(t)} \ne 0$
			\2	Closed curve:	Start and end points are equal

		\1	Isometry in Euclidean Space
			\2	$f : \mathbb{R}^n \rightarrow \mathbb{R}^n$
			\2	$f(p) = f(0) + A(p)$
				\3	$0$ is the origin of the space
				\3	$A$ is an orthogonal transformation (rotation)
				\3	This can shift the origin and rotate 
			\2	Preserves inner products
			\2	Preserves the length of curves
		\1	Curvature of Curves
			\2	Unit tangent vector $T(t) := \frac{\alpha'(t)}{\norm{\alpha'(t)}}$
			\2	$\kappa := \norm{\od{T}{s}}$ where $s$ is the length of the curve
			\2	If curve is parameterized by arc length, $\kappa = \norm{\alpha''(t)}$
			\2	$\kappa := \frac{\sqrt{\norm{\alpha'(t)}^2\norm{\alpha''(t)}^2 - 
											\braket{\alpha'(t)}{\alpha''(t)}^2  }}{\norm{\alpha'(t)}^3 }$
			\2	Fox-Milnor Theorem
				\3	Total $\kappa[\alpha] = \int_a^b \norm{\alpha''(t)}dt$
				\3	Only valid for unit speed curves ($\norm{\alpha'(t)} = 1$)
		\1	Convex Curves
			\2	Total Curvature $\int_I \kappa(t) dt \ge 2\pi$
			\2	For closed planar curves
			\2	Equality holds iff the curve is convex
			\2	vertex
				\3	Place where the curvature has a maximum or minimum
				\3	Convex curves have at least 4

		\1	Frenet-Serret Frame and Torsion
			\2	unit speed curve
			\2	Normal: $N(t) := \frac{T'(t)}{\kappa(t)}$
			\2	BiNormal: $B(t) := T(t) \wedge N(t)$	
			\2	Torsion: $\tau(t) := -\braket{B'(t)}{N'(t)}$
			\2	$\begin{bmatrix}
						T(t) \\ N(t) \\ B(t)
					\end{bmatrix}' = 
					\begin{bmatrix}
						0 & \kappa(t) & 0 \\
						-\kappa(t) & 0 & \tau(t) \\
						0 & -\tau(t) & 0 \\
					\end{bmatrix}
					\begin{bmatrix}
						T(t) \\ N(t) \\ B(t)
					\end{bmatrix}$
		
		\1	Surface
			\2	$graph(f) := \{(x,y,f(x,y) | (x,y) \in U \subset \mathbb{R}^2\}$
				\3	A regular (smooth) embedded surface
			\2	Surface of Revolution
				\3	$\alpha : I \rightarrow \mathbb{R}^2$
				\3	$\alpha(t) = (x(t),y(t))$
				\3	$X : I \cross \mathbb{R} \rightarrow \mathbb{R}^3$
				\3	$X(t,\theta) := (x(t)cos(\theta),x(t)sin(\theta),y(t))$
				\3	$X(t,\theta)$ is a regular embedded surface
	
	\1	Gaussian Curvature
		\2	Regular Embedded Surface $\mathcal{M} \subset \mathbb{R}^3$	
		\2	Local Gauss Map
			\3	$(\mathcal{V},n)$	
			\3	$V$ is a neighborhood of point $p \in \mathcal{M}$
			\3	$n : \mathcal{V} \rightarrow \mathbb{S}^2$
			\3	$n(p)$ is orthogonal to the tangent bundle $T_pM$
			\3	$N(u_1,u_2) := \frac{D_1X(u_1,u_2) \wedge D_2X(u_1,u_2)}{\norm{D_1X(u_1,u_2) \cross D_2X(u_1,u_2)}}$
				\4	$D_1$ and $D_2$ are directional derivatives
			\3	$V := X(U)$
			\3	$n(p) := N \circ X^{-1}(p)$
			\3	Maps the gradient to the sphere
		\2	Differential (form)
			\3	$\mathcal{M}_1$ and $\mathcal{M}_2$ are regular embedded surfaces in $\mathbb{R}^3$
			\3	$f : \mathcal{M}_1 \rightarrow \mathcal{M_2}$
			\3	Differential $df_p : T_p\mathcal{M}_1 \rightarrow T_{f(p)}\mathcal{M}_2$
		\2	Shape Operator
			\3	$S_p := -dn_p$
		\2	Gaussian Curvature 
			\3	$K(p) := det(S_p)$
	
	\1	Coefficients of first fundamental form (Metric tensor)
		\2	$g_{ij}(u_1,u_2) := \braket{D_iX(u_1,u_2)}{D_jX(u_1,u_2)}$
		\2	$l_{ij}(u_1,u_2) := \braket{D_{ij}X(u_1,u_2)}{N(u_1,u_2)}$
			\3	$N$ is the normal vector
			\3	Measures of second derivative in the normal direction
		\2	$K(p) = \frac{det(l_{ij}(p,p)}{det(g_{ij}(p,p)}$
	
	\1	Geometric Meaning of Gaussian Curvature
		\2	Local Convexity
			\3	$\mathcal{M}$ is locally convex if $K(p) > 0$
			\3	$\mathcal{M}$ is not locally convex if $K(p) < 0$
			\3	No conclusion on convexity if $K(p) = 0$
		\2	Ratio of Areas
			\3	$V_r = B_r(p) \cap \mathcal{M}$
			\3	$|K(p)| = \lim_{r\rightarrow 0} \frac{Area(n(V_r))}{Area(V_r)}$
			\3	$n(X)$ is the Gauss Map
	
	\1	Principle Curvatures
		\2	$k_v(p) := \braket{\gamma''(0)}{n(p)}$
			\3	$\norm{v} = 1$
			\3	$v \in T_p\mathcal{M}$
			\3	$\gamma(0) = p$
			\3	$\gamma'(0) = v$
		\2	$k_1(p) := min_v k_v(p)$
		\2	$k_2(p) := max_v k_v(p)$
		\2	$K(p) = k_1(p)k_2(p)$
			\3	$k_1(p)$ and $k_2(p)$ are eigenvalues of shape operator $S_p$
		\2	Second Fundamental Form (2-form)
			\3	$II_p : T_p\mathcal{M} \cross T_p\mathcal{M} \rightarrow \mathbb{R}$
			\3	$II_p(v,w) := \braket{S_p(v)}{w}$
			\3	$k_v(p) = II_p(v,v)$
			\3	$k_v(p) = k_1(p)cos^2(\theta) + k_2(p)sin^2(\theta)$
	
	\1	Theorema Egregium
		\2	$f : \mathcal{M} \rightarrow \mathcal{N}$ is an isometry
		\2	$K_n(f(p)) = K_m(p)$
		\2	Gaussian Curvature remains the same under isometry
		\2	Gaussian Curvature intrinsic to surface, doesn't depend on how its embedded 
		\2	$K = \frac{R^1_{121}g_{12} + R^2_{121}g_{22}}{det(g_{ij})}$
			\3	$R^i_{jkl}$ is the Riemann Curvature tensor

	\1	Geodisic Curvature
		\2	$\alpha : I \rightarrow \mathcal{M}$ a unit speed curve 
		\2	$|\kappa_g| := \norm{\alpha'' - \braket{\alpha''}{n(\alpha)}n(\alpha)}$
			\3	$n(\alpha)$ is the local Gauss Map
	
	\1	Gauss-Bonnet Theorem
		\2	$\int_{\mathcal{M}}K dA + \int_{\partial \mathcal{M}}\kappa_g ds = 2\pi \chi(\mathcal{M})$
		\2	$\chi(\mathcal{M})$ is the Euler Characteristic
			\3	For surfaces without boundary, it is $\chi(\mathcal{M}) = 2 - 2g$
				\4	$g$ is the genus, number of topological holes
		\2	Euler characteristic is a topological invariant
		\2	These integrals are invariant to deformations as long as the topology is unchanged
	\end{outline}
\end{document}


