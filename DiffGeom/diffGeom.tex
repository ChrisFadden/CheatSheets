\documentclass[14pt]{extarticle}
\usepackage{researchPaper}
\usepackage{outlines}

%https://ernestyalumni.files.wordpress.com/2015/05/gravity_notes1.pdf

\title{Differential Geometry Cheat Sheet}
\begin{document}
	\maketitle
	\begin{outline}
		\1	Topological Manifolds (d-dimensional)
			\2	Topological Space $(\mathcal{M},\mathcal{O})$ with:
				\3	$\forall~p\in \mathcal{M}:\exists~\mathfrak{U} \in \mathcal{O},~
						 p \in \mathfrak{U} :\exists~\phi:\mathfrak{U} \subseteq \mathcal{M} 
						 \rightarrow x(\mathfrak{U}) \subseteq \mathbb{R}^d$
				\3	x is a continuous bijection
			\2	Chart $(\mathfrak{U},x)$
				\3	Open Subset $\mathfrak{U}$ with homeomorphism $\phi$
				\3	Analogous to a coordinate system
			\2	Atlas is a collection of charts
				\3	$\bigcup_{\alpha \in \mathcal{A}} \mathfrak{U}_{\alpha} = \mathcal{M}$
				\3	$\mathcal{A}$ is an arbitrary (possibly uncountable) set that indexes the chart
			\2	Transition Map
				\3	$\tau_{a,b}~:~\phi_a(\mathfrak{U}_a \cap \mathfrak{U}_b) \rightarrow
							\phi_b(\mathfrak{U}_a \cap \mathfrak{U}_b)$
				\3	$\tau_{a,b} = \phi_b \circ \phi_a^{-1}$
	
		\1	Smooth Manifold $(\mathcal{M},\mathcal{O},\mathcal{A})$
			\2	$(\mathcal{M},\mathcal{O})$ form a topological manifold
			\2	$\mathcal{A}$ is a $C^{\infty}$-atlas (All transition maps are infinitely differentiable)
			\2	Diffeomorphic smooth manifolds
				\3	$\phi~:~\mathcal{M} \rightarrow \mathcal{N}$
				\3	$\phi~:~\mathcal{N} \rightarrow \mathcal{M}$
				\3	$\phi$ and $\phi^{-1}$ are $C^{\infty}$ bijections
			\2	Functions on Manifolds
				\3	$f \circ \phi^{-1} : \phi(\mathfrak{U}) \subset \mathbb{R}^d \rightarrow \mathbb{R}$
				\3	Function depends on the manifold, not the chart
				\3	Different paraterizations / coordinates / charts lead to different functions
				\3	Ex. Temperature is a function, independent of which coordinate system used to express the functional dependence
			\2	Boundary of a Manifold
				\3	Manifold $\mathcal{M}$ without boundary is homeomorphic to $\mathbb{R}^n$
				\3	Boundary of a Manifold $\partial \mathcal{M}$ is all points 
						in $\mathcal{M}$ without a neighborhood homeomorphic to $\mathbb{R}^n$
				\3	$\partial \mathcal{M}$ is always an $n-1$ manifold (without boundary)
					
		\1	Tangent Spaces
			\2	Velocity of a curve $\gamma$ and point $p$ 
				\3	$\gamma~:~\mathbb{R} \rightarrow \mathcal{M}$, $\gamma(\lambda_0) = p$
				\3	$v_{\gamma,p} : C^{\infty}(M) \rightarrow \mathbb{R}$	
				\3	$v_{\gamma,p} := (f \circ \gamma)'(\lambda_0)$
				\3	$v_{\gamma,p} = \gamma_x^i(0) \frac{\partial}{\partial x^i}$
					\4	$\frac{\partial}{\partial x^i}$ is a basis for the velocity vector
					\4	basis is chart dependent
					\4	vector components change under change of chart
					\4	vector itself is chart independent
			\2	Tangent Space at $p$
				\3	$T_pM := \{v_{\gamma,p}~|~\gamma~\text{smooth curves}\}$ 
				\3	Vector space spanned by all the smooth curves defined at $p$
				\3	$dim(T_pM) = d = dim(\mathcal{M})$
				\3	Change of basis: $V^j_y = \frac{\partial y^j}{\partial x^i}V^i_x$
				\3	Collection of all tangent spaces is given by $\Gamma(TM)$
			\2	Cotangent Spaces
				\3	$T^*_pM) := \{\phi : T_pM \rightarrow \mathbb{R}\}$
				\3	$(df)_p \in T_pM$
				\3	$((df)_p)_j = \partial_j(f \circ \phi^{-1})(\phi(p))$
				\3	$\{(dx^1)_p, (dx^2)_p,...,(dx^d)_p\}$ form a basis of $T^*_pM$
				\3	$dx^i$ are known as covectors (distinguished from vectors $\frac{\partial}{\partial x^i}$ of the tangent space)
				\3	Differential forms live on cotangent spaces
				\3	covectors are dual of vectors: $(dx^a)_p((\frac{\partial}{\partial x^b})_p) = \delta_b^a$
				\3	Change of basis:	$\omega_{y,~i} = \frac{\partial x^j}{\partial y^i}w_{x,~j}$
				\3	Covectors and vectors have their components transform differently under different charts
		
		\1	Tensors
			\2	Multilinear Map $T: V^* \cross V^* \cross ... V^* \cross V \cross V ... \cross V \rightarrow \mathbb{R}$
				\3	p copies of $V^*$, covectors
				\3	q copies of $V$, vectors
				\3	can be mapped to any field, doesn't have to be $\mathbb{R}$
			\2	$(p,q)$ Tensor
				\3	$p$ is the number of contravariant indices (covectors)
				\3	$q$ is the number of covariant indices (vectors)
				\3	Order of a Tensor = $p + q$
				\3	1-form is a $(0,1)$ tensor
				\3	2-form is a $(0,2)$ tensor
				\3	inner product is a $(0,2)$ tensor
				\3	vector is a $(1,0)$ tensor
				\3	Linear Maps are $(1,1)$ tensors
			\2	Summation convention
				\3	Repeated indices with one upper and one lower index are summed over
				\3	$\sum_i v_i w^i := v_iw^i$
			\2	Tensors are independent of basis
				\3	A tensor with value $0$ is $0$ in all charts

		\1	Connections
			\2	Terminology
				\3	$\nabla_Xf = Xf = (df)(X)$
				\3	$X~:~C^{\infty}(\mathcal{M}) \rightarrow C^{\infty}(\mathcal{M})$
				\3	$df~:~\Gamma(TM) \rightarrow C^{\infty}(\mathcal{M})$
				\3	$\nabla_X~:~C^{\infty}(\mathcal{M}) \rightarrow C^{\infty}(\mathcal{M})$
			\2	Manifold with connection is the quadruple $(\mathcal{M},\mathcal{O},\mathcal{A},\nabla)$
				\3	$\nabla : \Gamma(TM) \rightarrow \Gamma(TM)$
				\3	$\nabla_X f = Xf$
				\3	$\nabla_X(T + S) = \nabla_XT + \nabla_XS$
				\3	$\nabla_X(T(\omega,Y)) = (\nabla_XT)(\omega,Y) + T(\nabla_X\omega,Y) + T(\omega,\nabla_XY)$
				\3	$\nabla_{fX + Z}T = f\nabla_XT + \nabla_ZT$
			\2	Connection Coefficients (Christoffel Symbols)
				\3	$(dim(M))^3$ many connection coefficients
				\3	$\Gamma^i_{jk}~:~\mathfrak{U} \rightarrow \mathbb{R}$
				\3	$\nabla_{\frac{\partial}{\partial x^k}}dx^i = -\Gamma^i_{jk}dx^j$
				\3	$(\nabla_XY)^i = X(Y^i) + \Gamma^i_{jk}Y^jX^m$
				\3	$(\nabla_X\omega)_i = X(\omega_i) - \Gamma^j_{ik}\omega_jX^m$
				\3	Technically the Christoffel symbols are only the connection coefficients of the Levi-Civita connection

		\1	Parallel Transport along $\gamma$
			\2	Curve $\gamma : \mathbb{R} \rightarrow \mathcal{M}$	
			\2	$\nabla_{v_{\gamma}}X = 0$
			\2	Defined using the tangent space to a curve applied to a vector
		\1	Torsion
			\2	Torsion of connection $\nabla$ is a $(1,2)$ tensor field
			\2	$T(\omega,X,Y) := \omega(\nabla_XY - \nabla_YX - [X,Y])$
				\3	$[X,Y]f := X(Yf) - Y(Xf)$
			\2	$(\mathcal{M},\mathcal{O},\mathcal{A},\nabla)$ is torsion free if $T = 0$
			\2	$T^i_{ab} = \Gamma^i_{ab} - \Gamma^i_{ba}$
		
		\1	Curvature of $\nabla$ is a $(1,3)$ tensor field
			\2	$Riem(\omega,Z,X,Y) := \omega(\nabla_X\nabla_YZ - \nabla_Y\nabla_XZ - \nabla_{[X,Y]}Z)$
			\2	$R^i_{jab} = \frac{\partial}{\partial x^a}\Gamma^i_{ab} - 
					\frac{\partial}{\partial x^b}\Gamma^i_{ja} + 
					\Gamma^i_{ca}\Gamma^c_{jb} - \Gamma^i_{cb}\Gamma^c_{ja}$
		
		\1	Geodisics
			\2	Curve $\gamma : I \rightarrow \mathcal{M}$ is parameterized by t
			\2	Derivative of $\gamma = \gamma'$ with respect to t
			\2	Geodisic := $\gamma | \nabla_{\gamma'} \gamma' = 0$
			\2	Geodisic Equation
				\3	$\frac{d^2 \gamma^a}{dt^2} + \Gamma^a_{bc}\frac{d \gamma^b}{dt}\frac{d \gamma^c}{dt} = 0$
			\2	Geodisics are straight lines with respect to connection $\nabla$
			\2	Motion of geodisics completely determined by bending of the surface (curvature)

		\1	Metric $g$ on smooth manifold $(\mathcal{M},\mathcal{O},\mathcal{A})$ is a (0,2) tensor
			\2	$g(X,Y) = g(Y,X)~\forall~X,Y~\text{vector fields}$
			\2	$g^{-1}$ is a (2,0) tensor
			\2	$(g^{-1})^{am}g_{mb} = \delta^a_b$
			\2	$(0,2)$ tensors technically don't have eigenvalues with the properties of linear algebra
				\3	Instead use signature, the sign of the eigenvalues of the metric tensor matrix
				\3	$(+,+,+,+)$ for Riemannian metric	
				\3	$(+,-,-,-)$ for Lorentzian metric
			\2	Levi-Civita Connection
				\3	The unique $\nabla$ with $\nabla g = 0$ and $T = 0$
				\3	$\Gamma^a_{ij} := g^{am} \frac{1}{2}(\frac{\partial}{\partial x^i}g_{mj} +
							\frac{\partial}{\partial x^j}g_{mi} - \frac{\partial}{\partial x^m}g_{ij})$
				\3	Parallel transport is an isometry (inner products are preserved)
				\3	Levi-Civita is the connection with the Christoffel symbols chosen using the
						metric like above
			\2	Riemann-Christoffel curvature
				\3	$R_{abcd} := g_{am}R^m_{bcd}$
			\2	Ricci Curvature
				\3	$R_{ab} = R^m_{amb}$
			\2	Scalar Curvature
				\3	$R = g^{ab}R_{ab}$

		\1	Pullback and Pushforward
			\2	$\phi : \mathcal{M} \rightarrow \mathcal{N}$
			\2	Pushforward:	$\phi_* : T\mathcal{M} \rightarrow T\mathcal{N}$
				\3	Relates the vectors on the tangent bundles of two manifolds
				\3	Relates the derivatives on one manifold to another
				\3	$\phi_*(X)(f) := X(f \circ \phi)$
				\3	$\phi_*(T_p\mathcal{M}) \subseteq T_{\phi(p)}\mathcal{N}$
				\3	Vectors are pushed forward
			\2	Pullback:	$\phi^* : T^*\mathcal{N} \rightarrow T^*\mathcal{M}$
				\3	$\phi^*(\omega)(X) := \omega(\phi_*(X))$
				\3	Relates differential forms on different manifolds
				\3	Differential forms are pulled back

		\1	Lie Derivative
			\2	$\Gamma(T\mathcal{M}) = \{\text{set of all vector fields}(C^{\infty}-module)\}$
				\3	$X,Y \in \Gamma(T\mathcal{M})$
			\2	Lie Bracket
				\3	$[X,Y]f := X(Yf) - Y(Xf)$
				\3	$[X,Y] = [Y,X]$
				\3	$[\lambda X + Z, Y] = \lambda [X,Y] + [Z,Y]$
				\3	$[X,[Y,Z]] + [Z,[X,Y]] + [Y,[Z,X]] = 0$
			\2	Lie Algebra $(\Gamma(TM),[\cdot,\cdot])$
				\3	A $C^{\infty} Module$ (set of vector fields) with an associated Lie Bracket
			\2	Lie Derivative
				\3	On a smooth manifold $(\mathcal{M},\mathcal{O},\mathcal{A})$
					\4	Don't need a metric
				\3	Sends a pair (vector field $X$, $(p,q)$ tensor $T$) to a $(p,q)$ tensor field
				\3	$\mathcal{L}_Xf = Xf$
				\3	$\mathcal{L}_XY = [X,Y]$	
				\3	$\mathcal{L}_X(T + S) = \mathcal{L}_XT + \mathcal{L}_XS$
				\3	$\mathcal{L}_X(T(\omega,Y)) = (\mathcal{L}_XT)(\omega,Y) + T(\mathcal{L}_X\omega,Y) + T(\omega,\mathcal{L}_XY)$
				\3	$\mathcal{L}_{X + Y} T = \mathcal{L}_XT + \mathcal{L}_YT$
			\2	Symmetry of a metric tensor field $g$
				\3	Associated with a Lie Algebra $(L,[\cdot,\cdot])$
				\3	$X,Y \in L$	
				\3	$(\phi^*g)(X,Y) = g(X,Y)$
				\3	Isometry
				\3	$\mathcal{L}_Xg = 0$
				\3	Connection using this metric is called a Riemannian Connection
				\3	Riemannian geodesics have constant speed

	\end{outline}
\end{document}


