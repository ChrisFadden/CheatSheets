\documentclass[14pt]{extarticle}
\usepackage{researchPaper}
\usepackage{outlines}
\usepackage{mathrsfs}
\def\Definition{{\color{blue} \textbf{Definition:} }}
\def\Theorem{{\color{red} \textbf{Theorem:} }}

\title{Differential Geometry Cheat Sheet}
\begin{document}
	\maketitle
	
	\section*{Topological Manifolds and Bundles}
	\begin{outline}		
		\1	\Definition \textbf{Topological Manifolds}
			\2	A paracompact, Hausdorff, topological space $(M,\mathcal{O})$ is
					called a $d$-dimensional topological manifold if for every point
					$p \in M$, there exists a neighborhood $U(p)$ and a homeomorphism
					$x~:~U(p) \rightarrow \mathbb{R}^d$.  The dimension of the manifold
					is given by $\text{dim}(M) = d$
			\2	A submanifold can be defined if $N \subseteq M$ and $(N,\mathcal{O}|_N)$
					is again a manifold
			\2	The product manifold is defined if $(M,\mathcal{O}_M)$ and 
					$(N,\mathcal{O}_N)$ are topological manifolds of dimension $m$ and
					$n$, then the product manifold is of dimension $m + n$ and
					is represented by $(M \cross N,\mathcal{O}_{M \cross N})$.
		\1	\Definition \textbf{Bundle}
			\2	A bundle is a triple $(E,\pi,M)$ where $E$ and $M$ are topological
					manifolds called the total space (E) and base space (M), while
					$\pi$ is a continuous surjective map $\pi~:~E \rightarrow M$ called
					the projection map.
			\2	Bundles can often be denote by $(E,\pi,M) \equiv E \rightarrow_{\pi} M$
		\1	\Definition \textbf{Fibre}
			\2	For a bundle $E \rightarrow_{\pi} M$, let $p \in M$, then
					the fibre $F$ at point $p$ is $F_p = \pi^{-1}(p)$
		\1	\Definition \textbf{Fibre Bundle}
			\2	$E \rightarrow_{\pi} M$ and let $F$ be a manifold,\\ 
					and $\forall~p \in M~:~\pi^{-1}(p) \cong_{top} F$
		\1	\Definition \textbf{Pull-back Bundle}
			\2	Let $E \rightarrow_{\pi} M$ be a bundle, and let $f~:~N \rightarrow M$
					be a map between manifolds.  The pullback bundle is induced by $f$, so
					that $E_2 \rightarrow_{\tau} N$, where $E_2 = \{(n,e) \in N \cross E~:~f(n) = \pi(e)\}$
		\1	\Definition \textbf{Chart}
			\2	Let $(M,\mathcal{O})$ be a $d$-dimensional manifold, then the pair
					$(U,x)$ where $U \in \mathcal{O}$ and $x~:~U \rightarrow \mathbb{R}^d$
					is a homeomorphism, the pair $(U,x)$ is the chart of the manifold
			\2	$x^i(p)$ are the coordinates of $p \in U$ 
		\1	\Definition \textbf{Atlas}
			\2	A collection of charts $\mathscr{A} = \{(U_a,x_a) ~|~a \in \mathcal{A}\}$
					is called an atlas if $\bigcup U_a = M$
			\2	Two charts $(U,x)$ and $(V,y)$ are $C^k$ compatible if
					the map $y \circ x^{-1}~:~x(U \cap V) \rightarrow y(U \cap V)$ is
					$C^k$
			\2	An atlas is a maximal atlas if for every $(U,x) \in \mathscr{A}$,
					$(V,y) \in \mathscr{A}$ for all $(V,y)$ charts that are $C^k$ 
					compatible with $(U,x)$
			\2	$C^k$ atlases $\mathscr{A}$ and $\mathscr{B}$ are compatible if
					$\mathscr{A} \cup \mathscr{B}$ is again a $C^k$ atlas
	\section*{Differenitable Manifolds}
		\1	\Theorem \textbf{Whitney}
			\2	Any maximal $C^k$ atlas with $k \ge 1$ contains a $C^{\infty}$
					atlas
			\2	Any two maximal $C^k$ atlases which contain the same $C^{\infty}$
					atlas are identitcal
		\1	\Definition \textbf{Smooth Map}
			\2	Let $\phi~:~M \rightarrow N$, where $(M,\mathcal{O},\mathscr{A}_M)$
					and $(N,\mathcal{O}_N,\mathscr{A}_N)$ are $C^k$ manifolds.
					$\phi$ is $C^k$ differentiable at $p \in M$ if for some charts
					$(U,x) \in \mathscr{A}_M$, $p \in U$, $(V,y) \in \mathscr{A}_N$
					$\phi(p) \in V$ and $y \circ \phi \circ x^{-1}$ is $k$-times differentiable
					at $x(p) \in \mathbb{R}^{\text{dim}(M)}$
			\2	The map $\phi$ is smooth if it is infinitely differentiable $C^{\infty}$
		\1	\Definition \textbf{Diffeomorphism}
			\2	$\phi~:~M \rightarrow N$ is a bijective map between smooth manifolds,
					$\phi$ and $\phi^{-1}$ are smooth, then $\phi$ is a diffeomorphism
			\2	Two manifolds $(M,\mathcal{O}_M,\mathscr{A}_M)$ and 
					$(N,\mathcal{O}_N,\mathscr{A}_N)$ are diffeomorphic if there exists
					a diffeomorphism $\phi~:~M \rightarrow N$ between them.
			\2	$M \cong_{diff} N$
		\1	\Theorem \textbf{Moise}
			\2	Let $\text{dim}(M) = 1,2,\text{ or } 3$, then there is a unique
					smooth structure on $M$ up to diffeomorphism.
	
	\section*{Tensors}
		\1	\Definition \textbf{K-Vector Space}
			\2	A triple $(V,*,+)$ with a field $(K)$ with the following operations:
				\3	"Cross Product" $\cross~:~V \cross V \rightarrow V$
				\3	"Scalar Multiplication" $*~:~K \cross V \rightarrow V$
			\2	Vector Spaces are linear spaces
				\3	Elements $V$ are called vectors
				\3	Elements of $K$ are scalars
			\2	\Definition \textbf{Linear Map}
				\3	$f(\lambda v_1 + v_2) = \lambda f(v_1) + f(v_2)$
				\3	A bijective linear map is an isomorphism between vector spaces
		\1	\Definition \textbf{Hom Space}
			\2	$Hom(V,W) \equiv \{f~:~f~:~V \rightarrow_{Linear} W\}$
				\3	Also a vector space
			\2	\Definition \textbf{Endomorphism}
				\3	$End(V) \equiv Hom(V,V)$
		\1	\Definition \textbf{Dual Vector Space}
			\2	$V^* \equiv Hom(V,K)$
				\3	Space of linear maps from $V$ to the underlying field $K$
				\3	Linear functionals, covectors, one-forms are all equivalent names
						for elements of the dual space
		\1	\Definition \textbf{Bilinear Maps}
			\2	$f(\lambda v_1 + v_2,w) = \lambda f(v_1,w) + f(v_2,w)$
			\2	$f(v,\lambda w_1 + w_2) = \lambda f(v,w_1) + f(v,w_2)$
			\2	For a fixed second argument, is linear in the argument that is summed
			\2	NOT A LINEAR MAP (it is a special kind of nonlinear map)

		\1	\Definition \textbf{Tensor}
			\2	A $(p,q)$ tensor $T^p_q$ on $V$ is a multilinear map
			\2	$T~:~V^* \cross ... \cross V^*_p \cross V \cross ... \cross V_q \rightarrow K$
			\2	$T^1_1 = End(V^*)$

		\1	\Definition \textbf{Dual Basis}
			\2	Let a basis $B$ for $V$ be given as $B = \{e_1,...,e_n\}$
			\2	The dual basis is the unique basis $B_2 = \{f^1,...,f^n\}$ such
					that $f^i(e_j) = \delta_j^i$

	\section*{Tangent Vector Spaces}
		\1	Let $M$ be a manifold, define the infinite-dimensional vector space over
				$\mathbb{R}$ in the intuitive way (i.e. $(f+g)(p) \equiv f(p) + g(p)$ etc.)
		\1	\Definition \textbf{Smooth Curve}
			\2	Smooth map $\gamma~:~\mathbb{R} \rightarrow M$
			\2	\Definition \textbf{Directional Derivative} 
				\3	$(f \circ \gamma)'(0)$, where $\gamma(0) = p$
			\2	\Definition \textbf{Tangent Vector}
				\3	$X_{\gamma,p} ~:~f \rightarrow (f \circ \gamma)'(0)$
				\3	The "velocity" of the manifold at a point along curve $\gamma$
		\1	\Definition \textbf{Tangent Space}
			\2	$T_pM \equiv \{X_{\gamma,p}~|~\gamma \text{ is a smooth curve through p}\}$
			\2	$(\frac{\partial}{\partial x^a})_p \equiv X_{\gamma(a),p}$
				\3	Components of the Tangent space are the tangent vectors
		\1	\Definition \textbf{Cotangent Space}
			\2	$T_p^*M \equiv (T_pM)^*$
			\2	$(dx^a)_p \equiv X^*_{\gamma(a),p}$
				\3	Components of cotangent space are differential forms
			\2	$(dx^a) \frac{\partial}{\partial x^b} = \delta_b^a$
		\1	\Definition \textbf{Gradient}
			\2	$d_p f~:~T_pM \rightarrow \mathbb{R}$
		\1	\Definition \textbf{Push-forward}
			\2	$\phi~:~M \rightarrow N$
			\2	The pushforward of $\phi$ is the linear map
					$(\phi_*)_p~:~T_pM \rightarrow_{Linear} T_{\phi(p)}N$
			\2	push-forwards map vectors to vectors
		\1	\Definition \textbf{Pull-back}
			\2	The pullback of a differential form on $N$ is given by:
				\3	$(\phi^*)_p ~:~T^*_{\phi(p)}N \rightarrow T_p^*M$
				\3	$(\phi^*)_p(\omega)~:~T_pM \rightarrow \mathbb{R}$
			\2	Pull-backs map covectors to covectors
		\1	\Definition \textbf{Immersion}
			\2	$\phi$ is an immersion of $M$ into $N$ if $d_p\phi \equiv ~:~T_pM \rightarrow T_{\phi(p)}N$
					is injective $\forall~p \in M$
			\2	$M$ is an immersed submanifold of $N$
			\2	$\phi$ is a submersion if it is surjective
		\1	\Definition \textbf{Embedding}
			\2	$\phi$ must be an immersion
			\2	$M \cong_{top} \phi(M) \subseteq N$
			\2	$M$ is homeomorphic to $\phi(M)$ in the subset topology from $N$	
		\1	\Theorem \textbf{Whitney}
			\2	Any smooth manifold can be embedded in $\mathbb{R}^{\text{2 dim(M)}}$
			\2	Any smooth manifold can be immersed in $\mathbb{R}^{\text{2 dim(M) - 1}}$
		
	\section*{Tensors Over a Ring}
		\1	\Definition \textbf{Vector Fields}
			\2	Set of all vector vields on $M$ is $\Gamma(TM)$
			\2	$\Gamma(TM) \equiv \{ \sigma~:~M \rightarrow TM | \sigma 
					 \text{smooth and } \pi \circ \sigma = id_{M}\}$
				\3	$\pi$ is the projection map $\pi~:~TM \rightarrow M$
			\2	Equivalent definition:
				\3	$\sigma$ is a derivation on the algebra $C^{\infty}(M)$
				\3	$\sigma~:~C^{\infty}(M) \rightarrow C^{\infty}(M)$
				\3	$\sigma(fg) = g\sigma(f) + f\sigma(g)$
			\2	Example
				\3	$\sigma~:~U \rightarrow TU$
				\3	$p~\mapsto~(\frac{\partial}{\partial x^a})_p$
				\3	$\frac{\partial}{\partial x^a}~:~C^{\infty}(U) \rightarrow C^{\infty}(U)$
				\3	$f~\mapsto~\frac{\partial}{\partial x^a}(f) = \partial_a(f \circ x^{-1}) \circ x = \partial_a f$
		\1	\Definition \textbf{Push-forward on Manifold}
			\2	$\phi_*~:~TM \rightarrow TN$
			\2	$X~\mapsto~(\phi_*)_{\pi(X)}(X)$
		\1	\Definition \textbf{Push-forward Vector Field}
			\2	$\Phi_*(\sigma) \in \Gamma(TN)$	
			\2	$\Phi_*(\sigma) \equiv \phi_* \circ \sigma \phi^{-1}$
		\1	\Definition \textbf{Module}
			\2	$\Gamma(TM)$ is a module (vector space over a ring) over $C^{\infty}(M)$
			\2	The triple $(C^{\infty}(M),+,\cdot)$ is a ring
		\1	\Definition \textbf{Properties of Modules (M)}
			\2	Finitely generated if it has a finite generating set
			\2	Free if it has a basis
			\2	Projective if it is a direct summand of a free module $F$ if:
				\3	$M \oplus Q = F$
				\3	Every free module is also projective
			\2	Examples
				\3	$\Gamma(T\mathbb{R}^2)$ is free
				\3	$\Gamma(TS^2)$ is not free
		\1	\Definition \textbf{Basis of a Module}
			\2	Finitely generated module F is free
			\2	$d \in \mathbb{N}$ is the cardinality of the basis
			\2	$F \cong R \oplus R \oplus ..._d \equiv R^d$
		\1	\Theorem \textbf{Serre, Swan}
			\2	$E$ is a vector fibre bundle over smooth manifold $M$
				\3	Every fibre of a vector fibre bundle is a vector space
			\2	$\Gamma(E)$ is a finitely generated, projective $C^{\infty}(M)$ module
		\1	\Theorem \textbf{Hom spaces over R}
			\2	$\text{Hom}_R(P,Q) \equiv \{\phi~:~P \rightarrow Q\}$
			\2	$\text{Hom}_R(P,Q)$ is a finitely generated, projective $R$-module
			\2	Dual of the module
				\3	$\text{Hom}_{C^{\infty}(M)}(\Gamma(TM),C^{\infty}(M)) \equiv \Gamma(TM)^* = \Gamma(T^*M)$
				\3	Covectors (differential forms)
		\1	\Definition \textbf{Pull-back}
			\2	$\phi~:~M \rightarrow N$, $\omega \in \Gamma(T^*N)$
			\2	$\Phi^*(\omega) \in \Gamma(T^*M)$
			\2	$\Phi^*(\omega)~:~M \rightarrow T^*M$
			\2	$p~\mapsto~\Phi^*(\omega)(p)$
	
	\section*{Differential Forms}
		\1	\Definition \textbf{Differential Form}
			\2	$(0,n)$ smooth tensor field $\omega$ which is totally anti-symmetric
			\2	$\omega(X_1,...,X_n) = sgn(\pi) \omega(X_{\pi(1)},...,X_{\pi(2)})$
				\3	$pi \in S_n$
				\3	$X_i \in \Gamma(TM)$
			\2	$\Omega^n(M)$ is the set of all $n$-forms on $M$
				\3	$C^{\infty}$-module
				\3	$\Omega^0(M) = C^{\infty}(M)$
				\3	$\Omega^1(M) = \Gamma(T^*M)$
		\1	\Definition \textbf{Pull-back}
			\2	$\Phi^*(\omega) \in \Omega^n(M)$
			\2	$\Phi^*(\omega)~:~M \rightarrow T^*M$
			\2	$\Phi^*(\omega)(p)(X_1,...,X_n) \equiv \omega(\phi(p))(\phi_*(X_1),...,\phi_*(X_n))$
				\3	$X_i \in T_pM$
			\2	Vectors are pushed forward
			\2	Forms are pulled back
		\1	\Definition \textbf{Wedge Product}
			\2	$\wedge~:~\Omega^n(M) \cross \Omega^m(M) \rightarrow \Omega^{n+m}(M)$
			\2	$\sigma \wedge \omega = \omega \otimes \sigma - \sigma \otimes \omega$
			\2	$(f\omega_1 + \omega_2) \wedge \sigma = f \omega_1 \wedge \sigma + \omega_2 \wedge \sigma$
		\1	\Theorem \textbf{Pull-back and Wedge}
			\2	$\Phi^*(\omega \wedge \sigma) = \Phi^*(\omega) \wedge \Phi^*(\sigma)$
	
	\section*{Grassman Algebra}
		\1	\Definition \textbf{Grassman Algebra}
			\2	$Gr(M) \equiv \Omega(M) = \bigoplus_{n=0}^{\text{dim}M} \Omega^n(M)$
			\2	Grassman Algebra is the algebra $(\Omega(M),+,\cdot,\wedge)$
		\1	\Definition \textbf{Exterior Derivative}
			\2	$d~:~C^{\infty}(M) \rightarrow \Gamma(T^*M)$
			\2	$f~\mapsto~df$
			\2	$df~:~\Gamma(TM) \rightarrow C^{\infty}(M)$
			\2	$X~\mapsto~df(X) = X(f)$
			\2	$df = \partial_a f dx^a$
			\2	$d(fg) = gdf + fdg$
			\2	$d~:~\Omega^0(M) \rightarrow \Omega^1(M)$
		\1	\Definition \textbf{Lie Bracket (Commutator)}
			\2	$X,Y \in \Gamma(TM)$
			\2	$[X,Y]~:~C^{\infty}(M) \rightarrow C^{\infty}(M)$
			\2	$f~\mapsto~[X,Y](f) \equiv X(Y(f)) - Y(X(f))$
			\2	EXAMPLE:
				\3	$d\omega (X,Y) = X(\omega(Y)) - Y(\omega(X)) - \omega([X,Y])$
		\1	\Theorem \textbf{Exterior Derivative and Wedge Product}
			\2	$d(\omega \wedge \sigma) = d\omega \wedge \sigma + (-1)^n \omega \wedge d\sigma$
			\2	$d\omega = d\omega_A \wedge dx^A$
		\1	\Theorem \textbf{Exterior Derivative and Pullback}
			\2	$\Phi^*(d\omega) = d(\Phi^*(\omega))$
		\1	\Definition \textbf{Symplectic form}
			\2	$d\omega = 0$ on $M$, $\omega \in \Omega^2(M)$
			\2	$\forall~Y \in \Gamma(TM)~:~\omega(X,Y) = 0$ implies $X = 0$
			\2	Manifold with symplectic form is a symplectic manifold
	
	\section*{de Rham Cohomology}
		\1	\Definition \textbf{Closed and Exact Forms}
			\2	Closed: $d\omega = 0$
			\2	Exact: if $\exists~\sigma \in \Omega^{n-1}(M)~:~\omega = d\sigma$
			\2	$\omega \in \Omega^n(M)$ is closed iff $\omega in \text{ker}(d~:~\Omega^n(M) \rightarrow \Omega^{n+1}(M)$
			\2	$\omega \in \Omega^n(M)$ is exact iff $\omega \in \text{im}(d~:~\Omega^{n-1}(M) \rightarrow \Omega^n(M)$

		\1	\Theorem \textbf{Composition of Exterior Derivative}
			\2	$d^2 \equiv d \circ d = 0$
			\2	Every Exact form is Closed
		
		\1	\Definition \textbf{Cohomology Subspaces}
			\2	$Z^n \equiv \text{ker}(d~:~\Omega^n(M) \rightarrow \Omega^{n+1}(M))$
				\3	Space of closed $n$-forms
			\2	$B^n \equiv \text{im}(d~:~\Omega^{n-1}(M) \rightarrow \Omega^n(M))$
				\3	Space of exact $n$-forms
		\1	\Theorem \textbf{Poincare Lemma}
			\2	$M \subseteq \mathbb{R}^d$ is a simply connected domain
			\2	$Z^n = B^n$ $\forall~n > 0$
			\2	Closed forms are exact
		\1	\Definition \textbf{De Rham Cohomology Group}
			\2	$H^n(M) \equiv Z^n / B^n$
      \2	$Z^n / \sim$
				\3	$\omega \sim \sigma$ iff $\omega - \sigma \in B^n$
			\2	Exact forms are equivalent to closed forms iff
					$H^n(M) \cong 0$
		\1	\Theorem \textbf{De Rham's Theorem}
			\2	$H^0(M) \cong \mathbb{R}^c$
			\2	$c$ is the number of connected components of $M$
			\2	$H^0(\mathbb{R}) \cong H^0(S^1) \cong \mathbb{R}$
	\section*{Lie Groups and Lie Algebra}
		\1	\Definition \textbf{Lie Group}
			\2	A Group $(G,\cdot)$ where $G$ is a smooth manifold and:
				\3	$\mu~:~G \cross G \rightarrow G$
				\3	$(g_1,g_2) \mapsto g_1 \cdot g_2$
				\3	$i~:~G \rightarrow G$
				\3	$g~\mapsto g^{-1}$
				\3	Are all smooth
			\2	Dimension is the dimension of $G$ as a manifold
			\2	Examples:
				\3	$(\mathbb{R}^n,+)$ the $n$-dimensional translation group
				\3	$(S^1,\cdot)$ is represented by $U(1) \equiv \{z \in \mathbb{C}~|~\abs{z} = 1\}$
				\3	$[GL(n,\mathbb{R}),\circ] = \{\phi~:~\mathbb{R}^n \rightarrow 
							\mathbb{R}^n ~|~\text{det}\phi \ne 0\}$ General Linear Group
				\3	$[O(p,q),\circ] \equiv \{\phi~:~V \rightarrow V~|~
							\forall~v,w \in V~:~\braket{\phi(v)}{\phi(w)} = \braket{v}{w}\}$
		\1	\Definition \textbf{Left Translation map}
			\2	$l_g~:~G \rightarrow G$
			\2	$h \mapsto l_g(h) \equiv g \cdot h \equiv gh$
			\2	Always a diffeomorphism for Lie Groups
			\2	$l_g(hk) \ne l_g(h)l_g(k)$
			\2	$l_g \circ l_h = l_{gh}$
		\1	\Definition \textbf{Push Forward}
			\2	$(L_g)_*~:~\Gamma(TG) \rightarrow \Gamma(TG)$
			\2	$X \mapsto (L_g)_*(X)$
			\2	$(L_g)_*(X)(h) \equiv (l_g)_*(X(g^{-1}h))$
			\2	$(L_g)_*(X)(f) \equiv X(f \circ l_g)$
			\2	\Theorem $(L_g)_* \circ (L_h)_* = (L_{gh})_*$
		\1	\Definition \textbf{Left-Invariant}
			\2	$\forall~g \in G~:~(L_g)_*(X) = X$
			\2	$X(f \circ l_g) = X(f) \circ l_g$
			\2	Set of all Left Invariant vector fields is $\mathcal{L}(G) \subseteq \Gamma(TG)$
			\2	\Theorem $\mathcal{L}(G) \cong_{vec} T_eG$
				\3	$e \in G$ is the identity element
			\2	\Theorem $\text{dim}\mathcal{L}(G) = \text{dim} G$
			\2	\Theorem $\mathcal{L}(G)$ is a Lie subalgebra of $\Gamma(TG)$
				\3	$\forall~X,Y \in \mathcal{L}(G)~:~[X,Y] \in \mathcal{L}(G)$
				\3	$(\Gamma(TM),+,\cdot,[-,-])$ is an infinite-dimensional Lie algebra
			\2	\Definition $\mathcal{L}(G)$ is the associated Lie algebra of $G$
	\section*{Classification of Lie Algebras}
		\1	\Definition \textbf{Abelian Lie algebra}
			\2	$\forall~x,y \in L~:~[x,y] = 0$
		\1	\Definition \textbf{Lie Algebra Ideal}
			\2	$[I,L] \subseteq I$
			\2	$\forall~x \in I~:~\forall~y \in L~:~[x,y] \in I$
		\1	\Definition \textbf{Simple Lie algebra}
			\2	Simple if it is non-abelian and it contains no non-trivial ideals ($0$ and $L$)
			\2	Semi-simple if it contains to non-trivial abelian ideals
			\2	All simple Lie algebra is also semi-simple
		\1	\Definition \textbf{Derived Subalgebra Sequence}
			\2	$L' \equiv [L,L]$
			\2	$L \supseteq L' \supseteq L'' \supseteq ... \supseteq L^{(n)} ...$
		\1	\Definition \textbf{Solvable Lie Algebra}
			\2	$\exists~k \in \mathbb{N}$ s.t. $L^{(k)} = 0$
		\1	\Definition \textbf{Direct Sum of Lie Algebra}
			\2	$[x_1+x_2,y_1+y_2]_{L_1 \bigoplus_{Lie}L_2} \equiv [x_1,y_1]_{L_1} + [x_2,y_2]_{L_2}$
		\1	\Definition \textbf{Semi-Direct sum}
			\2	$[R,L]_{R \bigoplus_s L} \subseteq R$
			\2	$R$ is an ideal of $R \bigoplus_s L$
		\1	\Theorem \textbf{Levi}
			\2	Any finite-dimensional complex Lie Algebra $L$ can be decomposed as
					$L = R \bigoplus_s (L_1 \bigoplus ... \bigoplus L_n)$
			\2	$R$ is a solvable Lie Algebra
			\2	$L1,...,L_n$ are simple Lie Algebras
		\1	\Theorem A Lie Algebra is semi-simple iff it can be expressed as a direct
				sum of simple Lie Algebras
		\1	\Definition \textbf{Adjoint Map}
			\2	$ad_x~:~L \rightarrow L$
			\2	$y \mapsto ad_x(y) \equiv [x,y]$
			\2	\Theorem $ad~:~L \rightarrow \text{End}(L)$ is a Lie Algebra homomorphism
				\3	$ad([x,y]) = [ad(x),ad(y)]$
		\1	\Definition \textbf{Killing Form}
			\2	$K$-bilinear map
			\2	$\kappa~:~L \cross L \rightarrow K$
			\2	$(x,y) \mapsto \kappa(x,y) \equiv tr(ad_x \circ ad_y)$
			\2	$\kappa(x,y) = \kappa(y,x)$
			\2	\Theorem $\kappa([x,y],z) = \kappa(x,[y,z])$
			\2	\Theorem \textbf{Cartan}
				\3	A Lie Algebra is semi-simple iff the Killing form is non-degenerate
				\3	$(\forall~y \in L:\kappa(x,y) = 0) \Rightarrow x = 0$
		\1	\Definition \textbf{Structure Constants}
			\2	$L$ is a Lie Algebra over $K$ and $\{E_i\}$ is a basis
			\2	$[E_i,E_j] = C^k_{ij}E_k$
			\2	$C^k_{ij} = -C^k_{ji}$
			\2	\Theorem adjoint and killing form in basis
				\3	$(ad_{E_i})^k_j = C^k_{ij}$
				\3	$\kappa_{ij} = C^m_{ik}C^k_{jm}$
		
		\1	\Definition \textbf{Cartan subalgebra}
			\2	$H$ of $L$ is a maximal Lie subalgebra of $L$ such that:
			\2	$\exists$ a basis $\{h_1,...,h_r\}$ such that
					$\{h_1,...,h_r,e_1,...,e_{d-r}\}$ is a basis of $L$
			\2	$e_{1},...,e_{d-r}$ are eigenvectors of $ad(h)$ for any $h \in H$
				\3	$\forall~h \in H~:~\exists~\lambda_a(h) \in \mathbb{C}~:~ad(h)e_a = \lambda_a(h)e_a$
			\2	\Definition \textbf{Cartan Weyl Basis} is $\{h_1,...,h_r,e_1,...,e_{d-r}\}$
		
		\1	\Theorem Let $L$ be a finite-dimensional semi-simple complex Lie Algebra
			\2	$L$ possesses a Cartan subalgebra
			\2	All Cartan subalgebras of $L$ have the same dimension (rank of $L$)
			\2	any Cartan subalgebra of $L$ is abelian
		\1	\Definition \textbf{Roots of Lie Algebra}
			\2	$\lambda_1,...,\lambda_{d-r} \in H^*$ are the roots of $L$
			\2	$\Phi \equiv \{\lambda_a\} \subseteq H^*$ is the root set of $L$
			\2	$\Phi$ is not a linearly independent subset of $H^*$
		\1	\Definition \textbf{Fundamental Roots of Lie Algebra}
			\2	$\Pi \subseteq \Phi$
			\2	$\Pi$ is a linearly independent subset of $H^*$
			\2	$\forall~\lambda \in \Phi$, $\exists~n_1,...,n_m \in \mathbb{N}$ such that
					$\lambda = \pm \sum_{i=1}^m n_i \pi_i$
			\2	\Theorem
				\3	$\Pi \subseteq \Phi$ always exists
				\3	$\Pi$ is a basis of $H^*$
				\3	$\abs{\Pi} = \text{dim}H^* = \text{dim}H$ equals the rank of $L$
		\1	\Theorem Restriction of $\kappa$ to $H$ is a pseudo-inner product on $H$
			\2	$(\kappa^*)^{ij}\kappa_{jk} = \delta^i_k$
		\1	\Theorem Restriction of $\kappa^*$ to $H^*_{\mathbb{R}}$ is a true inner product
			\2	$a,b \in H^*_{\mathbb{R}}$
			\2	$\abs{a} \equiv \sqrt{\kappa^*(a,a)}$
			\2	$\phi \equiv \cos^{-1}(\frac{\kappa^*(a,b)}{\norm{a}\norm{b}})$
		\1	\Definition \textbf{Weyl Transformation}
			\2	$s_{\lambda}~:~H^*_{\mathbb{R}} \rightarrow H^*_{\mathbb{R}}$
			\2	$\mu \mapsto s_{\lambda}(\mu)$
			\2	$s_{\lambda}(\mu) \equiv \mu - 2 \frac{\kappa^*(\lambda,\mu)}{\kappa^*(\lambda,\lambda}\lambda$
			\2	$s_{\lambda}$ is a Weyl Transformation
			\2	\Definition \textbf{Weyl Group}
				\3	$W \equiv \{s_{\lambda}~|~\lambda \in \Phi\}$
				\3	Group under composition of maps
				\3	\Theorem The Weyl group is generated by the fundamental roots in $\Pi$
					\4	$\forall~w \in W~:~\exists~\pi_1,...,\pi_n \in \Pi~:~
								w = s_{\pi_1} \circ s_{\pi_2} \circ ... \circ s_{\pi_n}$
				\3	\Theorem Every root can be produced from a fundamental root by the
						action of W
					\4	$\forall~\lambda \in \Phi~:~\exists~\pi \in \Pi~:~\exists~w \in W~:~ \lambda = w(\pi)$
				\3	\Theorem The Weyl Group permutes the roots
					\4	$\forall~\lambda \in \Phi~:~\forall~w \in W~:~w(\lambda) \in \Phi$
		\1	\Definition \textbf{Cartan Matrix}
			\2	$C_{ij} \equiv 2 \frac{\kappa^*(\pi_i,\pi_j)}{\kappa^*(\pi_i,\pi_i)}$
			\2	$C_{ij}$ is $r \cross r$, where $r$ is the rank of the Lie Algebra
			\2	\Theorem Every simple finite dimensional complex Lie Algebra has a 
					unique Cartan matrix and vice versa
		\1	\Definition \textbf{Bond Number}
			\2	$n_{ij} \equiv C_{ij}C_{ji}$
			\2	$n_{ij} = 4 \cos^2(\phi)$
		\1	\Theorem The roots of a simple Lie algebra have, at most, two distinct
				lengths
		\1	\Theorem \textbf{Killing, Cartan}
			\2	Any simple finite-dimensional complex Lie Algebra can be reconstructed
					from its set of fundamental roots $\Pi$, which only come in the following
					forms:
				\3	4 infinite families
				\3	5 exceptional cases	
	\section*{Representation Theory of Lie Groups and Lie Algebras}
		\1	\Definition \textbf{Representation of $L$}
			\2	$\rho~:~L \rightarrow End(V)$
			\2	$V$ is a finite-dimensional vector space over the same field as $L$
				\3	$V$ is the representation space of $\rho$
				\3	The dimension of $\rho$ is $\text{dim} V$
			\2	EXAMPLE: $SL(2,\mathbb{C})$
				\3	$\text{im}_{\rho}(\mathfrak{sl}(2,\mathbb{C})) = 
						\{(a,b;c,d) \in \text{End}(\mathbb{C}^2)~|~a + d = 0\}$
				\3	$\text{im}_{\rho}(\mathfrak{sl}(2,\mathbb{C})) =
						\{\phi \in \text{End}(\mathbb{C}^2) ~|~\text{tr}\phi = 0\}$
		\1	\Definition \textbf{Homomorphism of Representations}
			\2	$\forall~x \in L~:~f \circ \rho_1(x) = \rho_2(x) \circ f$
		\1	\Definition \textbf{Faithful Representation}
			\2	$\text{dim}(\text{im}_{\rho}(L)) = \text{dim}L$
		\1	\Definition \textbf{Direct Sum Representation}
			\2	$\rho \oplus \rho_2~:~L \rightarrow \text{End}(V_1 \oplus V_2)$
				\3	$x \mapsto (\rho_1 \oplus \rho_2)(x) \equiv \rho_1(x) \oplus \rho_2(x)$
		\1	\Definition \textbf{Tensor Product Representation}
			\2	$\rho_1 \otimes \rho_2~:~L \rightarrow \text{End}(V_1 \cross V_2)$
				\3	$x \mapsto \rho_1(x) \otimes \text{id}_{V_2} + \text{id}_{V_1} \otimes \rho_2(x)$
		\1	\Definition \textbf{Reducible Representation}
			\2	$\exists U \subseteq V$ s.t. $\forall~x \in L~:~\forall~u \in U~:~\rho(x)u \in U$
			\2	$\rho|_U~:~L \rightarrow \text{End}(U)$
		\1	\Definition \textbf{$\rho$-Killing Form}
			\2	$\kappa_{\rho}~:~L \cross L \rightarrow \mathbb{C}$
			\2	$(x,y) \mapsto tr(\rho(x) \circ \rho(y))$
		\1 \Theorem If $\rho$ is a faithful representation of a complex semi-simple
				Lie algebra, then $\kappa_{\rho}$ is non-degenerate
		\1	\Definition \textbf{Casimir Operator}
			\2	$\kappa_{\rho}(x,\xi_i) = X^i(x)$
			\2	$\Omega_{\rho} \equiv \sum_{i=1}^{\text{dim} L} \rho(X_i) \circ \rho(\xi_i)$
		\1	\Theorem $[\Omega_{\rho},\rho(x)] = 0$
		\1	\Theorem $\Omega_{\rho} = c_{\rho} \text{id}_V$
			\2	$c_{\rho} = \frac{\text{dim}L}{\text{dim}V}$
	\section*{Reconstruction of a Lie Group from its Algebra}
		\1	\Definition \textbf{Integral Curve}
			\2	$Y \in \Gamma(TM)$
			\2	An integral curve of $Y$ is a $\gamma~:~(-\varepsilon,\varepsilon) \rightarrow M$
			\2	$\forall~\lambda \in (-\varepsilon,\varepsilon)~:~X_{\gamma,\gamma(\lambda)} = Y|_{\gamma(\lambda)}$
			\2	There always exists an integral curve for any $Y$
			\2	Integral Curves are locally unique
		\1	\Definition \textbf{Maximal Integral Curve}
			\2	$I^p_{\text{max}} \equiv \bigcup\{I \subseteq \mathbb{R}~|~$there 
					exists an integral curve $\gamma~:~I \rightarrow M$ of $Y$ through $p\}$
		\1	\Theorem A vector field is complete if $I_{\text{max}}^p = \mathbb{R}~\forall~p \in M$
		\1	\Theorem On a compact manifold, every vector field is complete
		\1	\Theorem Every left-invariant vector field on a Lie Group is complete
			\2	There always exist complete vector fields on a Lie group, even if 
					it is not compact
		\1	\Definition \textbf{Exponential Map}
			\2	$\text{exp}~:~T_eG \rightarrow G$
			\2	$A \mapsto \gamma^A(1)$
			\2	exp is smooth and a local diffeomorphism
			\2	If $G$ is compact, then exp is surjective
		\1	\Theorem The image of exp is the connected component
				of $G$ containing the identity
			\2	If $G$ is connected, then exp is surjective
	
	\section*{Principal fibre bundles}
		\1	A bundle whose typical fibre is a Lie Group
		\1	\Definition \textbf{Lie Group Action on Manifold}
			\2	$\rhd~:~G \cross M \rightarrow M$
			\2	$\forall~p \in M~:~e \rhd p = p$
			\2	$\forall~g_1,g_2 \in G~:~(g_1 \cdot g_2) \rhd p = g_1 \rhd (g_2 \rhd p)$
			\2	$\rhd$ is a left Lie Group action (left G-action) on $M$
			\2	Left G-manifold is a manifold equipped with a left G-action
			\2	Representations of Lie Groups are just a special case of left G-actions
		\1	\Definition \textbf{Orbit}
			\2	$G_p \equiv \{q \in M~:~\exists~g \in G~:~q = g \rhd p\}$
			\2	\Definition \textbf{Orbit Space}
				\2	$M/G \equiv \{G_p~|~p \in M\}$
		\1	\Definition \textbf{Stabiliser}
			\2	$S_p \equiv \{g \in G~|~g \rhd p = p\}$
			\2	$S_p$ is a subgroup of $G$
			\2	$\rhd$ is free if $\forall~p \in M$, $S_p = \{e\}$
			\2	$\rhd$ is transitive if $\forall~p,q \in M$, $\exists~g \in G$ s.t. 
					$p = g \rhd p$
			\2	\Theorem If $\rhd$ is a free action, then $g_1 \rhd p = g_2 \rhd p$ iff
					$g_1 = g_2$
			\2	\Theorem If $\rhd$ is a free action, then $G_p \cong_{diff} G$
		\1	\Definition \textbf{Principal G-bundle}
			\2	$\pi~:~E \rightarrow M$ is isomorphic as a bundle to 
					$\rho~:~E \rightarrow E/G$, where $\rho$ is the quotient map
					sending each $p \in E$ to its equivalence class (orbit)
	
	\section*{Connections}
		\1	$(P,\pi,M)$ is a principal $G$-bundle, and $A \in T_eG$
			\2	$X^A \in \Gamma(TP)$
			\2	$X^A_p~:~C^{\infty}(P) \rightarrow \mathbb{R}$
			\2	$f \mapsto \pd{}{t}[f(p \lhd exp(tA))](0)$
			\2	$i_p~:~T_eG \rightarrow T_pP$
			\2	$A \mapsto X^A_p$
		\1	\Definition \textbf{Vertical Subspace}
			\2	$V_pP = \text{ker}[(\pi_*)_p]$
			\2	$X^A_p \in V_pP$ 
		\1	\Definition \textbf{Horizonal Subspace}
			\2	$T_pP = H_pP \oplus V_pP$
		\1	\Definition \textbf{Connection}
			\2	A choice of horizontal space at each $p \in P$ such that
				\3	$(\lhd g)_*X_p \in H_{p \lhd g}P$
				\3	$X|_p = \text{hor}(X|_p) + \text{ver}(X|_p)$
		\1	\Definition \textbf{Connection One-Form}
			\2	$\omega_p~:~T_pP \rightarrow T_eG$
			\2	$X_p \mapsto i^{-1}_p (\text{ver}(X_p))$
			\2	$H_pP = \text{ker}(\omega_p)$
		\1	\Theorem $\omega_p \circ i_p = \text{id}_{T_eG}$
		\1	\Definition \textbf{Tang-Mills Field}
			\2	$\omega^U~:~\Gamma(TU) \rightarrow T_eG$
			\2	$\omega^U \equiv \sigma^*\omega$
				\3	$\pi \circ \sigma = \text{id}_U$
			\2	\Definition \textbf{Christoffel Symbol}
				\3	$\Gamma^i_{j\mu}(m) \equiv (\omega^U_{\mu}(m))^i_j$
		\1	\Definition \textbf{Gauge Map}
			\2	$\Omega~:~U_1 \cap U_2 \rightarrow G$
			\2	$\sigma_2(m) = \sigma_1(m) \lhd \Omega(m)$
			\2	$\Omega^i_j = \frac{\partial y^i}{\partial x^j}$
	
	\section*{Parallel Transport}
		\1	\Definition \textbf{Horizontal Lift}
			\2	$\gamma~:~[0,1] \rightarrow M$
			\2	$\gamma^{l}~:~[0,1] \rightarrow P$
			\2	$\gamma^{l}(0) = p_0 \in \text{preim}_{\pi}(\{\gamma(0)\})$
				\3	$\pi \circ \gamma^l = \gamma$
				\3	$\forall~\lambda \in [0,1]~:~\text{ver}(X_{\gamma^l,\gamma^l(\lambda)}) = 0$
				\3	$\forall~\lambda \in [0,1]~:~\pi_*(X_{\gamma^l,\gamma^l(\lambda)}) = X_{\gamma,\gamma(\lambda)}$
		\1	\Definition \textbf{Parallel Transport}
			\2	$T_{\gamma}~:~\text{preim}_{\pi}(\{\gamma(0)\}) \rightarrow \text{preim}_{\pi}(\{\gamma(1)\})$
			\2	$p \mapsto \gamma^l_p(1)$
		\1	\Definition \textbf{Holonomy Group}
			\2	$Hol_a(\omega) \equiv \{g_{\lambda} | \gamma^l_p(1) = p \lhd g_{\gamma}$
					for some loop $\gamma\}$
			\2	Holonomy group of $\omega$ at base point $a$
	
	\section*{Covariant Exterior Derivative}
		\1	\Definition \textbf{Exterior Covariant Derivative}
			\2	$\phi$ is a k-form
			\2	$D\phi~:~\Gamma(TP)^{k+1} \rightarrow V$
			\2	$(X_1,...,X_{k+1}) \mapsto d\phi(\text{hor}(X_1),...,\text{hor}(X_{k+1}))$
		\1	\Definition \textbf{Curvature of Connection One-Form}
			\2	$\Omega~:~\Gamma(TP) \cross \Gamma(TP) \rightarrow T_eG$
			\2	$\Omega \equiv D\omega$
		\1	\Definition \textbf{Yang-Mills Field Strength}
			\2	$\text{Riem} \equiv F \equiv \sigma^*\Omega \in \Omega^2(U) \otimes T_eG$
			\2	$\text{Riem}^i_{j\mu \nu} = \partial_{\nu}\Gamma^i_{j\mu} -
					\partial_{\mu}\Gamma^i_{j\nu} + \Gamma^i_{k\mu}\Gamma^k_{j\nu} -
					\Gamma^i_{k \nu}\Gamma^k_{j\mu}$
		\1	\Theorem \textbf{First Bianchi Identity}
			\2	$D \Omega = 0$
		\1	\Definition \textbf{Torsion}
			\2	$\theta \in \Omega^1(P) \otimes V$
			\2	$\Theta \equiv D\omega \in \Omega^2(P) \otimes V$
	
	\end{outline}
\end{document}
























