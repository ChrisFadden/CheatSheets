\documentclass[14pt]{extarticle}
\usepackage{researchPaper}
\usepackage{outlines}
\usepackage{mathrsfs}
\def\Definition{{\color{blue} \textbf{Definition:} }}
\def\Theorem{{\color{red} \textbf{Theorem:} }}

\title{Differential Geometry Cheat Sheet}
\begin{document}
	\maketitle
	
	\section*{Topological Manifolds and Bundles}
	\begin{outline}		
		\1	\Definition \textbf{Topological Manifolds}
			\2	A paracompact, Hausdorff, topological space $(M,\mathcal{O})$ is
					called a $d$-dimensional topological manifold if for every point
					$p \in M$, there exists a neighborhood $U(p)$ and a homeomorphism
					$x~:~U(p) \rightarrow \mathbb{R}^d$.  The dimension of the manifold
					is given by $\text{dim}(M) = d$
			\2	A submanifold can be defined if $N \subseteq M$ and $(N,\mathcal{O}|_N)$
					is again a manifold
			\2	The product manifold is defined if $(M,\mathcal{O}_M)$ and 
					$(N,\mathcal{O}_N)$ are topological manifolds of dimension $m$ and
					$n$, then the product manifold is of dimension $m + n$ and
					is represented by $(M \cross N,\mathcal{O}_{M \cross N})$.
		\1	\Definition \textbf{Bundle}
			\2	A bundle is a triple $(E,\pi,M)$ where $E$ and $M$ are topological
					manifolds called the total space (E) and base space (M), while
					$\pi$ is a continuous surjective map $\pi~:~E \rightarrow M$ called
					the projection map.
			\2	Bundles can often be denote by $(E,\pi,M) \equiv E \rightarrow_{\pi} M$
		\1	\Definition \textbf{Fibre}
			\2	For a bundle $E \rightarrow_{\pi} M$, let $p \in M$, then
					the fibre $F$ at point $p$ is $F_p = \pi^{-1}(p)$
		\1	\Definition \textbf{Fibre Bundle}
			\2	$E \rightarrow_{\pi} M$ and let $F$ be a manifold,\\ 
					and $\forall~p \in M~:~\pi^{-1}(p) \cong_{top} F$
		\1	\Definition \textbf{Pull-back Bundle}
			\2	Let $E \rightarrow_{\pi} M$ be a bundle, and let $f~:~N \rightarrow M$
					be a map between manifolds.  The pullback bundle is induced by $f$, so
					that $E_2 \rightarrow_{\tau} N$, where $E_2 = \{(n,e) \in N \cross E~:~f(n) = \pi(e)\}$
		\1	\Definition \textbf{Chart}
			\2	Let $(M,\mathcal{O})$ be a $d$-dimensional manifold, then the pair
					$(U,x)$ where $U \in \mathcal{O}$ and $x~:~U \rightarrow \mathbb{R}^d$
					is a homeomorphism, the pair $(U,x)$ is the chart of the manifold
			\2	$x^i(p)$ are the coordinates of $p \in U$ 
		\1	\Definition \textbf{Atlas}
			\2	A collection of charts $\mathscr{A} = \{(U_a,x_a) ~|~a \in \mathcal{A}\}$
					is called an atlas if $\bigcup U_a = M$
			\2	Two charts $(U,x)$ and $(V,y)$ are $C^k$ compatible if
					the map $y \circ x^{-1}~:~x(U \cap V) \rightarrow y(U \cap V)$ is
					$C^k$
			\2	An atlas is a maximal atlas if for every $(U,x) \in \mathscr{A}$,
					$(V,y) \in \mathscr{A}$ for all $(V,y)$ charts that are $C^k$ 
					compatible with $(U,x)$
			\2	$C^k$ atlases $\mathscr{A}$ and $\mathscr{B}$ are compatible if
					$\mathscr{A} \cup \mathscr{B}$ is again a $C^k$ atlas
	\section*{Differenitable Manifolds}
		\1	\Theorem \textbf{Whitney}
			\2	Any maximal $C^k$ atlas with $k \ge 1$ contains a $C^{\infty}$
					atlas
			\2	Any two maximal $C^k$ atlases which contain the same $C^{\infty}$
					atlas are identitcal
		\1	\Definition \textbf{Smooth Map}
			\2	Let $\phi~:~M \rightarrow N$, where $(M,\mathcal{O},\mathscr{A}_M)$
					and $(N,\mathcal{O}_N,\mathscr{A}_N)$ are $C^k$ manifolds.
					$\phi$ is $C^k$ differentiable at $p \in M$ if for some charts
					$(U,x) \in \mathscr{A}_M$, $p \in U$, $(V,y) \in \mathscr{A}_N$
					$\phi(p) \in V$ and $y \circ \phi \circ x^{-1}$ is $k$-times differentiable
					at $x(p) \in \mathbb{R}^{\text{dim}(M)}$
			\2	The map $\phi$ is smooth if it is infinitely differentiable $C^{\infty}$
		\1	\Definition \textbf{Diffeomorphism}
			\2	$\phi~:~M \rightarrow N$ is a bijective map between smooth manifolds,
					$\phi$ and $\phi^{-1}$ are smooth, then $\phi$ is a diffeomorphism
			\2	Two manifolds $(M,\mathcal{O}_M,\mathscr{A}_M)$ and 
					$(N,\mathcal{O}_N,\mathscr{A}_N)$ are diffeomorphic if there exists
					a diffeomorphism $\phi~:~M \rightarrow N$ between them.
			\2	$M \cong_{diff} N$
		\1	\Theorem \textbf{Moise}
			\2	Let $\text{dim}(M) = 1,2,\text{ or } 3$, then there is a unique
					smooth structure on $M$ up to diffeomorphism.
	
	\section*{Tensors}
		\1	\Definition \textbf{K-Vector Space}
			\2	A triple $(V,*,+)$ with a field $(K)$ with the following operations:
				\3	"Cross Product" $\cross~:~V \cross V \rightarrow V$
				\3	"Scalar Multiplication" $*~:~K \cross V \rightarrow V$
			\2	Vector Spaces are linear spaces
				\3	Elements $V$ are called vectors
				\3	Elements of $K$ are scalars
			\2	\Definition \textbf{Linear Map}
				\3	$f(\lambda v_1 + v_2) = \lambda f(v_1) + f(v_2)$
				\3	A bijective linear map is an isomorphism between vector spaces
		\1	\Definition \textbf{Hom Space}
			\2	$Hom(V,W) \equiv \{f~:~f~:~V \rightarrow_{Linear} W\}$
				\3	Also a vector space
			\2	\Definition \textbf{Endomorphism}
				\3	$End(V) \equiv Hom(V,V)$
		\1	\Definition \textbf{Dual Vector Space}
			\2	$V^* \equiv Hom(V,K)$
				\3	Space of linear maps from $V$ to the underlying field $K$
				\3	Linear functionals, covectors, one-forms are all equivalent names
						for elements of the dual space
		\1	\Definition \textbf{Bilinear Maps}
			\2	$f(\lambda v_1 + v_2,w) = \lambda f(v_1,w) + f(v_2,w)$
			\2	$f(v,\lambda w_1 + w_2) = \lambda f(v,w_1) + f(v,w_2)$
			\2	For a fixed second argument, is linear in the argument that is summed
			\2	NOT A LINEAR MAP (it is a special kind of nonlinear map)

		\1	\Definition \textbf{Tensor}
			\2	A $(p,q)$ tensor $T^p_q$ on $V$ is a multilinear map
			\2	$T~:~V^* \cross ... \cross V^*_p \cross V \cross ... \cross V_q \rightarrow K$
			\2	$T^1_1 = End(V^*)$

		\1	\Definition \textbf{Dual Basis}
			\2	Let a basis $B$ for $V$ be given as $B = \{e_1,...,e_n\}$
			\2	The dual basis is the unique basis $B_2 = \{f^1,...,f^n\}$ such
					that $f^i(e_j) = \delta_j^i$

	\section*{Tangent Vector Spaces}
		\1	Let $M$ be a manifold, define the infinite-dimensional vector space over
				$\mathbb{R}$ in the intuitive way (i.e. $(f+g)(p) \equiv f(p) + g(p)$ etc.)
		\1	\Definition \textbf{Smooth Curve}
			\2	Smooth map $\gamma~:~\mathbb{R} \rightarrow M$
			\2	\Definition \textbf{Directional Derivative} 
				\3	$(f \circ \gamma)'(0)$, where $\gamma(0) = p$
			\2	\Definition \textbf{Tangent Vector}
				\3	$X_{\gamma,p} ~:~f \rightarrow (f \circ \gamma)'(0)$
				\3	The "velocity" of the manifold at a point along curve $\gamma$
		\1	\Definition \textbf{Tangent Space}
			\2	$T_pM \equiv \{X_{\gamma,p}~|~\gamma \text{ is a smooth curve through p}\}$
			\2	$(\frac{\partial}{\partial x^a})_p \equiv X_{\gamma(a),p}$
				\3	Components of the Tangent space are the tangent vectors
		\1	\Definition \textbf{Cotangent Space}
			\2	$T_p^*M \equiv (T_pM)^*$
			\2	$(dx^a)_p \equiv X^*_{\gamma(a),p}$
				\3	Components of cotangent space are differential forms
			\2	$(dx^a) \frac{\partial}{\partial x^b} = \delta_b^a$
		\1	\Definition \textbf{Gradient}
			\2	$d_p f~:~T_pM \rightarrow \mathbb{R}$
		\1	\Definition \textbf{Push-forward}
			\2	$\phi~:~M \rightarrow N$
			\2	The pushforward of $\phi$ is the linear map
					$(\phi_*)_p~:~T_pM \rightarrow_{Linear} T_{\phi(p)}N$
			\2	push-forwards map vectors to vectors
		\1	\Definition \textbf{Pull-back}
			\2	The pullback of a differential form on $N$ is given by:
				\3	$(\phi^*)_p ~:~T^*_{\phi(p)}N \rightarrow T_p^*M$
				\3	$(\phi^*)_p(\omega)~:~T_pM \rightarrow \mathbb{R}$
			\2	Pull-backs map covectors to covectors
		\1	\Definition \textbf{Immersion}
			\2	$\phi$ is an immersion of $M$ into $N$ if $d_p\phi \equiv ~:~T_pM \rightarrow T_{\phi(p)}N$
					is injective $\forall~p \in M$
			\2	$M$ is an immersed submanifold of $N$
			\2	$\phi$ is a submersion if it is surjective
		\1	\Definition \textbf{Embedding}
			\2	$\phi$ must be an immersion
			\2	$M \cong_{top} \phi(M) \subseteq N$
			\2	$M$ is homeomorphic to $\phi(M)$ in the subset topology from $N$	
		\1	\Theorem \textbf{Whitney}
			\2	Any smooth manifold can be embedded in $\mathbb{R}^{\text{2 dim(M)}}$
			\2	Any smooth manifold can be immersed in $\mathbb{R}^{\text{2 dim(M) - 1}}$
		
		%Chapter 11 (Tensors over a ring)
		\end{outline}
\end{document}
























