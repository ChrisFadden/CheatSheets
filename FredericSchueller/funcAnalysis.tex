\documentclass[14pt]{extarticle}
\usepackage{researchPaper}
\usepackage{outlines}
\usepackage{mathrsfs}
\def\Definition{{\color{blue} \textbf{Definition:} }}
\def\Theorem{{\color{red} \textbf{Theorem:} }}

\title{Functional Analysis Cheat Sheet}
\begin{document}
	\maketitle
	
	\section*{Banach Spaces}
	\begin{outline}		
		\1	\Definition \textbf{Normed Space}
			\2	Complex vector space $(V,+,\cdot)$ equipped with a norm
					($\norm{\cdot}~:~V \rightarrow \mathbb{R}$)
				\3	$f,g \in V$, $z \in \mathbb{C}$	
				\3	$\norm{f} \ge 0$ (non-negativity)
				\3	$\norm{f} = 0$ iff $f = 0$ (definiteness)
				\3	$\norm{z \cdot f} = \abs{z}\norm{f}$ (scalability)
				\3	$\norm{f + g} \le \norm{f} + \norm{g}$ (triangle inequality)
			\2	Metric $d$ on $V$ is $d(f,g) \equiv \norm{f - g}$
			\2	Normed space $(V,\norm{\cdot})$ is complete if the metric space
					$(V,d)$ is complete (Every Cauchy sequence converges)
		\1	\Definition \textbf{Banach Space}
			\2	Complete normed vector space
		\1	\Definition \textbf{Bounded Linear Operator}
			\2	$\sup_{f \in V} \frac{\norm{Af}_W}{\norm{f}_V} < \infty$
		\1	\Theorem A linear operator $A~:~V \rightarrow W$ is bounded iff any of
				the following equivalent conditions are satisfied
			\2	$\sup_{\norm{f}_V = 1}\norm{Af}_W < \infty$
			\2	$\exists~k > 0~:~\forall~f \in V~:~\norm{f}_V \le 1$ implies $\norm{Af}_W \le k$
			\2	$\exists~k > 0~:~\forall~f \in V~:~\norm{Af}_W \le k\norm{f}_V$
			\2	$A$ is continuous with respect to the topologies induced by the
					respective norms on $V$ and $W$
			\2	$A$ is continuous at $0 \in V$
		\1	\Definition \textbf{Operator Norm}
			\2	$\norm{A} \equiv \sup_{\norm{f}_V = 1} \norm{Af}_W = 
					\sup_{f \in V}\frac{\norm{Af}_W}{\norm{f}_V}$
		\1	\Theorem The set $\mathcal{L}(V,W)$ of bounded linear operators from
				a normed space $(V,\norm{\cdot}_V)$ to a Banach space $(W,\norm{\cdot}_W)$
				equipped with pointwise addition and scalar multiplication and the operator
				norm, is a Banach space
		\1	\Definition \textbf{Dual Space}
			\2	Let $V$ be a normed space
			\2	The dual space is $V^* \equiv \mathcal{L}(V,\mathbb{C})$
			\2	Dual of a normed space is a Banach space
			\2	Elements of dual space are called covectors or functionals on $V$
		\1	\Definition \textbf{Weak Convergence}
			\2	$\forall~\phi \in V^*~:~\lim_{n \rightarrow \infty} \phi(f_n) = \phi(f)$
		\1	\Theorem 
			\2	Let $(V,\norm{\cdot})$ be a normed space, and let $D_A$ be a dense
					subspace of $V$
			\2	$\forall~f \in V$, $\exists~\{a_n\}_{n \in \mathbb{N}} \in D_A$ which
					converges to $f$
		\1	\Definition \textbf{Function Extension}
			\2	$A~:~D_A \rightarrow W$, $D_A \subseteq V$
			\2	$\hat{A}~:~V \rightarrow W$ such that
					$\forall~a \in D_A~:~\hat{A}a = Aa$
		\1	\Theorem \textbf{BLT (Bounded Linear Transformation)}
			\2	Let $V$ be a normed space and $W$ be a Banach space
			\2	Any densely defined lienar map $A~:~D_A \rightarrow W$ has a unique
					extension $\hat{A}~:~V \rightarrow W$ such that $\hat{A}$ is bounded
			\2	$\norm{\hat{A}} = \norm{A}$
	
	\section*{Separable Hilbert Spaces}
		\1	\Definition \textbf{Hilbert Space}
			\2	A vector space $(\mathcal{H},+,\cdot)$ equipped with a sesqui-linear
					inner product $\braket{\cdot}{\cdot}$ which incudes a norm
					$\norm{\cdot}_{\mathcal{H}}$ with respect to which $\mathcal{H}$ is
					a Banach space
					\3	$\norm{\cdot}~:~V \rightarrow \mathbb{R}$
					\3	$f \mapsto \sqrt{\braket{f}{f}}$
			\2	Sequi-linear inner products are maps 
					$\braket{\cdot}{\cdot}~:~\mathcal{H} \cross \mathcal{H} \rightarrow \mathcal{H}$
					wich are conjugate symmetric, linear in the second argument and 
					positive definite
		\1	\Theorem \textbf{Cauchy-Schwartz Inequality}
			\2	Let $\braket{\cdot}{\cdot}$ be a sesqui-linear inner product
					on $V$.  $\forall~f,g \in V$:
			\2	$\abs{\braket{f}{g}}^2 \le \braket{f}{f}\braket{g}{g}$
		\1	\Theorem \textbf{Jordan- Von Neumann}
			\2	A norm $\norm{\cdot}$ on $V$ is induced by a sequi-linear inner
					product $\braket{\cdot}{\cdot}$ on $V$ iff the parallelogram identity
					holds for all $f,g \in V$
			\2	$\norm{f + g}^2 + \norm{f - g}^2 = 2\norm{f}^2 + 2\norm{g}^2$
			\2	$\braket{f}{g} = \frac{1}{4}(\norm{f+g}^2 - \norm{f-g}^2 +
					j\norm{f - jg}^2 - j\norm{f + ig}^2)$ (polarisation identity)
		\1	\Theorem Let $\mathcal{H}$ be a Hilbert space, then its dual
				$\mathcal{H}^*$ is also a Hilbert Space
			\2	$\norm{f}_{\mathcal{H}^*} \equiv \sup_{\phi \in \mathcal{H}}
					\frac{\abs{f(\phi)}}{\norm{\phi}_{\mathcal{H}}}$
		\1	\Theorem Inner products on a vector space are sequentially continuous
			\2	$\lim_{n \rightarrow \infty} \braket{\phi}{\psi_n} = \braket{\phi}{\psi}$
		\1	\Definition \textbf{Hamel Basis}
			\2	Subset $\mathcal{B} \subseteq V$ such that:
				\3	Any finite subset $\{e_i\} \subseteq \mathcal{B}$ is linearly independent
					\4	$\sum_{i=1}^n \lambda^ie_i = 0$ implies $\lambda_i = 0~\forall~i$
				\3	$\mathcal{B}$ is a generating (or spanning) set for $V$
					\4	$\forall~v \in V,~v = \sum_{i=1}^n \lambda^i e_i$
			\2	\Definition \textbf{Finite Dimensional}
				\3	If a vector space $V$ admits a finite Hamel basis, it is finite-dimensional,
						and its dimension is $\text{dim}V \equiv \abs{\mathcal{B}}$
				\3	Otherwise, it is infinite-dimensional, and $\text{dim}V = \infty$
				\3	A basis on a Banach space is either finite or uncountably infinite
		\1	\Theorem Every vector space admits a Hamel basis
		\1	\Definition \textbf{Schauder basis}
			\2	Let $(W,\norm{\cdot})$ be a Banach space
			\2	A Schauder basis of $W$ is a sequence $\{e_n\}_{n \in \mathbb{N}}$ in
					$W$ such that, $\forall~f \in W$, $\exists~\{\lambda^n\}_{n \in \mathbb{N}} \in \mathbb{C}$
					which is unique such that
			\2	$\lim_{n \rightarrow \infty}\norm{f - \sum_{i=0}^n \lambda^ie_i} = 0$
			\2	Span of Schauder basis is dense in $W$
			\2	A Schauder basis is ordered
			\2	\Definition \textbf{Normalized Schauder basis}
				\3	$\forall~n \in \mathbb{N}~:~\norm{e_n} = 1$
		\1	\Definition \textbf{Separable Hilbert Space}
			\2	An infinite-dimensional Hilbert space is separable iff it
					admits an orthonormal Shauder basis
			\2	$\forall~i,j \in \mathbb{N}~:~\braket{e_i}{e_j} = \delta_{ij}$
		\1	\Theorem \textbf{Pythagoras}
			\2	Let $\mathcal{H}$ be a Hilbert space, and let $\{\psi_0,...,\psi_n\} \subset \mathcal{H}$
					be a finite orthogonal set.  Then:
			\2	$\norm{\sum_{i=0}^n \psi_i}^2 = \sum_{i=0}^n \norm{\psi_i}^2$
		\1	\Theorem $\norm{\psi}^2 = \sum_{i=0}^{\infty} \abs{\braket{e_i}{\psi}}^2$
		\1	\Definition \textbf{Unitary Map (Operator)}
			\2	Isomorphism of Hilbert Spaces
			\2	Let $\mathcal{H}$ and $\mathcal{G}$ be Hilbert spaces
			\2	A bounded bijection $U \in \mathcal{L}(\mathcal{H},\mathcal{G})$
					is called a unitary map or unitary operator if
				\3	$\forall~\psi,\phi \in \mathcal{H}~:~\braket{U\psi}{U\phi}_{\mathcal{G}} =
						\braket{\psi}{\phi}_{\mathcal{H}}$
		\1	\Theorem Let $U~:~\mathcal{H} \rightarrow \mathcal{G}$ be a surjective
				map which preserves the inner product.  Then $U$ is a unitary map.
			\2	$\norm{U} = 1$
			\2	$\forall~\psi \in \mathbb{H}~:~\norm{U \psi}_{\mathcal{G}} = \norm{psi}_{\mathcal{H}}$
		\1	\Theorem Every infinite dimensional separable Hilbert space is unitarily
				equivalent to $\ell^2(\mathbb{N})$
				\2	$\ell^2(\mathbb{N})$ are the square summable complex sequences
				\2	$\ell^2(\mathbb{N}) \equiv \{a~:~\mathbb{N} \rightarrow \mathbb{C}~|~
						\sum_{i=0}^{\infty} \abs{a_i}^2 < \infty\}$
	
	\section*{Projectors}
		\1	\Definition \textbf{Projection and Orthogonal Complement}
			\2	Fix a unit vector $e \in \mathcal{H}$, and have $\psi \in \mathcal{H}$
			\2	Projection $\psi_{\parallel} \equiv \braket{e}{\psi}e$
			\2	Orthogonal Complement $\psi_{\bot} \equiv \psi - \psi_{\parallel}$
		\1	\Theorem A closed subset $\mathcal{M}$ of a complete normed space $\mathcal{H}$ is
				complete
		\1	\Theorem A closed linear subspace of a Hilbert space is a sub-Hilbert space
				with the same inner product.  If the Hilbert space is separable, so is 
				the sub-Hilbert space
		\1	\Definition \textbf{Orthogonal Complement of Hilbert Space}
			\2	$\mathcal{M}^{\bot} \equiv \{\psi \in \mathcal{H}~|~\forall~\phi \in \mathcal{M}~:~
						\braket{\phi}{\psi} = 0\}$
		\1	\Theorem $\mathcal{M}^{\bot}$ is a closed linear subspace of $\mathcal{H}$
			\2	$\mathcal{H} = \mathcal{M} \oplus \mathcal{M}^{\bot}$
		\1	\Definition \textbf{Orthogonal Projector}
			\2	$P_{\mathcal{M}}~:~\mathcal{H} \rightarrow \mathcal{M}$
			\2	$\psi \mapsto \psi_{\parallel}$
			\2	$P_{\mathcal{M}} \circ P_{\mathcal{M}} = P_{\mathcal{M}}$
			\2	$\braket{P_{\mathcal{M}}\psi}{\phi} = \braket{\psi}{P_{\mathcal{M}}\phi}$
			\2	$P_{\mathcal{M}^{\bot}} \psi = \psi_{\bot}$
			\2	$P_{\mathcal{M}} \in \mathcal{L}(\mathcal{H},\mathcal{M})$
		\1	\Theorem Every projector is the orthogonal projector to some closed
				linear subspace
		\1	\Theorem \textbf{Riesz Representation}
			\2	Every $f \in \mathcal{H}^*$ is of the form $f_{\phi}$ for a unique
					$\phi \in \mathcal{H}$
				\3	$f_{\phi}~:~\mathcal{H} \rightarrow \mathbb{C}$
				\3	$\psi \mapsto \braket{\phi}{\psi}$
	
	\section*{Measure Theory}
		\1	\Definition \textbf{$\sigma$-algebra}
			\2	$\Sigma \subseteq P(M)$
			\2	$\Sigma$ is a $\sigma$-algebra for $M$ if:
				\3	$M \in \Sigma$
				\3	If $A \in \Sigma$ then $M \setminus A \in \Sigma$
				\3	For any sequence $A_n \in \Sigma$, $\bigcup_{n=0}^{\infty}A_n \in \Sigma$
		\1	\Definition \textbf{Measurable Space}
			\2	Pair $(M,\Sigma)$ 
			\2	The elements of $\Sigma$ are measurable subsets of $M$
		\1	\Definition \textbf{Measure}
			\2	$\mu~:~\Sigma \rightarrow [0,\infty]$
			\2	$\mu(\emptyset) = 0$
			\2	$\mu(\bigcup_{n=0}^{\infty}A_n) = \sum_{n=0}^{\infty}\mu(A_n)$
		\1	\Definition \textbf{Measure Space}
			\2	Triple $(M,\Sigma,\mu)$ where $(M,\Sigma)$ is a measurable space, and
					$\mu$ is a measure
			\2	$A,B \in \Sigma$ and $A \subseteq B$ then $\mu(A) \le \mu(B)$
			\2	$A,B \in \Sigma$ and $A \subseteq B$ and $\mu(A) < \infty$ then
					$\mu(B \setminus A) = \mu(B) - \mu(A)$
			\2	$\mu(\bigcup_{i=0}^{n}A_i) \le \sum_{i=0}^{n}\mu(A_i)$
			\2	\Definition \textbf{Finite Measure}
				\3	$\exists~\{A_n\}_{n \in \mathbb{N}}$ such that
						$\bigcup_{n=0}^{\infty}A_n = M$ and $\mu(A_n) < \infty$
		\1	\Definition \textbf{Generated $\sigma$-algebra}
			\2	$\sigma(\varepsilon)$ is the smallest $\sigma$-algebra on $M$ containing
					all the sets in $\varepsilon$
			\2	$\sigma(\varepsilon) = \bigcap_{i \in I} \Sigma_i$
		\1	\Definition \textbf{Borel $\sigma$-algebra}
			\2	Let $(M,\mathcal{O})$ be a topological space
			\2	The Borel $\sigma$-algebra on $(M,\mathcal{O})$ is $\sigma(\mathcal{O})$
		\1	\Definition \textbf{Set of measure zero}
			\2	$A \in \Sigma$ such that $\mu(A) = 0$
		\1	\Definition \textbf{Almost Everywhere}
			\2	A property holds everywhere $M$ except for a null subset of $M$
		\1	\Definition \textbf{Complete measure}
			\2	Every subset of every null set is measurable
			\2	$\forall~A \in \Sigma~:~\forall~B \in P(A)~:~\mu(A) = 0 \implies B \in \Sigma$
		\1	\Definition \textbf{Lesbegue measure on $\mathbb{R}^d$}
			\2	There exists a unique, complete, translation-invariant measure 
					$\lambda^d~:~\sigma(\mathcal{O}_{\mathbb{R}^d}) \rightarrow [0,\infty]$
			\2	$\lambda^d([a_1,b_1) \cross ... \cross [a_d,b_d)) = \Pi_{i=1}^d(b_i - a_i)$
		\1	\Theorem The Lesbegue measure on $\mathbb{R}$ is finite
	
	\section*{Integration of Measurable Functions}
		\1	\Definition \textbf{Characteristic function}
			\2	$\chi_A~:~M \rightarrow \mathbb{R}$
			\2	$\chi_A(m) = 1$ if $m \in A$
			\2	$\chi_A(m) = 0$ if $m \notin A$
		\1	\Definition \textbf{Simple Functions}
			\2	$s(M) = \{r_1,...,r_n\}$
			\2	$s$ is simple if it is a linear combination of characteristic functions
		\1	\Theorem $\chi_A$ is measurable iff $A \in \Sigma$
		\1	\Definition \textbf{Integral of a non-negative, measurable, simple function}
			\2	$s~:~M \rightarrow \mathbb{R}$, $s = \sum_{i=1}^n r_i \chi_{A_i}$
			\2	$\int_M s d\mu \equiv \sum_{i=1}^n r_i \mu(A_i)$
		\1	\Theorem Let $s$ and $t$ be non-negative, measurable, simple functions
				and let $c \in [0,\infty)$
				\2	$\int_M(cs + t)d\mu = c \int_M s d\mu + \int_M t d\mu$
		\1	\Definition \textbf{Integral of non-negative measurable functions}
			\2	$f~:~M \rightarrow \mathbb{R}$ is the non-negative measurable function
			\2	$s~:~M \rightarrow \mathbb{R}$ such that $s \le f$
			\2	$\int_M f d\mu \equiv \sup_{s \in S}\int_M s d\mu$
				\3	$\int_A f d\mu \equiv \int_M f \chi_A d\mu$
		\1	\Theorem \textbf{Markov Inequality}
			\2	Let $f~:~M \rightarrow \mathbb{R}$ be a non-negative,measurable function
			\2	$\forall~z \in [0,\infty]$:
			\2	$\int_M f d\mu \ge z \mu(f^{-1}([z,\infty]))$
		\1	\Theorem \textbf{Monotone Convergence}
			\2	$f_n$ is a sequence of non-negative measurable functions, such that
					$f_{n+1} \ge f_n$, and $\lim_{n\rightarrow \infty}f_n = f$
			\2	$\lim_{n\rightarrow \infty} \int_M f_n d\mu = \int_M f d\mu$
			\2	$\int_M(\sum_{n=0}^{\infty}f_n)d\mu = \sum_{n=0}^{\infty}\int_M f_n d\mu$
		\1	\Theorem Let $f$ be a non-negative, measurable function.  Then
				$\int_M f d\mu = 0$ iff $f = 0$ almost everywhere
		\1	\Definition \textbf{Lesbegue Integrable}
			\2	A function $f$ is measurable and:
			\2	$\int_M \abs{f}d\mu < \infty$
			\2	Set of all Lesbegue Integrable functions is represented by $L^1(M)$
		\1	\Theorem \textbf{Dominated Convergence}
			\2	$\abs{f}_n \le g$ almost everywhere
			\2	$\lim_{n \rightarrow \infty}\int_M\abs{f_n - f}d\mu = 0$
			\2	$\lim_{n \rightarrow \infty}\int_Mf_nd\mu = \int_M f d\mu$
		\1	\Theorem \textbf{Holder's Inequality}
			\2	$p,q \in [1,\infty]$, $\frac{1}{p} + \frac{1}{q} = 1$
			\2	$\abs{\int_M fg d\mu} \le (\int_M\abs{f}^pd\mu)^{\frac{1}{p}}
					(\int_M \abs{g}^qd\mu)^{\frac{1}{q}}$
			\2	$\norm{fg}_1 \le \norm{f}_p \norm{g}_p$
		\1	\Theorem $L^p$ are Banach Spaces for $p \in [0,\infty]$
			\2	$L^2$ is a Hilbert space
	
	\section*{Self-adjoint Operators}
		\1	\Definition \textbf{Adjoint Operator}
			\2	$A^*~:~D_{A^*} \rightarrow \mathcal{H}$
			\2	$D_{A^*} \equiv \{\psi \in \mathcal{H}~|~\exists~\eta \in \mathcal{H}~:~
					\forall~a \in D_{A}~:~\braket{\psi}{Aa} = \braket{\eta}{a}$
			\2	$A^*\psi = \eta$
		\1	\Theorem If $A$ is densely defined, then $(A + z)^* = A^* + z^*$
		\1	\Definition \textbf{Kernel and Range}
			\2	$\text{ker}(A) \equiv \{a \in D_A~|~Aa = 0\}$
			\2	$\text{ran}(A) \equiv \{Aa~|~a \in D_A\}$
			\2	image and range are equivalent
			\2	\Definition \textbf{injective}
				\3 iff $\text{ker}(A) = \{0\}$
			\2	\Definition \textbf{surjective}
				\3	iff $\text{ran}(A) = \mathcal{H}$
		\1	\Theorem	An operator $A$ is invertible iff $\text{ker}(A) = \{0\}$ and
				$\bar{\text{ran}(A)} = \mathcal{H}$
		\1	\Theorem If $A$ is densely defined, then $\text{ker}(A^*) = \text{ran}(A)^{\bot}$
		\1	\Theorem If $A,B$ are densely defined, If $A \subseteq B$ then
				$B^* \subseteq A^*$
		\1	\Definition \textbf{Symmetric Operator}
			\2	$\forall~a,b \in D_A~:~\braket{a}{Ab} = \braket{Aa}{b}$
		\1	\Theorem If $A$ is symmetric, then $A \subseteq A^*$
		\1	\Definition \textbf{Self-adjoint Operator}
			\2	$A~:~D_A \rightarrow \mathcal{H}$
			\2	$A = A^*$
			\2	$D_A = D_{A^*}$
			\2	$\forall~a \in D_A~:~Aa = A^*a$
			\2	Any self-adjoint operator is also symmetric (converse not true)
		\1	\Theorem A self-adjoint operator is maximal with respect to self-adjoint
				extension
			\2	$A \subseteq B = B^* \subseteq A^* = A$, $B = A$
		\1	\Definition \textbf{Closable Operator}
			\2	A densely defined operator $A$ is closable if its adjoint $A^*$ is also
					densely defined
			\2	Closure of a closable operator is $\bar{A} \equiv A^{**} = (A^*)^*$
			\2	An operator is closed if $A = \bar{A}$
		\1	\Theorem A symmetric operator is necessarily closable
		\1	\Theorem If $A$ is symmetric, then $A^{**} \subseteq A^*$
		\1	\Definition \textbf{Essentially self-adjoint operator}
			\2	A symmetric operator $A$ is essentially self-adjoint if 
					$\bar{A}$ is self-adjoint
		\1	\Theorem If $A$ is essentially self-adjoint, there exists a unique
				self-adjoint extension of $A$ ($\bar{A}$)
		\1	\Theorem A symmetric operator $A$ is self adjoint if
				$\text{im}(A + z) = \mathcal{H} = \text{im}(A + \bar{z})$
		\1	\Theorem A symmetric operator is essentially self-adjoint iff
				$\text{ker}(A^* + z) = \{0\} = \text{ker}(A^* + \bar{z})$
	
	\section*{Spectrum of Operators}
		\1	\Definition \textbf{Resolvent Map}
			\2	$R_A~:~\rho(A) \rightarrow \mathcal{L}(\mathcal{H})$
			\2	$z \mapsto (A - z)^{-1}$
			\2	$\mathcal{L}(\mathcal{H}) \equiv \mathcal{L}(\mathcal{H},\mathcal{H})$
			\2	$\rho(A) \equiv \{z \in \mathbb{C}~|~(A - z)^{-1} \in \mathcal{L}(\mathcal{H})$
		\1	\Theorem \textbf{Closed Graph Theorem}
			\2	If $A$ is closeed, then $(A - z)^{-1} \in \mathcal{L}(\mathcal{H})$
					iff $(A - z)$ is bijective
		\1	\Definition \textbf{Spectrum}
			\2	$\sigma(A) \equiv \mathbb{C} \setminus \rho(A)$
		\1	\Definition \textbf{Eigenvalues and Eigenvectors}
			\2	$A~:~D_A \rightarrow \mathcal{H}$
			\2	$\exists~\psi \in D_A \setminus \{0\}~:~A\psi = \lambda \psi$
			\2	$\psi$ is the eigenvector
			\2	$\lambda$ is the associated eigenvalue
		\1	\Theorem $\lambda$ an eigenvalue of $A$ then $\lambda \in \sigma(A)$
		\1	\Definition \textbf{Eigenspace}
			\2	$\text{Eig}_A(\lambda) \equiv \{\psi \in D_A~|~A\psi = \lambda \psi\}$
			\2	$\lambda$ is non-degenerate if $\text{dim Eig}_A(\lambda) = 1$ and 
					degenerate if $\text{dim Eig}_A(\lambda) > 1$
			\2	The degeneracy of $\lambda$ is $\text{dim Eig}_A(\lambda)$
		\1	\Theorem Eigenvectors associated to distinct eigenvalues of a self-adjoint
				operator are orthogonal
	\end{outline}
\end{document}
























