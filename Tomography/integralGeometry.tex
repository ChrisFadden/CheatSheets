\documentclass[14pt]{extarticle}
\usepackage{researchPaper}
\usepackage{outlines}
\usepackage{mathrsfs}
\def\Definition{{\color{blue} \textbf{Definition:} }}
\def\Theorem{{\color{red} \textbf{Theorem:} }}
\newcommand*\pFq[2]{{}_{#1}F_{#2}}

\title{Integral Geometry Cheat Sheet}
\begin{document}
	\maketitle	
	\begin{outline}	
		\section*{Chapter I}
		\section*{The Radon Transform and the Support Theorem}
		\1	\Definition \textbf{Radon Transform}
			\2	$\hat{f}(\xi) = \int_{\xi}f(x) dm(x)$
				\3	$\mathcal{R}~:~\mathbb{R}^n \rightarrow \mathbb{P}^n$
					\4	$\mathbb{P}^n$ is the space of all hyperplanes 
		\1	\Definition \textbf{Dual Radon Transform}
			\2	$\check{\phi}(x) = \int_{x \in \xi} \phi(\xi)d\mu(\xi)$
				\3	$\mathcal{R}^*~:~\mathbb{P}^n \rightarrow \mathbb{R}^n$
		\1	\Definition \textbf{Hyperplane Space $\mathbb{P}^n$}
			\2	$\xi \in \mathbb{P}^n, \xi = \{x \in \mathbb{R}^n~:~\braket{x}{\omega} = p\}$
				\3	$\omega \equiv (\omega_1,...,\omega_n)$ is a unit vector
				\3	$p \in \mathbb{R}$
			\2	$(\omega,p) \mapsto \xi$
			\2	$\mathbb{S}^{n-1} \cross \mathbb{R} \rightarrow \mathbb{P}^n$
		\1	\Definition \textbf{Radon Transform on $(\omega,p)$}
			\2	$f_t \equiv x \rightarrow f(t + x)$, $t \in \mathbb{R}^n$ 
			\2	$\hat{f}_t(\omega,p) = \int_{\braket{x}{\omega} = p}f(x + t)dm(x)$
			\2	$\hat{f}_t(\omega,p) = \int_{\braket{y}{\omega} = p + \braket{t}{\omega}}f(y)dm(y)$
			\2	$\hat{f}_t(\omega,p) = \hat{f}(\omega,p + \braket{t}{\omega})$
			\2	$L_p \coloneqq \phi(\omega,p) \rightarrow \pd[2]{}{p} \phi(\omega,p)$
		\1	\Theorem $f \rightarrow \hat{f}$ and $\phi \rightarrow \check{\phi}$ 
				intertwine $L_x$ and $L_p$
			\2	$\mathcal{R}(L_xf) = L_p(\hat{f})$
			\2	$\mathcal{R}^*(L_p \phi) = L_x \check{\phi}$
		\1	\Definition \textbf{Radon and Fourier Transform}
			\2	$\mathcal{F}\{f\} \equiv \tilde{f}(u) = \int_{\mathbb{R}^n} f(x) e^{-j \braket{x}{\omega}} dx$
				\3	$u \in \mathbb{R}^n$
			\2	$\tilde{f}(s\omega) = \int_{-\infty}^{\infty} \hat{f}(\omega,r)e^{-jsr}dr$
				\3	$s \in \mathbb{R}$
				\3	$\omega$ is a unit vector
			\2	$n$-dimensional Fourier Transform is the 1-dimensional Fourier transform
					of the Radon transform.
			\2	$f(x) = \int_{\mathbb{R}^n} f_1(x-y)f_2(y) dy$ then~\\
					$\hat{f}(\omega,p) = \int_{\mathbb{R}}\hat{f}_1(\omega,p-q)\hat{f}_2(\omega,q)dq$
		\1	\Definition \textbf{Schwartz Space}
			\2	$\mathcal{S}(\mathbb{R}^n)$ of complex-valued rapidly decreasing functions
				\3	$f \in C^{\infty}(\mathbb{R}^n)$ belongs to $\mathcal{S}(\mathbb{R}^n)$
						iff $\forall~k,l \in \mathbb{N}$
				\3	$\sup_{x \in \mathbb{R}^n}\abs{(1 + \abs{x})^k(L_x^lf)(x)} < \infty$
			\2	$\mathcal{S}(\mathbb{S}^{n-1} \cross \mathbb{R})$ 
				\3	$\phi \in C^{\infty}(\mathbb{S}^{n-1} \cross \mathbb{R})$, $\forall~k,l \in \mathbb{N}$
				\3	$\sup_{\omega \in \mathbb{S}^{n-1},r \in \mathbb{R}} 
						\abs{(1 + \abs{r}^k) \od[l]{}{r}(D\phi)(\omega,r)} < \infty$
			\2	$\mathcal{S}(\mathbb{P}^n) \equiv \phi \in \mathcal{S}(\mathbb{S}^{n-1} \cross \mathbb{R})
					~:~\phi(\omega,p) = \phi(-\omega,-p)$
		\1	\Theorem For each $f \in \mathcal{S}(\mathbb{R}^n)$ the Radon transform
				$\hat{f}(\omega,p)$ satisfies: $\forall~k \in \mathbb{N}$, 
				$\int_{\mathbb{R}}\hat{f}(\omega,p)p^kdp$ can be written as a $k$-th
				degree homogeneous polynomial in $\omega_1,...,\omega_n$
				\2	$\mathcal{S}^*(\mathbb{P}^n) = \{F \in \mathcal{S}(\mathbb{P}^n)~:~
				\forall~k \in \mathbb{N},~\int_{\mathbb{R}}F(\omega,p)p^k dp$ is a
				homogeneous polynomial in $\omega_1,...,\omega_n$ of degree $k\}$
		\1	\Definition \textbf{Dual of Compact Function Space}
			\2	$\mathcal{D}(\mathbb{P}^n)$ is the $C^{\infty}(\mathbb{P}^n)$ functions with compact support
			\2	$\mathcal{D}^*(\mathbb{P}^n) = \mathcal{S}^*(\mathcal{P}^n) \cap \mathcal{D}(\mathcal{P}^n)$ 
		\1	\Theorem \textbf{Schwartz Theorem}
			\2	The Radon transform $\mathcal{R}~:~f \rightarrow \hat{f}$ is a linear
					one-to-one mapping $\mathcal{S}(\mathbb{R}^n) \rightarrow \mathcal{S}^*(\mathbb{P}^n)$
		\1	\Definition \textbf{Mean Value}
			\2	$S_r(x) \coloneqq \{y~:~\norm{y - x} = r\} \in \mathbb{R}^n$
			\2	$A(r)$ is the area of the sphere $S_r(x)$
			\2	$B_r(x) \coloneqq \{y~:~\norm{y - x} < r\}$
			\2	$(M^rf)(x) \coloneqq \frac{1}{A(r)}\int_{S_r(x)}f(\omega)d\omega$
			\2	\Definition \textbf{Lie Group Interpretation}
				\3	$K$ is the orthogonal group $\bm{O}(n)$
				\3	$dk$ is the Haar measure, $\int dk = 1$
				\3	$y \in \mathbb{R}^n$ and $r = \abs{y}$
				\3	$(M^rf)(x) = \int_K f(x + k \cdot y)dk$
				\3	$\check{\phi}(x) = \int_K \phi(x + k \cdot \xi_0)dk$
					\4	$\xi_0$ is a fixed hyperplane through the origin
		\1	\Definition \textbf{Dual and Forward Radon Transform}
			\2	$\mathcal{R}^*\mathcal{R}(f)(x) = 
				\frac{\Omega_{n-1}}{\Omega_n}\int_{\mathbb{R}^n}\abs{x - y}^{-1}f(y)dy$
				\3	$\Omega_k = 2 \frac{\pi^{k/2}}{\Gamma(k/2)}$ is the area of unit
						sphere in $\mathbb{R}^k$
		\1	\Theorem \textbf{Support Theorem}
			\2	$f \in C^0(\mathbb{R}^n)$
			\2	$\forall~k > 0, \abs{x}^kf(x)$ is bounded
			\2	$\exists~\varepsilon > 0~:~\hat{f}(\xi) = 0~d(0,\xi) > \varepsilon$
				\3	$d$ is a distance function
			\2	If the conditions are satisfied, then $f(x) = 0~\forall~\abs{x} > \varepsilon$
		\1	\Theorem \textbf{Paley-Wiener Theorem}
			\2	The Radon transform is a bijection $\mathcal{R}~:~D(\mathbb{R}^n) \rightarrow D^*(\mathbb{P}^n)$
	
	\section*{The Inversion Formula and Injectivity Questions}
		\1	\Theorem \textbf{Radon Inversion Formula}
			\2	$cf = (-L_x)^{(n-1)/2}(\mathcal{R}^*\mathcal{R}f)$
				\3	$c = (4\pi)^{(n-1)/2}\Gamma(n/2)/\Gamma(1/2)$
				\3	$f \in C^{\infty}(\mathbb{R}^n)$
				\3	$f(x) = 0(\abs{x}^{-N})$ for some $N > n-1$

		\1	\Theorem \textbf{Radial Functions}
			\2	$f(x) \in C^2(\mathbb{R}^n)$, $f(x) = F(r)$, $r = \norm{x}$
			\2	$(L_xf)(x) = \od[2]{F}{r} + \frac{n-1}{r}\od{f}{r}$
			\2	$LM^r = M^rL~\forall~r > 0$
				\3	$M^r$ is the mean value operator

		\1	\Theorem Let $f \in L^1(\mathbb{R}^n)$.  Then $\hat{f}(\omega,p)$ exists
				for almost all $(\omega,p) \in \mathbb{S}^{n-1} \cross \mathbb{R}$
		\1	\Theorem $\mathcal{R}~:~f \rightarrow \hat{f}$ is injective on $L^1(\mathbb{R})$

		\1	\Theorem \textbf{Dual Radon Inversion Formula}
			\2	$c\phi = (-L_p)^{(n-1)/2}\mathcal{R}\mathcal{R}^*(\phi)$
				\3	$c = (4\pi)^{(n-1)/2}\Gamma(n/2)/\Gamma(1/2)$
				\3	$\phi \in \mathcal{S}^*(\mathbb{P}^n)$
		\1	\Theorem \textbf{Radon Inversion Using Hilbert Transform}
			\2	$(\mathcal{H}F)(t) = \frac{j}{\pi}\int_{-\infty}^{\infty}\frac{F(p)}{t - p}dp$
				\3	$F \in \mathcal{S}(\mathbb{R})$
			\2	$(\Lambda \phi)(\omega,p) = \begin{cases}
																				L_{p,(n-1)} \phi(\omega,p) & n \text{ odd } \\
																				\mathcal{H}_p L_{p,(n-1)}\phi(\omega,p) & n \text{ even} \\
																			\end{cases}$
				\3	$L_{p,(n-1)} \equiv \frac{d^{n-1}}{dp^{n-1}}$
				\3	$\phi \in \mathcal{S}(\mathbb{P}^n)$
				\3	$(\Lambda \phi) \in \mathbb{P}^n$
			\2	$cf = \mathcal{R}^*(\Lambda \hat{f})$
				\3	$c = (-4\pi)^{(n-1)/2}\Gamma(n/2)/\Gamma(1/2)$
				\3	$f \in \mathcal{S}(\mathbb{R}^n)$	
	\section*{The Plancherel Formula}
		\1	\Definition \textbf{Measure on $\mathbb{P}^n$}
			\2	$d\omega dp : \phi \rightarrow \int_{\mathbb{S}^{n-1}}\int_{mathbb{R}}\phi(\omega,p)d\omega dp$
			\2	$\phi \in C^0(\mathbb{P}^n)$ with compact support
		\1	\Theorem The mapping $f \rightarrow L_p^{(n-1)/4}\hat{f}$ extends to 
				an isometry of $L^2(\mathbb{R}^n)$ onto the space $L_e^2(\mathbb{S}^{n-1} \cross \mathbb{R})$
				of even functions in $L^2(\mathbb{S}^{n-1} \cross \mathbb{R})$
			\2	The measure on $\mathbb{S}^{n-1} \cross \mathbb{R}$ is
					$\frac{1}{2}(2 \pi)^{1-n} d\omega dp$
		\1	\Theorem \textbf{Plancherel Formula}
			\2	$c \int_{\mathbb{R}^n} f(x)g(x) = \int_{\mathbb{P}^n}(\Lambda \hat{f})(\xi)
					\hat{g}(\xi) d\xi$
	
	\section*{Radon Transform of Distributions}
		\1	\Theorem $\int_{\mathbb{P}^n}\hat{f}(\xi)\phi(\xi) d\xi =
				\int_{\mathbb{R}^n} f(x) \check{\phi}(x) dx$
			\2	$f \in C_c(\mathbb{R}^n)$
			\2	$\phi \in C(\mathbb{P}^n)$
			\2	$d\xi = (\Omega_n)^{-1} d\omega dp$ 
			\2	$\int_{\mathbb{R}^n} f(x) dx = \int_{\mathbb{R}} \hat{f}(\omega,p)dp$
		\1	\Definition For $S \in \mathcal{E}'(\mathbb{R}^n)$ the functional
				$\hat{S}$ is defined by:
			\2	$\hat{S}(\phi) = S(\check{\phi})$
				\3	$\phi \in \mathcal{E}(\mathbb{P}^n)$
		\1 \Definition 	For $\Sigma \in \mathcal{D}'(\mathbb{P}^n)$, the functional
				$\hat{\Sigma}$ is
			\2	$\hat{Sigma}(f) = \Sigma(\hat{f})$
				\3	$f \in D(\mathbb{R}^n)$
			\2	$\forall~S \in \mathcal{E}'(\mathbb{R}^n)$, $\hat{S} \in \mathcal{E}'(\mathbb{P}^n)$
		\1	\Theorem For $S \in \mathcal{E}'(\mathbb{R}^n)$, $\Sigma \in \mathcal{D}'(\mathbb{P}^n)$
			\2	$\mathcal{R}(L_xS) = L_p \hat{S}$
			\2	$\mathcal{R}^*(L_p\Sigma) = L_x \check{\Sigma}$
		\1	\Definition \textbf{Hilbert Transform on Distributions}
			\2	$\mathcal{H}(T)(F) = T(-\mathcal{H}F)$
				\3	$F \in \mathcal{D}(\mathbb{R})$
				\3	$\mathcal{H}(T) \in \mathcal{D}'(\mathbb{R})$

		\1	\Definition \textbf{Radon Inversion on Distributions}
			\2	$\mathcal{R} : S \rightarrow \hat{S}$
				\3	$S \in \mathcal{E}'(\mathbb{R}^n)$
			\2	$cS = \mathcal{R}^*(\Lambda \hat{S})$
				\3	$c = (-4\pi)^{(n-1)/2}\Gamma(n/2)/\Gamma(1/2)$
			\2	If $n$ is odd then $cS = L_x^{(n-1)/2}(\mathcal{R}^* \hat{S})$
		\1	\Theorem Let $f \in L^1(\mathbb{R}^n)$ vanish outside a compact set.
				Then the distribution $T_f$ has Radon transform given by:
			\2	$\hat{T}_f = T_{\hat{f}}$
			\2	$T_h : \phi \rightarrow \int \phi h d\mu$
			\2	$\mathcal{R}^*(T_{\phi}) = T_{\check{\phi}}$
		\1	\Theorem If $n$ is odd, $\forall~f \in L^1(\mathbb{R}^n)$ vanishing outside
				a compact set:
			\2	$c f = L_x^{(n-1)/2}(\mathcal{R}^*(\hat{f}))$
				\3	$c = (-4\pi)^{(n-1)/2}\Gamma(n/2)/\Gamma(1/2)$
	
	\section*{Integration over d-planes; X-ray Transforms; The Range of the d-plane Transform}
		\1	\Definition \textbf{d-dimensional Radon Transform}
			\2	$\mathcal{R} : f \rightarrow \hat{f}$
			\2	$\hat{f} = \int_{\xi}f(x) dm(x)$
				\3	$\xi$ is a $d$-plane
		\1	\Theorem	Let $f,g \in C(\mathbb{R}^n)$, $\norm{x}^mf(x)$ and
				$\norm{x}^mg(x)$ are bounded on $\mathbb{R}^n$.  Assume that 
				$\hat{f}(\xi) = \hat{g}(\xi)$ whenever $\xi$ lies outside the
				unit ball.
			\2	$f(x) = g(x)$ for $\norm{x} > 1$
		\1	\Theorem \textbf{Inverse of d-dimensional Radon Transform}
			\2	$cf = (-L_x)^{d/2}(\mathcal{R}^*\hat{f})$
				\3	$c = (4\pi)^{d/2}\Gamma(n/2)/\Gamma((n-d)/2)$
				\3	$f(x) = 0(\norm{x}^{-N})$ for some $N > d$
	
	\section*{Chapter II}
	\section*{A Duality in Integral Geometry}

	\section*{Example of a Radon Transform}
		\1	http://dsp-book.narod.ru/TAH/ch08.pdf
		\1	$exp(-x_1^2 - x_2^2 - ... - x_n^2) \mapsto (\pi)^{\frac{n-1}{2}	} e^{-p^2}$
			\2	2D: $exp(-x^2 - y^2) \mapsto \sqrt{\pi} exp(-p^2)$
			\2	3D: $exp(-x^2 - y^2 - z^2) \mapsto \pi exp(-p^2)$
		\1	$f(x) = \frac{-1}{8\pi^2} \nabla^2 \int_{\abs{\xi} = 1} \hat{f}(\xi \cdot x, \xi) d\xi$
			\2	$\hat{f}(p,\xi) = \pi exp(-p^2)$
			\2  $f(x,y,z) = exp(-x^2 - y^2 - z^2)$
			\2	$-8\pi^2 exp(-x^2 - y^2 - z^2) = \nabla^2 \Psi(x,y,z)$
			\2	$\od[2]{f}{x} = exp(-x^2) \rightarrow f(x) = c_1 + c_2 x + \frac{\sqrt{\pi}}{2}erf(x) + \frac{e^{-x^2}}{2}$
				\3	$erf(x) = \frac{2}{\sqrt{\pi}}\int_0^x e^{-t^2}dt$
			\2	$\Psi(x,y,z) =? c_1 + c_xx + c_yy + c_zz +
						\pi^kxyzerf(x,y,z) + \pi^{3/2} e^{-x^2 - y^2 - z^2}$
	\end{outline}
\end{document}

