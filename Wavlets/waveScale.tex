\documentclass[14pt]{extarticle}
\usepackage{researchPaper}
\usepackage{outlines}

\title{Scalar Wavelets Cheat Sheet}
\begin{document}
	\maketitle

	\begin{outline}		
		\1	Multiresolution Analysis
			\2	Refinable Function $\phi : \mathbb{R} \rightarrow \mathbb{C}$
				\3	$\phi(x) = \sqrt{2} = \sum_{k} h_k\phi(2x - k)$
				\3	Orthogonal if $\braket{\phi(x)}{\phi(x - k)} = \delta_{0k}, k \in \mathbb{Z}$
					\4	$\sum_k h_k \bar{h}_{k-2l} = \delta_{0l}$
					\4	$|h(\omega)|^2 + |h(\omega + \pi)|^2 = 1$
				\3	$\sum_k |h_k|^{1 + \epsilon} < \infty$
				\3	$h(\omega) = \frac{1}{\sqrt{2}}\sum_k h_k exp(-ik\omega)$
				\3	$\hat{\phi}(\omega) = h(\frac{\omega}{2})\hat{\phi}(\frac{\omega}{2})$ (Fourier Transform)
				\3	$\hat{\phi}(\omega) = (\prod_k h(2^{-k}\omega) ) \hat{\phi}(0)$
				\3	Fixed Point Iteration for finding refinable functions
					\4	$\phi^{(n)}(x) = \sqrt{2}\sum_k h_k \phi^{(n-1)}(2x - k)$
			\2	Haar function
				\3	$\phi(x) = 1, x = [0,1]$
				\3	$\phi(x) = \phi(2x) + \phi(2x-1) = \sqrt{2}(\frac{1}{\sqrt{2}}\phi(2x) + \frac{1}{\sqrt{2}}\phi(2x-1))$
				\3	$h_0 = h_1 = \frac{1}{\sqrt{2}}$
				\3	Orthogonal
			\2	Compact Support
				\3	Support is the closure of set $\{x : \phi(x) \ne 0\}$
				\3	Compact support means bounded support
				\3	Linearly Independent Shifts
					\4	$\sum_k \bar{c}_k \phi(x - k) = 0 \rightarrow c_k = 0~\forall~k$
				\3	$h(0) = 1$
				\3	$\sum_k \phi(x - k) = const \ne 0$
				\3	$h(\pi) = 0$
			
			\2	Multiresolution Approximation (MRA) of $\mathfrak{L}^2$
				\3	Doubly infinite nested sequence of subspaces of $\mathfrak{L}^2$
					\4	$... \subset V_{-1} \subset V_0 \subset V_1 ...$
				\3	$\bigcup_n V_n$ is dense in $\mathfrak{L}^2$
				\3	$\bigcap_n V_n = \{0\}$
				\3	$f(x) \in V_n \rightarrow f(2x) \in V_{n+1}~\forall~n\in \mathbb{Z}$
				\3	$f(x) \in V_n \rightarrow f(x - 2^{-n}k) \in V_n~\forall~n,k \in \mathbb{Z}$
				\3	$\phi \in \mathfrak{L}^2 : \{\phi(x - k) : k \in \mathbb{Z}\}$ forms a stable basis of $V_0$
				\3	Basis function $\phi$ is the scaling function
				\3	$f \in V_0 : f(x) = \sum_k \hat{f_k}\phi(x - k)$
				\3	$\phi(x) = \sqrt{2}\sum_k h_k \phi(2x - k)$
				\3	$\phi_{nk}(x) = 2^{\frac{n}{2}} \phi(2^nx - k)$
				\3	Orthogonal Projection 
					\4	Orthogonal basis functions
					\4	Arbitrary function $f \in \mathfrak{L}^2$ onto $V_n$
					\4	$P_nf = \sum_k \braket{f}{\phi_{nk}}\phi_{nk}$
					\4	$V_n$ has resolution (scale) $2^{-n}$
					\4	$\lim_{n \rightarrow \infty} P_nf = f$
				\3	Detail Subspace
					\4	$Q_nf(x) = P_{n+1}f(x) - P_nf(x)$
					\4	$W_n$ is the image of $Q_n$
					\4	$V_n \oplus W_n = V_{n+1}$
					\4	$P_nP_{n-k} = P_{n-k}P_n = P_{n-k}$
					\4	$P_nQ_n = 0$
					\4	$P_nQ_{n-k} = Q_{n-k}$
					\4	$V_n = \oplus_{k = -\infty}^{n-1} W_k$
					\4	$\oplus_nW_n$ is dense in $\mathfrak{L}^2$
					\4	$W_k \perp W_n (k \ne n)$
					\4	$f(x) \in W_n \rightarrow f(2x) \in W_{n+1}~\forall~n\in \mathbb{Z}$
					\4	$f(x) \in W_n \rightarrow f(x - 2^{-n}k) \in W_n~\forall~n,k \in \mathbb{Z}$
				\3	$\psi \in \mathfrak{L}^2 : \{\psi(x - k\} : k \in \mathbb{Z}$ forms a basis of $W_0$
				\3	$\psi_{nk}$ forms a basis for $\mathfrak{L}^2$
				\3	$\psi(x) = \sqrt{2} \sum_k g_k \phi(2x - k)$
				\3	$g_k = (-1)^kh_{N-k}$ where $N$ is an odd number
				\3	$\psi$ is the (mother) wavelet function
				\3	$\phi$ and $\psi$ together form a wavelet
				\3	$g(\omega) = \frac{1}{\sqrt{2}}\sum_k g_k exp(-ik\omega)$
				\3	$\hat{\psi}(\omega) = g(\frac{\omega}{2})\hat{\phi}(\frac{\omega}{2})$
				\3	Orthogonality of $\phi$ and $\psi$
					\4	$\sum_k h_k \bar{h}_{k-2l} = \sum_kg_k\bar{g}_{k-2l} = \delta_{0l}$
					\4	$\sum_k h_k \bar{g}_{k-2l} = \sum_kg_k\bar{h}_{k-2l} = 0$
					\4	$|h(\omega)|^2 + |h(\omega + \pi)|^2 = |g(\omega)|^2 + |g(\omega + \pi)|^2 = 1$
					\4	$h(\omega)\bar{g}(\omega) + h(\omega + \pi) \hat{g}(\omega + \pi) = \\ 
							 g(\omega)\bar{h}(\omega) + g(\omega + \pi) \hat{h}(\omega + \pi) = 0$
			
			\2	Biorthogonal MRA 
				\3	$\braket{\phi(x)}{\tilde{\phi}(x - k)}	= \delta_{0k}$
				\3	$\tilde{\phi}$ is the dual of $\phi$
				\3	$\sum_k h_k \tilde{h}^*_{k-2l} = \delta_{0l}$
				\3	$h(\omega) \tilde{h}^*(\omega) + h(\omega + \pi) \tilde{h}(\omega + \pi)^* = 1$
				\3	$P_nf = \sum_k \braket{f}{\tilde{\phi}_{nk}}\phi_{nk}$
				\3	$\tilde{P}_nf = \sum_k \braket{f}{\phi_{nk}}\tilde{\phi}_{nk}$
				\3	$Q_nf = P_{n+1}f - P_nf$
				\3	$\tilde{Q}_nf = \tilde{P}_{n+1}f - \tilde{P}_nf$
				\3	If $f \in W_n$, then $\braket{f}{\tilde{\phi}_{nk}} = 0$

			\2	Moments of a function
				\3	Discrete Moments
					\4	$m_k = \frac{1}{\sqrt{2}} \sum_l l^kh_l$
					\4	$n_k = \frac{1}{\sqrt{2}} \sum_l l^kg_l$
				\3	Continuous Moments
					\4	$\mu_k = \int x^k \phi(x) dx$
					\4	$\nu_k = \int x^k \psi(x) dx$
					\4	$\sum_k \phi(x - k) = \int phi(x) dx = \mu_0$

			\2	Approximation Order
				\3	$\norm{f - P_n f} = O(2^{-np})$
					\4	$f$ is $C^p$ differentiable
				\3	$\norm{Q_nf} = O(2^{-np})$
				\3	Polynomial approximation
					\4	$x^n = \sum_k c_{nk} \phi(x - k)$

		\1	Discrete Wavelet Transform (pg. 48)
			\2	Function Approximation
				\3	$s \in V_n$
				\3	$s = \sum_k \bar{s}_{nk} \phi_{nk}$
				\3	$s = \sum_k \bar{s}_{lk} + \sum_{j=l}^{n-1} \sum_k \bar{d}_{jk}\psi_{jk}$
					\4	$\bar{s}_{nk} = \braket{s}{\tilde{\phi}_{nk}}$
					\4	$\bar{d}_{nk} = \braket{s}{\tilde{\psi}_{nk}}$

			\2	DWT
				\3	$\mathbf{s}_n = \{s_{nk}\}$
				\3	Decomposition
					\4	$s_{n-1,j} = \sum_k \tilde{h}_{k - 2j} s_{nk}$
					\4	$d_{n-1,j} = \sum_k \tilde{g}_{k - 2j} s_{nk}$
				\3	Reconstruction
					\4	$s_{nk} = \sum_j [\bar{h}_{k - 2j} s_{n-1,j} + \bar{g}_{k-2j}d_{n-1,j}$
			\2	Modulation Formulation
				\3	Original signal $s_n(\omega)$
				\3	Decomposition
					\4	$s_{n-1}(2\omega) = \frac{1}{\sqrt{2}}(\tilde{h}(\omega)s(\omega) + 
								\tilde{h}(\omega + \pi) s_n(\omega + \pi))$
					\4	$d_{n-1}(2\omega) = \frac{1}{\sqrt{2}}(\tilde{g}(\omega)s(\omega) + 
								\tilde{g}(\omega + \pi) s_n(\omega + \pi))$
				\3	Reconstruction
					\4	$\frac{1}{\sqrt{2}}s_n(\omega + \pi) = (\bar{h}(\omega + \pi)s_{n-1}(2\omega) +
								\bar{g}(\omega + \pi) d_{n-1}(2\omega)$
				\3	Modulation Matrix
					\4	$M(\omega) = \begin{bmatrix}
															h(\omega) & h(\omega + \pi) \\
															g(\omega) & g(\omega + \pi)
														\end{bmatrix}$
					\4 Biorthogonality: $M(\omega)^H \tilde{M}(\omega) = I$ (Paraunitary)
			\2	Polyphase Formulation
				\3	Decomposition
					\4	$\begin{bmatrix} s_{n-1}(\omega) \\ d_{n-1}(\omega) \end{bmatrix}
								= \begin{bmatrix}
										\tilde{h}_0(\omega) & \tilde{h}_1(\omega) \\
										\tilde{g}_0(\omega) & \tilde{g}_1(\omega)
									\end{bmatrix}
									\begin{bmatrix}
										s_{n,0}(\omega) \\
										s_{n,1}(\omega)
									\end{bmatrix}$
				\3	Reconstruction
					\4$		\begin{bmatrix}
										s_{n,0}(\omega) \\
										s_{n,1}(\omega)
									\end{bmatrix} = 
									\begin{bmatrix}
										\bar{h}_0(\omega) & \bar{h}_1(\omega) \\
										\bar{g}_0(\omega) & \bar{g}_1(\omega)
									\end{bmatrix}		
								\begin{bmatrix} s_{n-1}(\omega) \\ d_{n-1}(\omega) \end{bmatrix}	
						$
				\3	Polyphase Matrix
					\4	$P(\omega) = \begin{bmatrix} 
								h_0(\omega) & h_1(\omega) \\
								g_0(\omega) & g_1(\omega)
							\end{bmatrix}$
					\4	Same Birothogonality condition as Modulation Matrix
	
		\1	Types of Wavelets	
			\2	Daubechies Wavelets
				\3	Highest number $k$ of vanishing moments for support $2k - 1$
				\3	$dbX$ where $X$ is the number of vanishing moments
			\2	Coiflets
				\3	Orthogonal wavelets with general number of vanishing moments $k$ for \\
						the scaling functions and $l$ for the wavelet functions
				\3	Useful in applications involving Calderon-Zygmund Operators
			\2	Cohen Wavelet
				\3	Biorthogonal wavelet
				\3	Scaling function is the B-spline of order $p$
		
		\1	Applications
			\2	Signal Processing
				\3	Wavelet Shrinkage (denoising / thresholding)
					\4	Threshold the detail coefficients
					\4	Hard Thresholding $\rightarrow$ set coefficient below threshold to zero
					\4	Soft Thresholding $\rightarrow$ subtract threshold from signal
			\2	Linear Algebra
				\3	Sparsifying Matrix Vector multiplication
				\3	Galerkin Methods
					\4	Expand operator in a scaling function series	
					\4	Differentiation
					\4	Integral Methods

	\end{outline}
\end{document}


