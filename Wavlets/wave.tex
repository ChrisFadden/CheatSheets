\documentclass[14pt]{extarticle}
\usepackage{researchPaper}
\usepackage{outlines}

\title{Multiwavelets Cheat Sheet}
\begin{document}
	\maketitle

	\begin{outline}		
		\1	Multiresolution Analysis of $\mathfrak{L}^2$
			\2	$m \in \mathbb{N}$ is number of filters (2 for standard wavelets)
			\2	$r \in \mathbb{N}$ is the order of the multiwavelet (1 for standard wavelets)
			\2	Scaling Functions	
				\3	Scaling Subspace	
					\4	$... \subset V_{-1} \subset V_0 \subset V_1 \subset V_1...$
					\4	$\bigcup_n V_n$ is dense in $\mathfrak{L}^2$ (can express every function in $\mathfrak{L}^2$)
					\4	$\bigcap_n V_n = \{0\}$
					\4	$f(x) \in V_n \rightarrow f(mx) \in V_{n+1}~~~n \in \mathbb{Z}$
					\4	$f(x) \in V_n \rightarrow f(x - m^{-n}k) \in V_n~~~n,k \in \mathbb{Z}$
				\3	$\bm{\phi} \in \mathfrak{L}^2 : \{\phi_l(x-k) : l = 1,...,r~k \in \mathbb{Z}\}$ forms a basis of $V_0$
				\3	$\bm{\phi}_{nk}(x) = m^{\frac{n}{2}}\bm{\phi}(m^nx - k)$ forms a basis of $V_n$
				\3	Scaling Projection 
					\4	$P_nf(x) = \sum_k \braket{f}{\tilde{\bm{\phi}}_{nk}}\bm{\phi}_{nk}(x)$
					\4	Resolution $m^{-n}$
					\4	Birorthogonal dual represented by tilde	
			\2	Wavelet Functions
				\3	Wavelet Subspace
					\4	$\bigoplus_n W_n$ dense in $\mathfrak{L}^2$
					\4	$W_k \perp \tilde{W}_n$
					\4	$f(x) \in W_n \rightarrow f(mx) \in W_{n+1}~~~n\in \mathbb{Z}$
					\4	$f(x) \in W_n \rightarrow f(x - m^{-n}k) \in W_n~~~n,k \in \mathbb{Z}$
					\4	$V_n \bigoplus W_n = V_{n+1}$
					\4	$V_{n} = \bigoplus_{k=-\infty}^{n-1} W_k$
				\3	$\bm{\psi}^{(s)}(x - k) : s = 1,...,m-1,~~~k \in \mathbb{Z}$ forms a stable basis of $W_0$
				\3	$\bm{\psi}_{nk}(x) = m^{\frac{n}{2}}\bm{\psi}(m^nx - k)$ forms a basis of $W_n$
				\3	Wavelet Projection
					\4	$Q_nf = P_{n+1}f - P_nf$
					\4	$f \in V_n$
					\4	$f = P_nf = P_lf + \sum_{k=l}^{n-1}Q_kf$
		
		\1	Scaling and Wavelet Function Properties
			\2	Scaling Functions
				\3	$\bm{\phi}(x) = \begin{pmatrix} \phi_1(x) \\ ... \\ \phi_r(x)\end{pmatrix}$
				\3	$\phi_n : \mathbb{R} \rightarrow \mathbb{C}$
				\3	$\bm{\phi}(x) = \sqrt{m} \sum_{k = k_0}^{k_1} \bm{H}_k \bm{\phi}(mx - k)~~~k\in \mathbb{Z}$
					\4	$\bm{H}(\omega) = \frac{1}{\sqrt{m}}\sum_{k=k_0}^{k_1}\bm{H}_k exp(-ik\omega)$
					\4	$\bm{\phi}(\omega) = \bm{H}(\frac{\omega}{m})\bm{\phi}(\frac{\omega}{m})$
					\4	$\bm{\phi}(\omega) = (\prod_{k=1}^{\infty} \bm{H}(m^{-k}\omega))\bm{\phi}(\omega = 0)$
					\4	$\bm{\phi}(\omega = 0)$ is an eigenvector of $\bm{H}(\omega = 0)$ at eigenvalue 1.	
				\3	Support
					\4	Closure of $\{x : \phi(x) \ne 0\}$
					\4	$supp~\bm{\phi} = \bigcup_k supp~\phi_k$
					\4	$supp~\bm{\phi} \subset [\frac{k_0}{m-1},\frac{k_1}{m-1}]$ (compact support)
			
				\3	Biorthogonality
					\4	$\braket{\bm{\phi}(x - k)}{\tilde{\phi}(x - l)} = \delta_{kl}\bm{I}$
					\4	$\sum_k \bm{H}_k \tilde{\bm{H}}^*_{k - ml} = \delta_{0l}\bm{I}$
					\4	$\bm{H}(\omega)\tilde{\bm{H}}^*(\omega) + 
							\tilde{\bm{H}}(\omega + \pi){\bm{H}}(\omega + \pi)^* = \bm{I}$
			\2	Wavelet Functions
				\3	$\bm{\psi}^{(s)}(x) = \sqrt{m}\sum_k \bm{G}_k^{(s)}\bm{\phi}(mx - k)$
					\4	$\bm{G}^{(s)}(\omega) = \frac{1}{\sqrt{m}}\sum_k \bm{G}_k^{(s)}exp(-ik\omega)$
					\4	$\bm{\psi}^{(s)}(\omega) = \bm{G}^{(s)}(\frac{\omega}{m})\bm{\psi}(\frac{\omega}{m})$
				\3	Birothogonality
					\4	$\sum \bm{G}_k^{(s)} \tilde{\bm{G}}_{k - kl}^{*(t)} = \delta_{0l}\delta_{st}\bm{I}$
					\4	$\bm{H}_k\bm{G}_{k-ml}^{(s)*} = \sum \bm{G}_k^{(s)}\bm{H}_{k - ml}^{*} = 0$
			\2	Moments
				\3	$k$th Discrete Moment 
					\4	$\bm{M}_k = \frac{1}{\sqrt{m}}\sum_l l^n \bm{H}_l$
					\4	$\bm{N}_k^{(s)} = \frac{1}{\sqrt{m}}l^n \bm{G}_l^{(s)}$
				\3	$k$th Continuous Moments
					\4	$\bm{\mu}_k \int x^k \bm{\phi}(x)dx$
					\4	$\bm{\nu}_k^{(s)} = \int x^k \bm{\psi}^{(s)}(x) dx$
					\4	If biorthogonal: $\tilde{\bm{\mu}}_0^* \bm{\mu}_0 = 1$
				\3	Relations between Continous and Discrete
					\4	$\bm{\mu}_k = m^{-k} \sum_{t = 0}^k \binom{k}{t}\bm{M}_{k-t}\bm{\mu}_t$
					\4	$\bm{\nu}_k = m^{-k} \sum_{t = 0}^k \binom{k}{t}\bm{N}_{k-t}^{(s)}\bm{\mu}_t$
		\1	Modulation and Polyphase Matrices
			\2	Modulation Matrix
			\2	Polyphase Matrix
		
	%USE OTHER RESOURCES TO ADD THINGS

	\end{outline}
\end{document}


