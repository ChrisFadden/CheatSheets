\documentclass[14pt]{extarticle}
\usepackage{researchPaper}
\usepackage{outlines}
\usepackage{mathrsfs}
\def\Definition{{\color{blue} \textbf{Definition:} }}
\def\Theorem{{\color{red} \textbf{Theorem:} }}
\newcommand*\pFq[2]{{}_{#1}F_{#2}}
%Hartshorne
\title{Integral Geometry Cheat Sheet}
\begin{document}
	\maketitle	
	\begin{outline}	
		\section*{Varieties}
		\subsection*{Affine Varieties}
		\1	\Definition \textbf{Affine n-space over $\mathbb{K}$}
			\2	$\mathbb{A}^n$
			\2	The set of all $n$-tuples of elements of $\mathbb{K}$
				\3	$P \in \mathbb{A}^n$
				\3	$P = (a_1,...,a_n)$, $a_i \in \mathbb{K}$
				\3	$a_i$ are the coordinates of point $P$
		\1	\Definition \textbf{Zero-Set}
			\2	$Z(T) = \{P \in \mathbb{A}^n~|~f(P) = 0~\forall~f \in T\}$
			\2	If $a$ is the ideal of $A$ generated by $T$, then $Z(T) = Z(a)$
				\3	$A = \mathbb{K}[X]$ is a polynomial ring 
				\3	$A$ is a Noetherian Ring
					\4	$\exists~n \in \mathbb{N}$ s.t. $I_n = I_{n+1} = ...$ where $I_j$ are ideals
			\2	Any ideal $a$ has a finite set of generators $f_1,...,f_r$
			\2	$Z(T)$ can be expressed as the common zeros of $f_1,...,f_r$
		\1	\Definition \textbf{Algebraic Set}
			\2	$Y \subseteq \mathbb{A}^n$ is an algebraic set if:
			\2	$\exists T \subseteq A$ s.t. $Y = Z(T)$
		\1	\Theorem \textbf{Properties of Algberaic Sets}
			\2	Union of algebraic sets is an algebraic set
			\2	Intersection of algebraic sets is an algbebraic set
			\2	The empty set nad the whole space are algebraic sets

		\1	\Definition \textbf{Zariski Topology on $\mathbb{A}^n$}
			\2	Take the open sets to be the complements of the algebraic sets

		\1	\Definition \textbf{Irreducible sets}
			\2	A nonempty $Y \subseteq X$ is irreducible if it cannot be expressed
					as $Y = Y_1 \cup Y_2$, where $Y_1$ and $Y_2$ are closed proper subsets
			\2	The empty set is not irreducible

		\1	\Definition \textbf{Affine Algebraic Variety}
			\2	Irreducible closed subset of $\mathbb{A}^n$
		
		\1	\Theorem \textbf{Hilbert's Nullstellensatz}
			\2	Let $\mathbb{K}$ be an algebraically closed field, and let $a$ be
					an ideal in $A = \mathbb{K}[X]$, and let $f \in A$ be a polynomial
					which vanishes at all points $Z(a)$
			\2	Then $\exists~r \in \mathbb{N}$ s.t. $f^r \in a$
			\2	An algebraic set is irreducible iff its ideal is a prime ideal
		\1	\Definition \textbf{Affine Curve}
			\2	Let $f$ be an irreducible polynomial $A = \mathbb{K}[x,y]$
			\2	$f$ is a prime ideal in $A$
			\2	$Y = Z(f)$ is an affine curve, defined by $f(x,y) = 0$
			\2	$Y$ is a curve of the same degree as $f$
		\1	\Definition \textbf{Affine Surface}
			\2	$f$ is an irreducible polynomial in $A = \mathbb{K}[x_1,...,x_n]$
			\2	$Y = Z(f)$ is a affine (hyper)-surface
		\1	\Definition \textbf{Affine Coordinate Ring}
			\2	$Y \subseteq \mathbb{A}^n$ is an affine algebraic set
			\2	The affine coordinate ring is $A/I(Y)$
		\1	\Theorem \textbf{Remarks on $A(Y)$}
			\2	If $Y$ is an affine variety, then $A(Y)$ is an integral domain
			\2	$A(Y)$ is a finitely-generated $\mathbb{K}$-algebra
			\2	Any finitely generated $\mathbb{K}$-algebra which is a domain is the 
					affine coordinate ring of some affine variety
		\1	\Definition \textbf{Noetherian Topological Space}
			\2	$Y_1 \supseteq Y_2 \supset ...$ of closed subsets
			\2	$\exists~r \in \mathbb{N}$ s.t. $Y_r = Y_{r+1} = ...$
		\1	\Theorem In a NOetherian topological space, every nonempty closed subset
				can be expressed as a finite union $Y = Y_1 \cup ... \cup Y_r$ of 
				irreducible closed subsets $Y_i$
		\1	\Theorem Every algebraic set in $\mathbb{A}^n$ can be expressed uniquely
				as a union of varieties, no one containing the other
		\1	\Definition \textbf{Dimension}
			\2	The dimension of topological space $X$ is the supremum of all integers
					$n$ such that $\exists$ a chain $Z_0 \subset Z_1 \subset ... \subset Z_n$
					of distinct irreducible closed subsets of $X$
			\2	The dimension of an affine variety is its dimension as a topological space

		\1	\Definition \textbf{Krull Dimension}
			\2	In a ring $A$, the height of a prime ideal $p$ is the supremum of all
					integers $n$ s.t. $\exists$ a chain $p_0 \subset p_1 \subset ... \subset p_n = p$
					of distinct prime ideals
			\2	The Krull dimension of $A$ is the supremum of the heights of all prime
					ideals
		\1	\Theorem If $Y$ is an affine algebraic set, then the dimension of $Y$
				is equal to the dimension of its affine coordinate ring $A(Y)$
		\1	\Theorem \textbf{Krull's Hauptidealsatz}
			\2	Krull's prinicipal ideal theorem
			\2	Let $A$ be a Noetherian ring, and let $f \in A$ be an element which
					is neither a zero divisor nor a unit.  Then every minimal prime ideal
					$p$ containing $f$ has height 1
		\1	\Theorem A Noetherian integral domain $A$ is a unique factorization domain
				iff every prime ideal of height 1 is principal
		\1	\Theorem A variety $Y \in \mathbb{A}^n$ has dimension $n-1$ iff it is
				the zero set $Z(f)$ of a single nonconstant irreducible polynomial in
				$A = \mathbb{K}[X]$
		\subsection*{Projective Varieties}	
	\end{outline}
\end{document}

