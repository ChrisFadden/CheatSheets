\documentclass[14pt]{extarticle}
\usepackage{researchPaper}
\usepackage{outlines}

\title{EE 453 Cheat Sheet}
\begin{document}
	\maketitle
	
	\begin{outline}		
		\1	Discrete Signals
			\2	Obtained by sampling a continuous-time signal
			\2	$x[n] = x_a(nT)$
				\3	$T$ = sampling period
				\3	$\frac{1}{T}$ = sampling frequency
				\3	$n = ...,-2,-1,0,1,2,...$
			\2	Discrete Sequences
				\3	Finite: $N_1 \le n \le N_2$
				\3	Right sided: $N_1 \le n$ (Causal if $N_1 \ge 0$)
				\3	Left-sided:	$n \le N_2$ (anti-causal)
				\3	Conjugate Symmetric: $x_{cs}[n] = x^*_{cs}[-n] = x_r[-n] - jx_i[-n]$
					\4	$x_{cs}[n] = \frac{1}{2}(x[n] + x^*[-n]) = x_{cs}^*[-n]$
					\4	If real:	even ($x_r[-n] = x_r[n]$)
				\3	Conjugate Antisymmetric:	$x_{ca}[n] = -x^*_{ca}[-n] = -x_r[-n] + jx_i[-n]$
					\4	$x_{ca}[n] = \frac{1}{2}(x[n] - x^*[-n]) = -x_{ca}^*[-n]$
					\4	$x[n] = x_{cs}[n] + x_{ca}[n]$
					\4	If real:	odd($x_r[-n] = -x_r[n]$)	
				\3	Periodic:	$x[n] = x[n + kN]$
					\4	Period: $N \in \mathbb{N}$
					\4	$k \in \mathbb{Z}$
					\4	$exp(j\omega_0 n)$ has fundamental period N iff $\omega_0N = 2\pi r, r \in \mathbb{N}$
					\4	$r = 1$ means one sinusoid cycle per N samples
					\4	$r > 1$ means $r$ cycles per N samples
					\4	Periodic conjugate symmetric sequence
					\4	$x_{pcs}[n] = \frac{1}{2}(x_f[n] + x_f^*[N-n]), 1 \le n \le N$
			\2	Basic Sequences
				\3	Unit Sample sequence (delta)
					\4	$\delta[n] = \begin{cases}  1 & n = 0 \\ 0 & n \ne 0 \end{cases}$
					\4	$x[n] = \sum_{k = -\infty}^{\infty} x[k]\delta[n - k]$
				\3	Unit Step sequence (Heaviside)
					\4	$u[n] = \begin{cases} 1 & n \ge 0 \\ 0 & n \le 0 \end{cases}$
					\4	$\delta[n] = u[n] - u[n-1]$
					\4	$u[n] = \sum_{k = -\infty}^n \delta[k]$
				\3	Exponential Sequence (Eigenfunctions)
					\4	$x[n] = c\alpha^n$
					\4	Converge if $|\alpha| < 1$

			\2	Sampling
				\3	Downsampling
					\4	Discarding samples
					\4	Decimation
					\4	$x_d[n] = x[Mn]$

				\3	Upsampling
					\4	Adding more samples
					\4	Interpolation
					\4	$x_u[n] = \begin{cases}	x[n/L] & n = 0, \pm L, \pm 2L...\\
																				0 & \text{otherwise} \end{cases}$
					\4	Not inverse of downsampling
		
		\1	Complex Numbers
			\2	Rectangular form
				\3	$x = x_r + jx_i$
				\3	$|x| = \sqrt{x_r^2 + x_i^2}$
				\3	Phase: $\angle x = \theta = arctan(\frac{x_i}{x_r})$
			\2	Polar form
				\3	$x = |x|exp(j\theta) = |x|(cos(\theta) + jsin(\theta)$	
			\2	Addition
				\3	$(a + jb) + (c + jd) = (a + c) + j(b + d)$
			\2	Multiplication
				\3	$(a + jb)(c + jd) = (ac - bd) + j(bc + ad)$
				\3	$r_1exp(j\theta_1)r_2exp(j\theta_2) = r_1r_2exp(j(\theta_1 + \theta_2))$
				\3	Magnitudes multiply, phases add
				\3	Phase repeats modulo $2\pi$
			\2	Complex Conjugate
				\3	$x^* = x_r - jx_i = |x|exp(j(-\theta))$
				\3	$x + x^* = 2x_r$
				\3	$xx^* = |x|^2$
			
		\1	Discrete-time systems
			\2	$y[n] = f(x[n])~\forall~n$
			\2	Linearity
				\3	$(\alpha x_1[n] + \beta x_2[n]) \rightarrow (\alpha y_1[n] + \beta y_2[n])$
			\2	Shift-Invariance
				\3	$x_1[n-n_0] \rightarrow y_1[n-n_0]$
			\2	Causality
				\3	$x_1[n] = x_2[n]~\forall~n < N \rightarrow y_1[n] = y_2[n]~\forall~n < N$
				\3	Only depends on past and current inputs
			\2	Moving Average Filter (Smooth out)
				\3	$y[n] = \frac{1}{M} \sum_{k=0}^{M-1}x[n-k]$
			\2	Accumulator
				\3	$y[n] = \sum_{l = -\infty}^nx[l] = y[n-1] + x[n]$
			\2	Teager Energy operator
				\3	$y[n] = x^2[n] - x[n-1]x[n+1]$
		
		\1	Impulse Response
			\2	$x[n] = \delta[n] \rightarrow y[n] = h[n]$
			\2	Linear Time-Invariant (LTI) System completely specified by $h[n]$
			\2	For LTI Systems $y[n] = x[n] * h[n]$
				\3	$y[n] = \sum_{k = -\infty}^{\infty} x[k]h[n-k]$
				\3	$x[n] * h[n] = h[n] * x[n]$
				\3	$(x[n]*h[n])*y[n] = x[n]*(h[n]*y[n])$
				\3	$h[n]*(x[n] + y[n]) = h[n]*x[n] + h[n]*y[n]$
				\3	Convolving sequence of length $N$ and $M$ results in $N + M - 1$ non-zero elements
			\2	Cascaded Systems
				\3	$h_1[n]$ into $h_2[n]$ is $h[n] = h_1[n] * h_2[n]$
			\2	Inverse systems
				\3	$h_1[n] * h_2[n] = \delta[n]$
			\2	Parallel systems
				\3	$h[n] = h_1[n] + h_2[n]$
		
		\1	BIBO Stability
			\2	Bounded-Input, Bounded-Output stability
			\2	$|x[n]| < A~\forall~n \rightarrow |y[n]| < B~\forall~n$
			\2	$\sum_{n=-\infty}^{\infty} |h[n]| < \infty$
			\2	$|H(exp(j\omega)| \le \sum_{n=-\infty}^{\infty} |h[n]|$
			\2	$|H(exp(j\omega)| < \infty \rightarrow$ BIBO stable (and existence of DTFT)
			\2	FIR systems are always stable
				\3	All-sero systems
			\2	IIR systems depending can be stable or unstable

		\1	LCCDE
			\2	Linear Constant-Coefficient Difference Equations
			\2	$\sum_{k=0}^Nd_ky[n-k] = \sum_{k=0}^Mp_kx[n-k]$
			\2	Order = max(N,M)
			\2	$y[n] = -\sum_{k=1}^N\frac{d_k}{d_0}y[n-k] + \sum_{k=0}^M\frac{p_k}{d_0}x[n-k]$
			\2	$y[n] = y_c[n] + y_p[n]$
				\3	Complementary solution $y_c[n]$
					\4	$\sum_{k=0}^N d_ky_c[n-k] = 0$
					\4	$y_c[n] = \lambda^n \rightarrow \sum_{k=0}^Nd_k\lambda^{N-k} = 0$
					\4	Find roots of the equation $\lambda_1,\lambda_2,...$
					\4	$y_c[n] = \alpha_1\lambda_1^n + \alpha_2\lambda_2^n + ...	$
					\4	$\alpha$'s must match intial conditions
					\4	Repeated roots: $y_c[n] = \alpha_1\lambda_r^n + \alpha_2n\lambda_r^n + \alpha_3n^2\lambda_r^n$
					\4	BIBO Stability: $|\lambda| < 1$
				\3	Particular solution $y_p[n]$
					\4	Dependent on forcing function $x[n]$

		\1	z-transform
			\2	$X(z) = \sum_{n=-\infty}^{\infty}x[n]z^{-n}$
				\3	$z = r exp(j\omega)$
				\3	$r = 1, z = exp(j\omega) \rightarrow DTFT$
				\3	Z-transform is a generalization of the DTFT
				\3	DTFT is the z-transform evaluated on unit circle
			\2	Region of Convergence (ROC)
				\3	$X(z) \sum_{n=-\infty}^{\infty} x[n]z^{-n}$ Converges
				\3	ROC is region of complex plane where a particular $X(z)$ converges
				\3	ROC always defined in terms of $|z|$ (annulus in complex plane)
				\3	If ROC includes unit circle, then the sequence has a DTFT (finite energy)
				\3	Same $X(z)$ can desribe different sequences with different ROCs
				\3	Total ROC is the intersection of individual ROCs
					\4	If there is no intersection, then z-transform doesn't exist
			\2	Factored Z-transforms (Partial Fraction Expansion)
				\3	$X(z) = \frac{z^{-M}p_0 \prod_{m=1}^M (z - \beta_m)}{z^{-N}\prod_{n=1}^N(z - \alpha_n)}$
				\3	$\beta_m$ are the zeros of $X(z) = 0$
				\3	$\alpha_n$ are the poles of $X(z) = \infty$
				\3	Method of Weighted Residuals (Cauchy Reside Theorem applied)
					\4	$H(z) = \frac{1 + 2z^{-1}}{1 + 0.4z^{-1} - 0.12z^{-2}}$
					\4	$H(z) = \frac{\beta_1}{1 + 0.6z^{-1}} + \frac{\beta_2}{1 - 0.2z^{-1}}$
					\4	$\beta_1 = (1 + 0.6z^{-1})H(z)|_{z = -0.6} = -1.75$
					\4	$\beta_2 = (1 - 0.2z^{-2})H(z)|_{z = 0.2} = 2.75$
			\2	Transfer Function
				\3	$H(z) := \frac{Y(z)}{X(z)} = \frac{\sum_{k=0}^M p_k z^{-k}}{\sum_{n=0}^N d_k z^{-k}}$
				\3	$Y_c(z) = \frac{\alpha_1}{1 - \lambda_iz^{-1}} + ...$
					\4	$\lambda_i^n$ corresponds to a pole of $Y_c(z)$
					\4	LCCDE solutions are right sided $\rightarrow$ ROC has $|z| > |\lambda_i|$
		
		\1	Discrete Time Fourier Transform
			\2	$X(exp(j\omega)) = \sum_{n=-\infty}^{\infty} x[n] exp(-j\omega n)$
			\2	$X(exp(j(\omega + 2\pi))) = X(exp(j\omega))$ Periodic with period $2\pi$ in $\omega$
			\2	$x[n] = \frac{1}{2\pi}\int_{-\pi}^{\pi}X(exp(j\omega))e^{j\omega n}d\omega$
			\2	$y[n] = x[n] * h[n] \rightarrow Y(exp(j\omega)) = X(exp(j\omega))H(exp(j\omega))$
			\2	Energy Density Spectrum (EDS)
				\3	$S(exp(j\omega)) = |H(exp(j\omega))|^2$

		\1	Sampling
			\2	$x[n] = x_a(nT)$
				\3	Sampling Period $T$
				\3	Sampling rate $\Omega_{s} = \frac{2\pi}{T}$
			\2	$X(exp(j\omega)) = X_a(j\Omega)|_{\Omega T = \omega}$
			\2	$X(exp(j\omega)) = \frac{1}{T}\sum_{k = -\infty}^{\infty} 
					X_a(j(\frac{\omega}{T} - k \frac{2\pi}{T}))$
			\2	$\Omega_s \ge 2\Omega_{max}$ (Nyquist Sampling Theorem)
		
		\1	Discrete Fourier Transform (DFT)
			\2	$X[k] = \sum_{n=0}^{N-1} x[n] exp(-j\frac{2\pi k n}{N}$	
			\2	$X[k] = X(exp(j \omega )) |_{\omega = \frac{2\pi k}{N}}$	
				\3	DFT is sampled DTFT
			\2	$X[k] = \sum_{n=0}^{N-1} x[n] W_N^{kn}$
				\3	$W_N = exp(-j \frac{2\pi}{N})$
			\2	$x[n] = \frac{1}{N}\sum_{k=0}^{N-1}X[k] W_N^{-nk}$
			\2	Modulo-N indexing
				\3	$x[<x - n_0>_N] = \begin{cases}
																x[n - n_0] & n_0 \le n < N \\
																x[N + n - n_0] & 0 \le n < n_0
															\end{cases}$
				\3	$x[<x - n_0>_N] \rightarrow W_N^{kn_0}X[k]$
				\3	$W_N^{-k_0n}x[n] \rightarrow X[<k - k_0>_N]$
			\2	Circular Convolution
				\3	Because of modulo-N indexing, DFT convolution is circular convolution
				\3	Equivalent to time aliased linear convolution
				\3	Convolve $L-pt$ with $M-pt$, must have $N \ge L + M - 1$ 
					\4	Pad sequences until both are length $N$
					\4	Then convolve / IDFT
	\end{outline}
\end{document}


















