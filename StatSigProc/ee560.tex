\documentclass[14pt]{extarticle}
\usepackage{researchPaper}
\usepackage{outlines}

\title{EE 560 Cheat Sheet}
\begin{document}
	\maketitle
	
	\begin{outline}		
		\1	Basic Definitions
			\2	$P[A] = lim_{n \rightarrow \infty} \frac{n(A)}{n}$
			\2	$P[A] = \frac{N_A}{N}$
			\2	Axioms of Probability
				\3	$P[A] \ge 0$
				\3	$P[\Omega] = 1$
				\3	$P[\bigcup_{i=1}^{\infty} A_i] = \sum_{i=1}^{\infty} P[A_i]$
					\4	$A_i \cap A_j = \emptyset~\forall~i,j~i \ne j$
			\2	Probability Space
				\3	$\Omega$ The Sample Space:	Set of all possible outcomes of an experiment
					\4	Given a set with $N$ elmentary outcomes, there can be $2^N$ events in F
					\4	$2^N$ field of events is called the power set
				\3	$F$ the field of events.  Subsets of sample space which are assigned probabilities
					\4	$\emptyset \in F$
					\4	$\Omega \in F$
					\4	$A \in F$ and $B \in F \rightarrow A\cup B \in F$ and $A \cap B \in F$
					\4	$A \in F \rightarrow \bar{A} \in F$
				\3	$P$ A probability function which assigns a real number to each event in F
		\1	Set Theory
			\2	Set is a collection of objects, with no repetition
			\2	Set membership
				\3	$a \in A$ a is in the set A
				\3	$d \notin A$ d is not in the set A
				\3	$\emptyset = \{\}$ is the empty or null set
			\2	Types of Sets
				\3	Universal set (or Sample Space) is the set of all elements
				\3	Subset
					\4	$A \subset B$ iff $x \in A \rightarrow x \in B$, $x \in B$ does not mean $x \in A$
					\4	$\emptyset \subset A \subset B \subset \Omega$
					\4	$A \subset B$ and $B \subset C \rightarrow A \subset C$
				\3	Equality between sets: $A = B$ iff $A \subset B$ and $B \subset A$
			\2	Set Operations
				\3	Union:	$x \in A \cup B$ iff $x \in A$ or $x \in B$
					\4	$A \cup B = B \cup A$
					\4	$(A \cup B) \cup C = A \cup (B \cup C)$
					\4	if $A \subset B \rightarrow A \cup B = B$
					\4	$A \cup A = A$
					\4	$A \cup \emptyset = A$
					\4	$A \cup \Omega = \Omega$
				\3	Intersection
					\4	$x \in A \cap B$ iff $x \in A$ and $x \in B$
					\4	Commutative and Associative
					\4	$A \cap (B \cup C) = (A \cap B) \cup (A \cap C)$
					\4	$A \subset B \rightarrow A \cap B = A$
					\4	$A \cap \emptyset = \emptyset$
					\4	$A \cap \Omega = A$
			\2	Mutual Exclusion: $A \cap B = \emptyset$
			\2	De-Morgan's Law
				\3	$\bar{A \cup B} = \bar{A} \cap \bar{B}$
				\3	$\bar{A \cap B} = \bar{A} \cup \bar{B}$
			\2	Inclusion-Exclusion
				\3	Size of a union of sets can be written as:
					\4	Sum of Individual sets
					\4	Minus sum over all pairs of sizes of intersection
					\4	plus the sum over all triples of the sizes of their intersections
			\2	Probability Results
				\3	$P[A] = P[A \cup \emptyset] \rightarrow P[\emptyset] = 0$
				\3	$P[A] = 1 - P[\bar{A}] \le 1$
				\3	$A \subset B \rightarrow P[A] \le P[B]$
				\3	$P[A \cup B] = P[A] + P[B] - P[A \cap B]$

	\end{outline}
\end{document}


















