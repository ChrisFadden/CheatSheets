\documentclass[14pt]{extarticle}
\usepackage{researchPaper}
\usepackage{outlines}

\title{EE 453 Cheat Sheet 2}
\begin{document}
	\maketitle
	
	\begin{outline}		
		\1	Basic Filer Design
			\2	For LTI systems, completely characterized by impulse response $h[n]$
			\2	FIR
				\3	No feedback
				\3	No poles, only zeros
			\2	IIR
				\3	Feedback
				\3	poles, and often zeros
			\2	Cutoff Frequency $\omega_c$
				\3	$|H(exp(j\omega_c))|^2 = \frac{1}{2}max\{|H(exp(j\omega))|^2\}$
				\3	$H_c = H_{max} - 3~dB$
		\1	Filter Properties
			\2	$Y(e^{j\omega}) = X(e^{j\omega})H(e^{j\omega})$
			\2	$Y(e^{j\omega}) = |X(e^{j\omega})||H(e^{j\omega})|e^{j(\theta_X(\omega) + \theta_H(\omega))}$
				\3	Magnitude response = $|H(e^{j\omega})|$
				\3	Phase Response = $\theta_H(\omega)$
				\3	Phase Response very important, desired $\theta_H(\omega) = 0$
			\2	Linear Phase Systems
				\3	Zero phase preferred, but non causal
				\3	Linear phase is a compromise
				\3	$\theta_{H_L} (\omega) = -L \omega$
			\2 Types of Linear Phase FIR Filters
				\3	Type   I:	Odd-length Symmetric
				\3	Type  II:	Even-length Symmetric
					\4	Zeros at $z = -1$ or $H(e^{j\pi})$
					\4	Can be used for LPF, or bandpass
				\3	Type III:	Odd-length Anti-symmetric
				\3	Type  IV:	Even-length Anti-symmetric
		\1	Filter Design
			\2	Ideal filter requirements (magnitude)
				\3	Passband:	gain = 1
				\3	Stopband:	gain = 0
			\2	Least Squares Filter Design
				\3	Given $H_d(e^{j\omega})$, what is the best finite $h_t[n]$ approximation
				\3	$h_t[n] = \frac{1}{2\pi}\int_{-\pi}^{\pi}H_d(e^{j\omega}) e^{j\omega n} d\omega~~-M \le n \le M$
		\1	Windowing
			\2	Window parameters
				\3	Main lobe width $\Delta \omega_M$
					\4	narrower main lobe means sharper transition
					\4	narrow main lobe means high SLL
					\4	high SLL means more ripple
				\3	Passband cutoff freq $\omega_p$ Passband ripple $\delta_p$
				\3	Stopband cutoff freq $\omega_s$ Stopband ripple $\delta_s$
				\3	Center freq $\omega_c = \frac{\omega_p + \omega_s}{2}$
				\3	Transition bandwidth $\Delta \omega = \omega_s - \omega_p$
				\3	Peak Side Lobe Level (SLL)
				\3	Stopband attenuation $\alpha_s = 20 log (\frac{1}{\delta_s})$
			\2	Types of Windows ($-M \le n \le M$)
				\3	Rectangular:	$w[n] = 1$
					\4	Main Lobe:	$\frac{4\pi}{2M + 1}$
					\4	Stopband attenuation (dB): 21
					\4	Transition BW ($\Delta \omega$):	$\frac{0.92\pi}{M}$
				\3	Hann (Hanning):	$w[n] = 0.5 + 0.5 cos(\frac{2\pi n}{2M + 1})$
					\4	Main Lobe:	$\frac{8\pi}{2M + 1}$
					\4	Attenuation:	44
					\4	BW:	$\frac{3.11 \pi}{M}$
				\3	Hamming:	$w[n] = 0.54 + 0.46cos(\frac{2\pi n}{2M + 1})$
					\4	Main Lobe:	$\frac{8\pi}{2M + 1}$
					\4	Attenuation:	54
					\4	BW:	$\frac{3.32 \pi}{M}$
				\3	Blackman:	$w[n] = 0.42 + 0.5 cos(\frac{2\pi n}{2M + 1}) + 
															0.08 cos(\frac{2\pi(2n)}{2M + 1}$
					\4	Main Lobe:	$\frac{12 \pi}{2M + 1}$
					\4	Attenuation:	75
					\4	BW:	$\frac{5.56 \pi}{M}$
				\3	Kaiser:	$w[n] = \frac{I_0(\beta\sqrt{1 - (\frac{n}{N})^2})}{I_0(\beta)}$
			\2	Parks-McClellan
				\3	Optimization like Chebyshev for equiripple response
	\end{outline}
\end{document}


















