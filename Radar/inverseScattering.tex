\documentclass[14pt]{article}
\usepackage{researchPaper}

\title{Inverse Scattering Cheat Sheet}
\begin{document}
	\maketitle
	
	\section{Lippman Schwinger Equation}
		Considered here is a two dimensional and time-harmonic scattering problem.
	The background medium is inhomogeneous and decsribed by the following refractive
	index:	
		\begin{align}
			n_b(x) &= \frac{1}{\varepsilon_0}[\varepsilon(x) + j\frac{\sigma(x)}{\omega}]
		\end{align}
	The $x = (x_1,x_2) \in \mathbb{R}^2$, $\varepsilon_0 > 0$ is the background
	permittivity, $\varepsilon(x) \ge 1$ and $\sigma(x) \ge 0$ are the relative
	permittivity and conductivity of the medium, and $\omega$ is the temporal
	frequency.~\\
	
		The domain is made up of several open and connected $C^2$-domains $\Omega_i \subset \mathbb{R}^2$.
	This means that $\Omega_i \cap \Omega_j = \emptyset$ for $i \ne j$.  
	$\bigcup_{i=0}^N \bar{\Omega_i} = \mathbb{R}^2$.  $\Omega_i$ is bounded for $i \ne 0$.
	$n_b|_{\Omega_i} \in C^1(\bar{\Omega_i})~\forall~i$.  $n_b = n_0 \in \mathbb{C},
	~x \in \Omega_0$.  The space $\bar{\Omega_0}$ is made up of all the regions with
	permittivity equal to the background, i.e. $x \in \mathbb{R}^2~:~n_b(x) = n_0$.
	The domain is assumed to be non-magnetic.  The regions $\Omega_i$ need not be
	simply connected, and the boundaries $\partial \Omega_i$ may be either curves
	where a discontinuity of $n_b$ occurs, or boundaries of virtual domains that
	contain a scatterer.~\\	

		The Green's function is the solution of:
			\begin{align}
				\Delta_x G(x,y) + k^2n_b(x)G(x,y) = -\delta(x - y)
			\end{align}
	The wavenumber is given by $k = \frac{\omega}{c}$, where $c$ is the wave
	propagation speed in the background material.  In free space, the two dimensional
	Green's function is given by $\Phi(x,y) = \frac{j}{4}H_0^{(1)}(k_0|x - y|)$, 
	where $H_0^{(1)}$ is the Hankel function of the first kind or order 0.  The
	complete Green's function can be written $G(x,y) = \Phi(x,y) + u_b^s(x,y)$ 
	where $u_b^s(x,y)$ is the pertubation due to the inhomogeneous medium in
	$\mathbb{R}^2 \backslash \bar{\Omega_0}$.

		The differential form of the scattering problem can be stated as:
			\begin{align}
				&\Delta_xu(x,x_0) + k^2n(x) u(x,x_0) = -\delta(x - x_0) \\
				&u(x,x_0) = u^i(x,x_0) + u^s(x,x_0) \\
				&\lim_{r \rightarrow \infty} [\sqrt{r}(\pd{u^s}{r} - jk_0u^s)] = 0 
			\end{align}
		Where $u^i = G(x,x_0)$ is the incident field, $u^s(x,x_0)$ is the scattered
		field, and the last condition is the Sommerfeld radiation condition.  $x_0$
		is the location of a point source.  The differential form can be rewritten
		in the Lippmann-Schwinger Integral Equation form as:
			\begin{align}
				u^s(x,x_0) = -k_0^2 \int_{\mathbb{R}^2} G(x,y) m(y)u(y,x_0) dy
			\end{align}
		Where $m(y) := \frac{n_b(x) - n(x)}{n_0}$.
	
	\section{Constrast Source Inversion Method}
		The contrast source inversion method splits the inverse scattering problem
	up into two parts.  The first is defined on a region $T$ and the other on the
	boundary where field measurements are taken $\partial T = \Gamma$.  The 
	Lippman Schwinger equation for the total field is given by:
		\begin{align}
			u(x,x_0) = u^i(x,x_0) - k_0^2 \int_T G(x,y) m(y)u(y,x_0) dy
		\end{align}
	This integral equation is replaced by the system of equations, where
	$w(x,x_0) := m(x)u(x,x_0)$:
		\begin{align}
			w(x,x_0) = m(x)u^i(x,x_0) - m(x) k_0^2 \int_T G(x,z)w(z,x_0) dz~\forall~x \in T \\
			u^s(x,x_0) = -k_0^2 \int_T G(x,z)w(z,x_0) dz~\forall~x \in \Gamma 
		\end{align}
			
	The system of equations can be written as the minimization of a functional,
	if we first define the operators $\mathcal{G}^T : L_2(T) \rightarrow L_2(T)$
	and $\mathcal{G}^{\Gamma} : L_2(T) \rightarrow L_2(\Gamma)$
		\begin{align}
			[\mathcal{G}^Tf](x) := k_0^2 \int_T G(x,z) f(z) dz~\forall~x\in T \\
			[\mathcal{G}^{\Gamma}f](x) := k_0^2 \int_T G(x,z) f(z) dz~\forall~x\in \Gamma 
		\end{align}
	The functional to be minimized is then given by:
		\begin{align}
			F(w,m) = \frac{\norm{u^s - \mathcal{G}^{\Gamma}w}_{\Gamma}^2}{\norm{u^s}_{\Gamma}^2} +
							 \frac{\norm{mu^i - w - m\mathcal{G}^Tw}_T^2}{\norm{mu^i}_T^2}
		\end{align}
	\section{Born Approximation}
		The Lippmann Schwinger equation is a nonlinear problem, and is not trivially
		solved.  Therefore in order to make some problems tractable, a linear 
		approximation to it is made, the Born approximation.  The Born approximation
		replaces the total field by the incident field inside the integral, leading
		to a linear equation that can be solved.  However, this is only valid in the
		weakly scattering case, where there is little difference between the 
		scatterer and the background material.
		\begin{align}
			u(x,x_0) = u^i(x,x_0) - k_0^2 \int_T G(x,y) m(y)u^i(y,x_0) dy
		\end{align}
		For a given total field, incident field, and Green's function, this can
		be solved for the contrast difference $m(y)$ for all $y \in \mathbb{R}^2$,
		giving the shape and permittivity of the scatterer.

	\section{Time Reversal MUSIC (TR-MUSIC)}

			The basic idea behind time reversal is that any function that satisfies the
	wave equation with time moving in the positive direction will also satisfy the
	wave equation with time moving in the negative direction.  This allows the
	scattered fields to be focused back onto the scatterer, giving the location
	of the scatterer.~\\

		The equation these fields satisfy is the frequency domain Helmholtz equation
	\begin{align}
		\Delta u(x,\omega) + k^2 u(x,\omega) = 0
	\end{align}
	If $u(x,\omega)$ is a solution, its time reversed version, which corresponds
	to phase conjugation in the frequency domain, is also a solution.  Therefore,
	both $u(x,\omega)$ and $u^*(x,\omega)$ are solutions.~\\

		The incident field and scattered field due to the $j$th antenna in a system
	of $N$ antennas could be given by
	\begin{align}
		u_j^{(i)}(x,\omega) = G(x,x_j)e_j(\omega) \\
		u_j^{(s)}(x,\omega) = \sum_{m=1}^MG(x,x_m)\tau_m(\omega)G(x_m,x_j)e_j(\omega) \\
	\end{align}
	It is assumed there are $M$ scatterers at locations $x_m$, with $\tau_m(\omega)$ being the 
	scattering amplitude (reflection coefficient).  When each antenna is excited,
	the incident and scattered field at each point in space can be given as
	\begin{align}
		u^{(i)}(x,\omega) = \sum_{j=1}^N G(x,x_j)e_j(\omega) \\
		u^{(s)}(x,\omega) = \sum_{j=1}^N \sum_{m=1}^M G(x,x_m)\tau_m(\omega)G(x_m,x_j)e_j(\omega)
	\end{align}
	
	The multi-static response matrix (MSR matrix) can be represented as
	\begin{align}
		K = K_{ij} = \sum_{m=1}^M G(x_i,x_m)\tau_m(\omega)G(x_m,x_j)
	\end{align}
	Where $x_i$ is the location of the $i$th source, and $x_j$ the location of the
	$j$th receiver.
	The output voltage (scattered field) at each antenna can then be written as
	$v_l = \sum_{j=1}^N K_{lj}e_j = K e$.~\\
	
	The time reversal operator is given by
	\begin{align}
		T = K^H K
	\end{align}
	The rank of the time reversal matrix $T$ is the same as that of $K$ and is
	an extremely important quantity for identifying how many scatterers there are.
	The time reversal matrix $T$ for an $N$ element array embedded in a 
	three-dimensional homogeneous background and containing $1 < M \le N$ targets
	will have $rank~<~M$ if and only if the following conditions hold:
		\begin{itemize}
			\item	All of the targets are located in a single plane $P$ that is orthogonal
			to at least $N - M + 1$ lines connecting the different antenna elements
			\item	The plane $P$ is the perpendicular bisector of the $N - M + 1$ lines
			or all targets on $P$ are equidistant from every one of the $N - M + 1$ lines
		\end{itemize}
	
	It is usually assumed that the rank of $T$ is equal to $M$, and so there are
	exactly $M$ non-zero eigenvalues, corresponding to each of the scatterers.~\\

	Targets can be well resolved when the Green's function vectors form an 
	approximate orthogonal set betweent the scatterers.
	\begin{align}
		\sum_{n=1}^N G^*(x_n,x_i)G(x_n,x_j) = \sum_{n=1}^N g_i^*(n) g_j(n) \approx 0
	\end{align}
	
	The time reversal matrix can be written using this condensed notation as
	\begin{align}
		T &= (\sum_{i=1}^M \tau_i g_ig_i^T)^*\sum_{j=1}^M \tau_j g_jg_j^T \\
			&=	\sum_{i=1}^M\sum_{j=1}^M \Lambda_{ij}g_i^*g_j^T \\
		\Lambda_{ij} &= \tau_i^* \tau_j g_i^Hg_j \\
								 &= \tau_i^* \tau_j \sum_{n=1}^N g_i^*(n)g_j(n) \\
								 &= \tau_i^* \tau_j \braket{g_i}{g_j} 
	\end{align}

	\begin{align}
		H(x,x_m) = \sum_{n=1}^N G^*(x_n,x)G(x_n,x_m)
	\end{align}

	The coherent point spread function (CPSF) of the antenna array given by $H(x,x_m)$
	is an indication about how well the array can resolve a target.  
	It will have a maximum at the scatterer location $x_m$, and will decay in
	amplitude away from this point.  The transverse effective spatial region for 
	the CPSF is roughly a square having sides of length $\frac{z\lambda}{a}$.  
	$z$ is the distance from the array, $a = N \delta$ is the length of the 
	side of the array with $\delta$ being the inter element spacing.  When the
	targets are well-resolved (i.e. larger than the effective spatial region of the
	CPSF) then $\Lambda_{ij}$ becomes diagonolized to
	\begin{align}
		\Lambda_{ij} &= |\tau_i|^2\rho_i \delta_{ij}
		\rho_i &= H(x_i,x_i) = \braket{g_i}{g_i}
	\end{align}
	
	This gives the time reversal matrix a very simple form, where the eigenvectors
	are exactly the green's function vectors of the location of the scatterer.  
	This makes the time reversal operator a projection onto the subspace spanned
	by the green's function vectors associated with the scatterers.
	\begin{align}
		T &= \sum_{m=1}^M |\tau_m|^2 \rho_m  g_m^*g_m^T
		Tg_{m_0}^* &= |\tau_{m_0}|^2|\rho_{m_0}|^2 g_{m_0}^*
	\end{align}
		
	The green's function vectors not associated with scatterers must then be 
	projected to zero, $Tv_{noise} = 0$.  This allows the space to be broken down
	into a signal and noise subspace, $\mathbb{C}^n = \mathcal{S} \oplus  \mathcal{N}$.~\\

	For the non-resolved case, everything is more or less the same, but
	$\Lambda_{ij}$ is not a diagonal matrix.  Therefore, the time-reversal matrix
	still acts as a projection operator, but this time onto linear combinations
	of the green's function vectors, and onto the entire space spanned by them.

	\begin{align}
		T = \sum_{i=1}^M \sum_{j=1}^M \Lambda_{ij}g_i^* g_j^T
	\end{align}
	The decomposition into signal and noise is still valid.  The set of eigenvectors
	of the time reversal operator is given by $v_m$, with $m = 1,2,...,M,M+1,...,N$.
	The first $M$ eigenvalues are non-zero, and their eigenvectors span the signal
	subspace, while the other eigenvalues are zero, adn their eigenvectors span
	the noise subspace.  
	\begin{align}
		T v_m &= \lambda_m v_m~m = 1,2,...,M \\
		T v_m &= 0~m = M+1,...,N \\
		\braket{v_i}{v_j} &= \delta_{ij} \\
	\end{align}
	The projection operators onto the signal and noise subspaces are given by
	\begin{align}
		P_{\mathcal{S}} &= \sum_{m=1}^M v_mv_m^H \\
		P_{\mathcal{N}} &= \sum_{m=M+1}^N v_m v_m^H \\
		P_{\mathcal{S}} + &P_{\mathcal{N}} = I_N
	\end{align}

	The MUSIC algorithm is one way of using these eigenvalues and eigenvectors in
	order to identify the location of the scatterers.  This is achieved by projecting
	the green's function from each antenna onto the noise eigenvectors of the 
	time reversal matrix.  The inverse of this operation will result in peaks 
	at the scatterer's location.

	\begin{align}
		g_p(\omega) &= [G(x_1,x_p), G(x_2,x_p),...,G(x_N,x_p)]^T \\
		D(x_p) &= \frac{1}{\sum_{m_0 = M+1}^N |\braket{v_{m_0}^*}{g_p}|^2}
	\end{align}
	
	This denominator can be rewritten in terms of either the noise or signal 
	projection operators, and whichever subspace has smaller dimension can be used
	to identify the signals.
	\begin{align}
		\sum_{m_0 = M+1}^N |\braket{v_{m_0}^*}{g_p}|^2 &= \sum_{m_0 = M+1}^N |\braket{v_{m_0}}{g_p^*}|^2 \\
					&=	|P_{\mathcal{N}} g_p^*|^2 \\
					&= |[I - P_{\mathcal{S}}]g_p^*|^2 
	\end{align}
\end{document}


