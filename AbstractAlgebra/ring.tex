\documentclass[14pt]{extarticle}
\usepackage{researchPaper}
\usepackage{outlines}

%alistairsavage.ca/mat3143/notes/MAT3143-Ring_Theory.pdf

\def\Definition{{\color{blue} \textbf{Definition:} }}
\def\Theorem{{\color{red} \textbf{Theorem:} }}

\title{Ring Theory Cheat Sheet}
\begin{document}
	\maketitle

	\begin{outline}			
		\1	\Definition \textbf{Ring}
			\2	Set $R$ together with the binary operations $+$ and $\cdot$
			\2	Ring Axioms:
				\3	$R$ is an abelian group under $+$
					\4	$(a+b)+c = a+(b+c)~\forall~a,b,c \in R$
					\4	$a+b = b + a$
					\4	$a + 0 = a$
					\4	$a + (-a) = 0$
				\3	$R$ is a monoid under $\cdot$
					\4	$(a \cdot b) \cdot c = a \cdot (b \cdot c)~\forall~a,b,c \in R$
					\4	$a \cdot 1 = a$ and $1 \cdot a = a$
				\3	$\cdot$ is distributive w.r.t $+$
					\4	$a \cdot (b + c) = (a\cdot b) + (a\cdot c)~\forall~a,b,c \in R$
					\4	$(b + c) \cdot a = (b \cdot a) + (c \cdot a)~\forall~a,b,c \in R$
		\1	\Definition \textbf{Commutative Ring}
			\2	Ring with commutative multiplication $ab = ba$
		\1	\Definition \textbf{Nonassociative Ring}
			\2	Ring where $(ab)c \ne a(bc)$ for some elements in $R$
			\2	Multiplication is not necessarily associative
		\1	\Definition \textbf{Subring}
			\2	$S \subset R$, $\forall~x,y \in S$:
				\3	$(xy) \in S$
				\3	$(x + y) \in S$
				\3	$(-x) \in S$
		\1	\Definition \textbf{Ring Generator}
			\2	Any intersection of subrings of $R$ is a subring of $R$
			\2	$X \subset R$, intersections of all subrings of $R$ containing $X$ is
					a subring $S$ of $R$
			\2	$S$ is the smallest subring of $R$ containing $X$
				\3	Smallest means $S \subset T$ if $T$ is any other subring containing $X$
			\2	$S$ is the subring of $R$ generated by $X$
		\1	\Definition \textbf{Characteristic} char(R)
			\2	char(R) is the smallest $n \in \mathbb{N}$ st. $(1 + 1 + 1...) = 0$
			\2	Sum the multiplicative identity element to get the additive identity element
			\2	Kind of like Order of a group
		\1	\Definition \textbf{Center} Z(R)
			\2	$Z(R) \cong \{r \in R | rs = sr~\forall~s \in R\}$
			\2	Ring $R$ is commutative iff $Z(R) = R$
			\2	$Z(R)$ is a subring of $R$

		\1	\Definition \textbf{Idempotent}
			\2	$a \in R$ st. $a^2 = a$
			\2	Nilpotent $a^n = 0~n \in \mathbb{N}$
		\1	\Definition \textbf{Ring Isomorphism}
		\1	\Definition \textbf{Ideal}
		\1	Quotient Rings
		\1	Polynomial Rings
		\1	Integral Domains
	\end{outline}
\end{document}


