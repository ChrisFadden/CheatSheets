\documentclass[14pt]{extarticle}
\usepackage{researchPaper}
\usepackage{outlines}

%alistairsavage.ca/mat3143/notes/MAT3143-Ring_Theory.pdf

\def\Definition{{\color{blue} \textbf{Definition:} }}
\def\Theorem{{\color{red} \textbf{Theorem:} }}

\title{Ring Theory Cheat Sheet}
\begin{document}
	\maketitle

	\begin{outline}			
		\1	\Definition \textbf{Ring}
			\2	Set $R$ together with the binary operations $+$ and $\cdot$
			\2	Ring Axioms:
				\3	$R$ is an abelian group under $+$
					\4	$(a+b)+c = a+(b+c)~\forall~a,b,c \in R$
					\4	$a+b = b + a$
					\4	$a + 0 = a$
					\4	$a + (-a) = 0$
				\3	$R$ is a monoid under $\cdot$
					\4	$(a \cdot b) \cdot c = a \cdot (b \cdot c)~\forall~a,b,c \in R$
					\4	$a \cdot 1 = a$ and $1 \cdot a = a$
				\3	$\cdot$ is distributive w.r.t $+$
					\4	$a \cdot (b + c) = (a\cdot b) + (a\cdot c)~\forall~a,b,c \in R$
					\4	$(b + c) \cdot a = (b \cdot a) + (c \cdot a)~\forall~a,b,c \in R$
		\1	\Definition \textbf{Commutative Ring}
			\2	Ring with commutative multiplication $ab = ba$
		\1	\Definition \textbf{Nonassociative Ring}
			\2	Ring where $(ab)c \ne a(bc)$ for some elements in $R$
			\2	Multiplication is not necessarily associative
		\1	\Definition \textbf{Divisors of zero}
			\2	$ab = 0$ for some $a,b \in R$
		\1	\Theorem	\textbf{Cancellation Theorem}
			\2	Cancellelation property: $ab = ac$ or $ba = ca$ $\Rightarrow$ $b = c$
			\2	{\color{purple} A ring has cancellation property iff it has no divisors of zero}
		\1	\Definition \textbf{Subring}
			\2	$S \subset R$, $\forall~x,y \in S$:
				\3	$(xy) \in S$
				\3	$(x + y) \in S$
				\3	$(-x) \in S$
		\1	\Definition \textbf{Ring Generator}
			\2	Any intersection of subrings of $R$ is a subring of $R$
			\2	$X \subset R$, intersections of all subrings of $R$ containing $X$ is
					a subring $S$ of $R$
			\2	$S$ is the smallest subring of $R$ containing $X$
				\3	Smallest means $S \subset T$ if $T$ is any other subring containing $X$
			\2	$S$ is the subring of $R$ generated by $X$
		\1	\Definition \textbf{Characteristic} char(R)
			\2	char(R) is the smallest $n \in \mathbb{N}$ st. $(1 + 1 + 1...) = 0$
			\2	Sum the multiplicative identity element to get the additive identity element
			\2	Kind of like Order of a group
		\1	\Definition \textbf{Center} Z(R)
			\2	$Z(R) \equiv \{r \in R | rs = sr~\forall~s \in R\}$
			\2	Ring $R$ is commutative iff $Z(R) = R$
			\2	$Z(R)$ is a subring of $R$
		\1	\Definition \textbf{Idempotent}
			\2	$a \in R$ st. $a^2 = a$
			\2	Nilpotent $a^n = 0~n \in \mathbb{N}$
		\1	\Definition \textbf{Ideal}
			\2	$(I,+) \le (R,+)$
			\2	$\forall~x \in I, \forall~r \in R~:~x\cdot r \in I,~r\cdot x \in I$
			\2	Kind of like \textbf{Normal Subgroup}
		\1	\Definition \textbf{Principal Ideal}
			\2	$b \in R$
			\2	Left Principal Ideal:	$Rb = \{rb~:~\forall~r \in R\}$ \\
			\2	Right Principal Ideal:	$bR = \{br~:~\forall~r \in R\}$ \\	
			\2	For commutative rings, these are equivalent, and it is written as:
					The ideal \textbf{generated by b}: $\langle b \rangle$
			\2	Similar to \textbf{cosets}
		\1	\Definition \textbf{Prime Ideal}
			\2	$a,b \in R$, $I$ is an ideal
			\2	$ab \in I$, then $a \in I$ or $b \in I$
		\1	\Definition \textbf{Proper Ideal}
			\2	$I \ne R$	
		\1	\Definition \textbf{Ring Homomorphism}
			\2	$f~:~R \rightarrow S$
			\2	$f(a + b) = f(a) + f(b)$
			\2	$f(ab) = f(a)f(b)$
			\2	$f(1_R) = 1_S$
			\2	$char(S)$ divides $char(R)$
		\1	\Theorem \textbf{Kernel of Ring Homomorphism}
			\2	$f$ is a homomorphism
			\2	$ker(f) = \{r \in R~:~f(r) = 0\}$
			\2	{\color{purple} $ker(f)$ is an ideal of $R$}
		\1	\Definition \textbf{Ring Isomorphism}
			\2	$f~:~R \rightarrow S$
			\2	$f$ is a homomorphism	
			\2	$f$ is bijective
			\2	$R \cong S$
		\1	\Definition \textbf{Cosets of Rings}
			\2	$R$ is a ring, $I$ is an ideal of $R$
			\2	$I + r \equiv \{i + r~:~i \in I\}$
			\2	$I + a$ is the \textbf{coset} of $I$ in $R$	
		\1	\Definition \textbf{Quotient Rings} $R/I$
			\2	Group of cosets $R/I$ with coset addition and coset multiplication
			\2	$I$ is an ideal
		\1	\Definition \textbf{Natural Ring Homomorphism}
			\2	$f~:~R \rightarrow R/I$, $f$ is a homomorphism
			\2	{\color{purple} $f(r) = I + r$}
	
		\1	\Definition \textbf{Integral Domains}
			\2	$R$ is a commutative ring
			\2	If $ab = 0$ then $a = 0$ or $b = 0$
			\2	if $a \ne = 0$, $ab = ac \Rightarrow b = c$
		\1	\Theorem	\textbf{Integral Domain: additive order}
			\2	{\color{purple} All nonzero elements of an integral domain have the same additive order (characteristic)}
			\2	{\color{purple} In integral domain with nonzero characteristic, the characteristic is prime}
		
		\1	\Definition \textbf{Polynomial Rings}

	\end{outline}
\end{document}


